\section{مقایسه مدل‌های توسعه}
در فرایند توسعه نرم‌افزار، انتخاب مدل مناسب توسعه نقش مهمی در موفقیت پروژه دارد. هر مدل توسعه چرخه عمر نرم‌افزار (SDLC) دارای ساختار، رویکرد و ویژگی‌های خاصی است که بر نحوه برنامه‌ریزی، طراحی، پیاده‌سازی و تحویل محصول تأثیر می‌گذارد. مدل‌های مختلف مانند \lr{Waterfall}، \lr{Incremental}، \lr{Spiral} و \lr{Agile} هر یک مزایا و محدودیت‌های مخصوص به خود را دارند و بسته به نوع پروژه، اهداف سازمان و پویایی نیازمندی‌ها انتخاب می‌شوند. در این بخش، این مدل‌ها از نظر ساختار، انعطاف‌پذیری، مدیریت ریسک، مشارکت مشتری و کیفیت نهایی مورد مقایسه و تحلیل قرار می‌گیرند.

\subsection{\lr{Waterfall SDLC Model}}
\textbf{مزایا:}
\begin{itemize}
    \item \textbf{سادگی:} ماهیت خطی و ترتیبی بودن این مدل منجر به فهم و اجرای آسان آن می‌گردد.
    \item \textbf{مستندسازی شفاف:} هر مرحله مستندات مربوط به خود را دارد که منجر به پیگیری آسان پیشرفت و مدیریت آن می‌گردد.
    \item \textbf{نیازمندی‌های پایدار:} برای پروژه‌هایی که نیازمندی‌های آن در ابتدا به صورت پایدار و واضح تعریف شده، مناسب می‌باشد.
    \item \textbf{قابلیت پیش‌بینی:} ماهیت ساختارمند بودن آن منجر به پیش‌بینی دقیق‌تر از نظر زمان‌بندی و نتایج نهایی می‌گردد.
\end{itemize}

\textbf{معایب:}
\begin{itemize}
    \item \textbf{انعطاف‌ناپذیری:} این مدل پس از تکمیل یک مرحله غیرمنعطف بوده و تطبیق تغییرات چالش‌برانگیز است.
    \item \textbf{تست دیرهنگام:} تست پس از مرحله پیاده‌سازی انجام می‌شود؛ بنابراین ممکن است خطاها تا اواخر فرایند کشف نشوند.
    \item \textbf{مشارکت محدود مشتری:} مشتریان عمدتاً در مرحله ابتدایی درگیر هستند و تغییرات قابل توجه نمی‌توانند به راحتی در مراحل بعدی اعمال شوند.
    \item \textbf{فاقد نمونه‌سازی اولیه:} این مدل فاقد نمونه اولیه است که در پروژه‌هایی با نیاز به بازخورد کاربر، نقطه ضعف محسوب می‌شود.
\end{itemize}

\subsection{\lr{Incremental SDLC Model}}
\textbf{مزایا:}
\begin{itemize}
    \item \textbf{نتایج زودهنگام و ملموس:} ذی‌نفعان در مراحل اولیه نتایج قابل مشاهده‌ای دارند، زیرا هر بخش عملکردی ارائه می‌دهد.
    \item \textbf{انعطاف‌پذیری و سازگاری:} امکان اعمال آسان تغییرات در هر بخش فراهم است.
    \item \textbf{مدیریت ریسک:} تقسیم فرآیند به بخش‌های کوچک‌تر موجب کاهش ریسک و تشخیص زودهنگام مشکلات می‌شود.
    \item \textbf{زمان سریع‌تر برای ورود به بازار:} محصول نهایی سریع‌تر به بازار وارد می‌شود که در محیط‌های پویا ارزشمند است.
\end{itemize}

\textbf{معایب:}
\begin{itemize}
    \item \textbf{افزایش پیچیدگی:} با اضافه شدن بخش‌ها، مدیریت و نگهداری دشوارتر می‌شود.
    \item \textbf{هزینه‌های بالا:} هر بخش نیاز به طراحی، کدنویسی، آزمایش و استقرار دارد و این هزینه کلی را افزایش می‌دهد.
    \item \textbf{دشواری در پیگیری پیشرفت:} توسعه همزمان چند بخش باعث دشواری در کنترل پیشرفت کلی پروژه می‌شود.
\end{itemize}

\subsection{\lr{Spiral SDLC Model}}
\textbf{مزایا:}
\begin{itemize}
    \item \textbf{کاهش ریسک:} تمرکز بر تحلیل و مدیریت ریسک احتمال شکست پروژه را کاهش می‌دهد.
    \item \textbf{انعطاف‌پذیری در نیازمندی‌ها:} تغییر نیازمندی‌ها در هر مرحله ممکن است.
    \item \textbf{محصولات باکیفیت:} ارزیابی و آزمایش مداوم موجب کیفیت بالاتر نرم‌افزار می‌شود.
    \item \textbf{مشارکت مشتری:} مشتریان در طول فرایند مشارکت دارند و بازخورد آن‌ها باعث تطبیق بهتر محصول می‌شود.
\end{itemize}

\textbf{معایب:}
\begin{itemize}
    \item \textbf{پیچیدگی:} برای پروژه‌های کوچک و کم‌ریسک مناسب نیست.
    \item \textbf{تخصص بالا:} تحلیل ریسک به تخصص خاصی نیاز دارد.
    \item \textbf{زمان نامشخص:} تعداد چرخه‌ها در ابتدا مشخص نیست و تخمین زمان دشوار است.
\end{itemize}

\subsection{\lr{Agile SDLC Model}}
\textbf{مزایا:}
\begin{itemize}
    \item \textbf{انعطاف‌پذیری و سازگاری:} تیم می‌تواند به سرعت با تغییرات نیازمندی سازگار شود.
    \item \textbf{رضایت مشتری:} درگیری مداوم مشتری تضمین‌کننده تطابق محصول با انتظارات اوست.
    \item \textbf{تحویل زودهنگام و قابل پیش‌بینی:} تکرارهای منظم باعث تحویل تدریجی و قابل مشاهده می‌شود.
    \item \textbf{کیفیت بهبود یافته:} تست و ادغام مداوم کیفیت نهایی را افزایش می‌دهد.
\end{itemize}

\textbf{معایب:}
\begin{itemize}
    \item \textbf{وابستگی به مشتری:} بازخورد و مشارکت مداوم مشتری ضروری است.
    \item \textbf{مقیاس‌پذیری دشوار:} توسعه پروژه‌های بزرگ با این روش سخت‌تر است.
    \item \textbf{افزایش سربار:} نیاز به هماهنگی، ارتباط و برنامه‌ریزی مداوم دارد.
\end{itemize}

\subsection{معیارهای انتخاب مدل مناسب}
انتخاب مدل مناسب \lr{SDLC} برای پروژه تصمیمی حیاتی است که بر موفقیت پروژه تأثیر زیادی دارد. این انتخاب بر اساس عوامل زیر صورت می‌گیرد:

\begin{itemize}
    \item \textbf{براساس نیازمندی‌های پروژه:}  
    برای نیازمندی‌های واضح، مدل آبشاری مناسب است؛ اما در نیازمندی‌های در حال تغییر، مدل‌های چابک یا تکرارشونده توصیه می‌شوند.
    
    \item \textbf{براساس اندازه و پیچیدگی پروژه:}  
    پروژه‌های کوچک معمولاً از مدل آبشاری بهره می‌برند، در حالی‌که پروژه‌های بزرگ و پیچیده از چابک یا اسکرام.
    
    \item \textbf{براساس انعطاف‌پذیری:}  
    اگر پروژه به سازگاری نیاز دارد، مدل‌های چابک یا تکرارشونده مناسب هستند.
    
    \item \textbf{براساس مشارکت مشتری:}  
    در پروژه‌هایی با بازخورد زیاد مشتری، مدل‌های چابک بهتر عمل می‌کنند؛ در غیر این صورت مدل آبشاری کافی است.
    
    \item \textbf{براساس تحمل ریسک:}  
    در پروژه‌های با ریسک بالا، مدل‌های مارپیچی یا چابک توصیه می‌شوند.
    
    \item \textbf{براساس محدودیت زمانی:}  
    پروژه‌های دارای مهلت دقیق بهتر است با مدل آبشاری انجام شوند، در حالی‌که پروژه‌های منعطف از مدل‌های چابک بهره می‌برند.
    
    \item \textbf{براساس تخصص تیم:}  
    تیم‌های چندوظیفه‌ای برای مدل چابک مناسب‌اند، در حالی‌که تیم‌های با نقش‌های تخصصی‌تر برای آبشاری.
\end{itemize}
