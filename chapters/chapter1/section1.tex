مهندسی نرم‌افزار شاخه‌ای از مهندسی است که به مطالعه، طراحی، توسعه، آزمون و نگهداری سیستم‌های نرم‌افزاری می‌پردازد. هدف اصلی آن، ایجاد نرم‌افزارهایی با کیفیت بالا، قابل اعتماد، کارایی مناسب، مقرون به صرفه و نگهداری آسان است. برخلاف برنامه‌نویسی صرف، مهندسی نرم‌افزار بر اصول علمی، متدولوژی‌های ساختاریافته، و ابزارهای مهندسی برای مدیریت پیچیدگی پروژه‌های نرم‌افزاری بزرگ تمرکز دارد. با رشد سریع فناوری اطلاعات و افزایش نیاز به سیستم‌های نرم‌افزاری در حوزه‌های مختلف مانند بانکداری، آموزش، بهداشت و صنعت، مهندسی نرم‌افزار به یکی از حیاتی‌ترین رشته‌های فناوری تبدیل شده است. این علم تلاش می‌کند تا توسعه نرم‌افزار را از یک فعالیت هنری یا تجربی به یک فرآیند نظام‌مند و قابل تکرار تبدیل کند.\cite{Software-development-history}