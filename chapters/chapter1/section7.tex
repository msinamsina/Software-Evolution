\section{جمع‌بندی فصل}

در این فصل، سیر تحول مهندسی نرم‌افزار از روش‌های ابتدایی تا رویکردهای نوین بررسی شد. مهندسی نرم‌افزار با هدف ایجاد سامانه‌های قابل اعتماد و باکیفیت، از مدل‌های خطی و مستندسازی‌محور به مدل‌های تکرارشونده و چابک‌تر تکامل یافته است. بررسی تاریخچهٔ چرخهٔ عمر توسعهٔ نرم‌افزار (\lr{SDLC}) نشان داد که نیاز به انعطاف‌پذیری، بازخورد سریع و خودکارسازی، منجر به پیدایش رویکردهای مدرن شده است. 

به طور کلی، تکامل فرایندهای مهندسی نرم‌افزار بازتابی از حرکت مداوم صنعت به‌سوی چابکی، خودکارسازی و یادگیری مستمر است.
