چرخه عمر توسعه نرم‌افزار (\lr{Software Development Life Cycle}) مجموعه‌ای از مراحل منظم برای تولید، استقرار و نگهداری نرم‌افزار است که در طول تاریخ توسعه‌ی مهندسی نرم‌افزار تکامل یافته است.\cite{SDLC}

در دهه 1960، مدل کد و فیکس بدون ساختار مشخص به‌کار می‌رفت و مفهومی از چرخه عمر وجود نداشت. با رشد پروژه‌ها و پیچیدگی سیستم‌ها در دهه 1970، مدل آبشاری معرفی شد و برای نخستین‌بار مراحل SDLC به‌صورت خطی تعریف شدند: تحلیل، طراحی، پیاده‌سازی، تست و نگهداری.\cite{SDLC}

در دهه 1980، مدل‌های افزایشی و تکاملی مفهوم تکرارپذیری را وارد SDLC کردند. نرم‌افزار در چند چرخه‌ی کوچک توسعه می‌یافت و بازخورد کاربران باعث تکامل تدریجی محصول می‌شد.\cite{SDLC}

مدل مارپیچی در دهه 1990 با تمرکز بر مدیریت ریسک، SDLC را به فرآیندی پویا و تکرارشونده تبدیل کرد. در هر چرخه، برنامه‌ریزی، تحلیل ریسک، طراحی و ارزیابی انجام می‌شد.\cite{SDLC}

در نهایت، با ظهور چابک (Agile) و سپس DevOps در دهه 2000 به بعد، SDLC از رویکردهای سنگین و مستندسازی محور فاصله گرفت و به فرآیندی سریع، انعطاف‌پذیر و مبتنی بر بازخورد تبدیل شد. اکنون SDLC شامل فازهای پویا و پیوسته‌ای مانند برنامه‌ریزی، توسعه، تست خودکار، استقرار و نگهداری مستمر است که به بهبود مداوم نرم‌افزار و رضایت کاربر منجر می‌شود.\cite{SDLC}

\noindent \textbf{فازهای اصلی SDLC}


\begin{description}
  \item[برنامه‌ریزی (Planning):] در این مرحله اهداف پروژه، نیازمندی‌های کلی، منابع، بودجه و زمان‌بندی تعیین می‌شوند. تحلیل ریسک‌ها و تهیه طرح مدیریت پروژه نیز در این فاز انجام می‌گیرد.\cite{SDLC}

  \item[تحلیل نیازمندی‌ها (\lr{Requirement Analysis}):] تیم تحلیل، نیازهای کاربران و ذی‌نفعان را شناسایی و مستند می‌کند. خروجی این فاز، سند مشخصات نیازمندی‌های نرم‌افزار (\lr{SRS}) است.\cite{SDLC}

  \item[طراحی سیستم (Design):] ساختار کلی سیستم، معماری نرم‌افزار، طراحی پایگاه داده و رابط کاربری مشخص می‌شود. در این مرحله، مدل‌های \lr{UML} و دیاگرام‌های مختلف برای شفاف‌سازی طراحی استفاده می‌شوند.\cite{SDLC}

  \item[پیاده‌سازی (Implementation):] کدنویسی بر اساس طراحی انجام می‌شود. توسعه‌دهندگان از زبان‌ها، فریم‌ورک‌ها و ابزارهای مختلف برای تولید نرم‌افزار استفاده می‌کنند.\cite{SDLC}

  \item[تست (Testing):] در این فاز، نرم‌افزار از نظر عملکردی، امنیتی، سازگاری و کارایی مورد آزمون قرار می‌گیرد. هدف، شناسایی و رفع خطاها پیش از استقرار است.\cite{SDLC}

  \item[نگهداری (Maintenance):] پس از استقرار نرم‌افزار، ممکن است نیاز به اصلاح خطاها، افزودن قابلیت‌های جدید یا بهینه‌سازی عملکرد باشد. نگهداری مناسب، عمر مفید نرم‌افزار را افزایش می‌دهد و از افت کیفیت آن جلوگیری می‌کند.\cite{SDLC}
\end{description}