\section{مطالعه‌ی موردی}
تحول در فرایندهای توسعه نرم‌افزار در شرکت‌های بزرگ فناوری، نمونه‌ای بارز از انطباق سازمان‌ها با تغییرات سریع دنیای دیجیتال است. شرکت‌هایی مانند \lr{Microsoft} و \lr{Google} با بازنگری در مدل‌های سنتی توسعه و حرکت به‌سوی رویکردهای نوین مانند \lr{Agile}، \lr{DevOps} و \lr{SRE}، توانسته‌اند سرعت، کیفیت و نوآوری را به‌طور چشمگیری افزایش دهند. این تغییرات نه تنها ساختار تیم‌ها و چرخه‌های انتشار را دگرگون کرده، بلکه فرهنگ سازمانی و نحوه‌ی همکاری میان تیم‌های مختلف را نیز متحول ساخته است. در ادامه، دو نمونه از این تحولات در \lr{Microsoft} و \lr{Google} بررسی می‌شود.

\subsection{\lr{Microsoft}}
در دهه‌های ۱۹۸۰ و ۱۹۹۰، \lr{Microsoft} مانند بسیاری از شرکت‌های نرم‌افزاری از مدل آبشاری (\lr{Waterfall}) برای توسعهٔ نرم‌افزار استفاده می‌کرد. در این روش، مراحل تحلیل، طراحی، پیاده‌سازی و تست به‌صورت خطی انجام می‌شد و انتشار نسخه‌های جدید معمولاً سالی یک‌بار یا چند سال یک‌بار اتفاق می‌افتاد.

اما با رشد اینترنت و نیاز به به‌روزرسانی‌های سریع‌تر، این مدل ناکارآمد شد. از حدود سال ۲۰۱۰، \lr{Microsoft} به‌تدریج به سمت متدولوژی‌های \lr{Agile} و سپس \lr{DevOps} حرکت کرد.

در این تحول:
\begin{itemize}
    \item تیم‌ها به‌صورت کوچک‌تر و خودگردان سازماندهی شدند.
    \item فرایند انتشار از نسخه‌های بزرگ چندساله، به انتشار مداوم تغییر کرد.
    \item همکاری میان تیم‌های توسعه، تست و عملیات افزایش یافت تا چرخهٔ \lr{Build–Measure–Learn} سریع‌تر انجام شود.
\end{itemize}

\subsection{\lr{Google}}
\lr{Google} از همان ابتدای کار (دههٔ ۲۰۰۰) روش‌های متفاوتی برای توسعهٔ نرم‌افزار اتخاذ کرد. برخلاف مدل‌های سنتی، \lr{Google} از ابتدا بر مهندسی در مقیاس بزرگ، خودکارسازی و قابلیت اطمینان سرویس‌ها تمرکز داشت.

در \lr{Google}، تحول فرآیند توسعه حول مفهوم \lr{SRE (Site Reliability Engineering)} شکل گرفت؛ روشی که ترکیبی از مهندسی نرم‌افزار و عملیات سیستم است. ویژگی‌های این تحول شامل موارد زیر بود:
\begin{itemize}
    \item استفاده از \lr{Continuous Integration (CI)} و \lr{Continuous Deployment (CD)} از همان مراحل اولیه پروژه‌ها.
    \item فرهنگ «تست خودکار در همه‌چیز» و تحلیل داده‌های واقعی کاربران برای بهبود مستمر.
    \item تعریف شاخص‌های کمی مانند \lr{SLO (Service Level Objective)} و \lr{Error Budget} برای تصمیم‌گیری دربارهٔ انتشار نسخه‌ها.
    \item ایجاد فرهنگ «\lr{Blameless Postmortem}» برای یادگیری از خطاها بدون سرزنش افراد.
\end{itemize}
