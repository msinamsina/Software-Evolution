بازخورد نقش حیاتی در فرایند توسعه نرم‌افزار دارد. بدون دریافت بازخورد از کاربران، ذی‌نفعان یا اعضای تیم، نرم‌افزار نمی‌تواند با نیازهای واقعی محیط و کاربران هماهنگ شود. تکامل نرم‌افزار نتیجهٔ بازخوردهای پی‌درپی و اصلاحات مستمر است. در مهندسی نرم‌افزار مدرن، بازخورد از طریق آزمون‌های خودکار، بازبینی کد (\lr{Code Review})، و جلسات مرور عملکرد پروژه (\lr{Sprint Review}) جمع‌آوری می‌شود. این بازخوردها موجب بهبود کیفیت، افزایش رضایت مشتری و کاهش هزینه‌های بلندمدت می‌شوند.

