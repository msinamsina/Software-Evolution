\section{هر بخش از فایل PE چه مزایایی دارد و چه اطلاعاتی را در خود ذخیره می‌کند؟}

فایل‌های اجرایی در سیستم‌عامل ویندوز با قالب \en{PE (Portable Executable)} شناخته می‌شوند. این قالب، ساختار استاندارد تمام فایل‌های اجرایی مانند \en{.exe}، \en{.dll}، \en{.sys} و بسیاری از فایل‌های سیستم است. هدف از طراحی قالب \en{PE} ایجاد یک چارچوب قابل‌حمل میان نسخه‌های مختلف ویندوز است تا سیستم‌عامل بتواند برنامه‌ها را به‌صورت بهینه در حافظه بارگذاری، مدیریت و اجرا کند.

ساختار فایل \en{PE} از چندین بخش اصلی تشکیل شده است که هر کدام وظیفه‌ی خاصی دارند. این بخش‌ها اطلاعات ضروری مانند کد اجرایی، داده‌ها، منابع و تنظیمات بارگذاری را ذخیره می‌کنند. در ادامه، هر بخش فایل \en{PE} به همراه وظایف، مزایا و نوع اطلاعات ذخیره‌شده در آن توضیح داده می‌شود.

\subsection{\en{Header} (هدر فایل \en{PE})}
هدر فایل \en{PE} در ابتدای فایل قرار دارد و شامل اطلاعات کلی درباره‌ی ساختار، نسخه و نحوه‌ی بارگذاری برنامه در حافظه است. این بخش به سیستم‌عامل کمک می‌کند تا بداند فایل چگونه باید تفسیر و اجرا شود.

\textbf{زیرمجموعه‌های مهم هدر \en{PE} عبارت‌اند از:}
\begin{itemize}
    \item \textbf{\en{DOS Header}:} برای سازگاری با نسخه‌های قدیمی \en{DOS} استفاده می‌شود. اگر برنامه در محیط \en{DOS} اجرا شود، پیامی مانند \en{This program cannot be run in DOS mode} نمایش داده می‌شود. مزیت آن \textbf{سازگاری عقب‌رو} \en{(Backward Compatibility)} است.
    \item \textbf{\en{PE Signature}:} رشته‌ای با مقدار \code{PE\textbackslash0\textbackslash0} که آغاز ساختار واقعی \en{PE} را مشخص می‌کند.
    \item \textbf{\en{File Header}:} حاوی اطلاعات عمومی مانند نوع فایل (\en{exe} یا \en{dll})، تعداد بخش‌ها، تاریخ ساخت و اندازه‌ی داده‌ها است.
    \item \textbf{\en{Optional Header}:} شامل اطلاعات حیاتی برای بارگذاری در حافظه مانند آدرس نقطه‌ی ورود \en{(Entry Point)}، اندازه‌ی پشته و نوع پردازنده است.
\end{itemize}

\textbf{مزایا:}
\begin{itemize}
    \item فراهم‌کردن نقشه‌ی دقیق ساختار فایل برای \en{Loader}
    \item افزایش سازگاری با نسخه‌های مختلف ویندوز
    \item بهینه‌سازی فرآیند بارگذاری و تخصیص حافظه
\end{itemize}

\subsection{\en{.text}}
بخش \en{.text} شامل \textbf{کدهای اجرایی برنامه} (دستورالعمل‌های ماشین) است. این قسمت معمولاً به‌صورت فقط‌خواندنی در حافظه نگهداری می‌شود تا از تغییرات ناخواسته جلوگیری شود.

\textbf{اطلاعات موجود در این بخش:}
\begin{itemize}
    \item دستورالعمل‌های کامپایل‌شده
    \item آدرس و محل قرارگیری توابع
    \item داده‌های ثابت و ثابت‌های برنامه
\end{itemize}

\textbf{مزایا:}
\begin{itemize}
    \item افزایش امنیت از طریق جلوگیری از تغییر کدها در زمان اجرا
    \item صرفه‌جویی در حافظه به دلیل امکان اشتراک کد بین چند فرآیند
    \item تسهیل فرآیند اشکال‌زدایی \en{(Debugging)}
\end{itemize}

\subsection{\en{.data}}
این بخش شامل \textbf{داده‌های مقداردهی‌شده} \en{(Initialized Data)} است؛ یعنی متغیرهایی که در زمان کامپایل مقدار اولیه دارند. برای مثال، در زبان \en{C} اگر دستور \code{int a = 10;} وجود داشته باشد، مقدار آن در این بخش ذخیره می‌شود.

\textbf{مزایا:}
\begin{itemize}
    \item فراهم‌کردن دسترسی سریع به داده‌های مهم در زمان اجرا
    \item حفظ داده‌های پایدار در طول اجرای برنامه
    \item ساختاردهی مناسب حافظه و جلوگیری از تداخل متغیرها
\end{itemize}

\subsection{\en{.bss}}
بخش \en{.bss} برای \textbf{داده‌های بدون مقدار اولیه} \en{(Uninitialized Data)} استفاده می‌شود. این داده‌ها در زمان بارگذاری برنامه به‌صورت خودکار با مقدار صفر مقداردهی می‌شوند.

\textbf{نمونه:} در برنامه‌ای با دستور \code{int counter;}، متغیر \en{counter} در بخش \en{.bss} قرار می‌گیرد.

\textbf{مزایا:}
\begin{itemize}
    \item کاهش اندازه‌ی فایل اجرایی چون داده‌های بدون مقدار ذخیره نمی‌شوند
    \item تخصیص پویا و بهینه‌ی حافظه در زمان اجرا
    \item تفکیک واضح میان داده‌های مقداردهی‌شده و نشده
\end{itemize}

\subsection{\en{.rdata}}
بخش \en{.rdata} شامل \textbf{داده‌های فقط‌خواندنی} \en{(Read-Only Data)} است؛ مانند رشته‌های ثابت، جداول واردات و صادرات و برخی ساختارهای زمان اجرا.

\textbf{اطلاعات ذخیره‌شده در این بخش:}
\begin{itemize}
    \item نام توابع و کتابخانه‌های وارداتی
    \item داده‌های ثابت و اشاره‌گرهای ثابت
    \item اطلاعات موردنیاز زمان اجرا
\end{itemize}

\textbf{مزایا:}
\begin{itemize}
    \item جلوگیری از تغییر داده‌های حیاتی در زمان اجرا
    \item بهبود امنیت و پایداری برنامه
    \item افزایش کارایی حافظه با امکان اشتراک داده‌ها میان چند فرآیند
\end{itemize}

\subsection{\en{.idata}}
این بخش مربوط به \textbf{جدول واردات} \en{(Import Table)} است و فهرستی از کتابخانه‌ها و توابع خارجی که برنامه از آن‌ها استفاده می‌کند را ذخیره می‌کند.

\textbf{مزایا:}
\begin{itemize}
    \item تسهیل مدیریت وابستگی‌های خارجی
    \item پشتیبانی از بارگذاری پویا \en{(Dynamic Linking)}
    \item امکان به‌روزرسانی کتابخانه‌ها بدون نیاز به تغییر برنامه‌ی اصلی
\end{itemize}

\subsection{\en{.edata}}
بخش \en{.edata} مربوط به \textbf{جدول صادرات} \en{(Export Table)} است. این جدول فهرستی از توابع یا داده‌هایی را نگهداری می‌کند که برنامه برای استفاده‌ی سایر برنامه‌ها در دسترس قرار می‌دهد.

\textbf{مزایا:}
\begin{itemize}
    \item امکان اشتراک توابع و داده‌ها میان چند برنامه
    \item پشتیبانی از ساختار ماژولار نرم‌افزار
    \item کاهش حجم کد تکراری در سیستم
\end{itemize}

\subsection{\en{.rsrc}}
بخش \en{.rsrc} محل ذخیره‌ی \textbf{منابع} \en{(Resources)} مانند آیکون‌ها، منوها، تصاویر، فایل‌های زبان و داده‌های رابط کاربری است. این بخش به برنامه امکان می‌دهد بدون تغییر کد، ظاهر و محتوای خود را به‌روزرسانی کند.

\textbf{اطلاعات موجود در این بخش:}
\begin{itemize}
    \item آیکون‌ها، تصاویر و صداها
    \item اطلاعات نسخه‌ی نرم‌افزار
    \item داده‌های مربوط به زبان و محلی‌سازی \en{(Localization)}
\end{itemize}

\textbf{مزایا:}
\begin{itemize}
    \item جدا کردن منابع از منطق اصلی برنامه
    \item تسهیل تغییر ظاهر یا زبان نرم‌افزار
    \item پشتیبانی از چندزبانگی در نرم‌افزارهای بزرگ
\end{itemize}

\subsection{\en{.reloc}}
بخش \en{.reloc} برای \textbf{اطلاعات جابجایی} \en{(Relocation Information)} استفاده می‌شود. زمانی‌که فایل \en{PE} در آدرس مجازی پیش‌فرض خود بارگذاری نشود، این بخش به سیستم‌عامل کمک می‌کند تا آدرس‌های مطلق اصلاح شوند.

\textbf{مزایا:}
\begin{itemize}
    \item پشتیبانی از بارگذاری انعطاف‌پذیر در حافظه
    \item جلوگیری از تداخل آدرس‌ها میان چند ماژول
    \item بهبود قابلیت استفاده‌ی مجدد از کدها در محیط‌های مختلف
\end{itemize}

\subsection{جمع‌بندی}
هر بخش از فایل \en{PE} نقشی مشخص و حیاتی در عملکرد صحیح برنامه دارد. این تقسیم‌بندی باعث می‌شود سیستم‌عامل بتواند برنامه‌ها را سریع‌تر، ایمن‌تر و با مدیریت بهینه‌ی حافظه اجرا کند.

از بخش هدر که نقشه‌ی کلی فایل را در اختیار \en{Loader} قرار می‌دهد تا بخش‌های کد و داده که منطق برنامه را تشکیل می‌دهند، و همچنین بخش منابع که داده‌های ظاهری و رابط کاربری را نگهداری می‌کند، همگی در کنار هم ساختار منظم و کارآمدی برای اجرای برنامه‌ها در ویندوز ایجاد می‌کنند.

به‌طور کلی، مزیت اصلی ساختار \en{PE} در \textbf{قابلیت حمل، انعطاف‌پذیری، امنیت و ماژولار بودن} آن است؛ ویژگی‌هایی که باعث شده این قالب طی سال‌ها همچنان استاندارد اصلی فایل‌های اجرایی ویندوز باقی بماند.
