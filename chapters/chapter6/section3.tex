% chapter6/section3.tex
\section{توصیه‌ها برای تیم‌های مهندسی نرم‌افزار}
\label{sec:ch6-recommendations}

بر اساس یافته‌های این گزارش و چالش‌های مطرح شده در فصول قبل، توصیه‌های زیر برای تیم‌های مهندسی نرم‌افزار جهت بهبود فرآیندهای توسعه، نگهداری و تکامل نرم‌افزار ارائه می‌گردد:

\begin{enumerate}
    \item پذیرش فرهنگ DevOps فراتر از ابزارها: به یاد داشته باشید که DevOps پیش از هر چیز یک تغییر فرهنگی است. بر شکستن سیلوها (تیم های ایزوله)، ایجاد ارتباط ارتباط باز، همکاری میان‌تیمی و مسئولیت مشترک تمرکز کنید تا مقاومت‌های فرهنگی کاهش یابد.
    \item بازطراحی به عنوان یک فرصت استراتژیک: بازطراحی را نه فقط یک رفع نقص فنی، بلکه یک فرصت استراتژیک برای بازنگری در مدل کسب‌وکار (مانند فوربیکس ) یا بهبود چشمگیر تجربه کاربری (مانند پی‌پال ) در نظر بگیرید.
    \item مدیریت فعال بدهی فنی: بدهی فنی را به‌عنوان بخشی از فرآیند بپذیرید اما برای بازپرداخت آن برنامه‌ریزی منظم داشته باشید. برای توجیه اقتصادی بازطراحی، از معیارهای کمی مانند «نسبت بدهی فنی» (TDR) و محاسبه‌ی هزینه‌ی فرصت ناشی از زمان صرف شده بر روی بدهی فنی (حدود ۴۲٪) استفاده کنید.
    \item اجرای فرآیندهای مدیریت تغییر (change Management): برای جلوگیری از آشفتگی و ناسازگاری‌های ناشی از تغییرات کنترل‌نشده (که عامل شکست پروژه‌هایی چون LASCAD بود)، فرآیندهای رسمی مدیریت تغییر را پیاده‌سازی کنید. این فرآیند باید شامل مستندسازی تغییرات، تحلیل تاثیر و تست پیش از اجرا باشد.
    \item سرمایه‌گذاری بر خودکارسازی(CI/CD): از ابزارهای ادغام مداوم (CI) و تحویل مداوم (CD) مانند Jenkins یا GitLab CI/CD برای خودکارسازی ساخت، تست و استقرار استفاده کنید. این کار به کشف سریع خطاها و افزایش پایداری استقرارها کمک شایانی می‌کند.
    \item برنامه‌ریزی استراتژیک برای سیستم‌های قدیمی (Legacy):
    \begin{itemize}
        \item مهاجرت افزایشی: در مواجهه با سیستم‌های حیاتی، از استراتژی‌های پرخطر «انفجار بزرگ» پرهیز کرده و از «الگوی انجیر خفه‌کننده» (Strangler Fig Pattern) برای مهاجرت افزایشی و کم‌ریسک استفاده کنید.
        \item مهندسی معکوس هوشمند: برای درک ساختار سیستم‌های قدیمی فاقد مستندات ، از ابزارهای مدرن هوش مصنوعی به عنوان «ابزار باستان‌شناسی» برای تسریع فرآیند تحلیل کد و مهندسی معکوس بهره ببرید.
    \end{itemize}
    \item توجه به عوامل انسانی و آموزش مستمر: موفقیت پروژه به عوامل انسانی نیز وابسته است. با ایجاد فرهنگ کاری سالم از فرسودگی تیم جلوگیری کنید. با سرمایه‌گذاری بر آموزش و اشتراک دانش، مهارت‌های مورد نیاز برای فناوری‌های نوین را در تیم تقویت نمایید.
    \item مستندسازی به عنوان یک دارایی کلیدی: با مستندسازی ضعیف که فرآیندهای نگهداری و ورود اعضای جدید به تیم را مختل می‌کند، مقابله کنید. مستندات باید به عنوان منبع حیاتی دانش سازمان تلقی شده و شامل مستندات سیستم، فرآیند و تصمیمات طراحی باشند.
\end{enumerate}
