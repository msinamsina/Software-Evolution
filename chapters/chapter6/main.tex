\chapter{فایل های PE}


فرمت \lr{Portable Executable (PE)} قالب استاندارد فایل‌های اجرایی بومی در خانواده ویندوز (شامل .exe، .dll، .sys و غیره) است. PE فرمتی است مبتنی بر COFF (\lr{Common Object File Format}) که از Unix آمده و توسط Microsoft برای سیستم‌عامل Windows تطبیق داده شد. این فرمت نحوه نگهداری هدرها، بخش‌ها و جداولی را مشخص می‌کند که لودر ویندوز برای نقشه‌برداری فایل به حافظه و اجرای صحیح آن به آن نیاز دارد. در عمل، PE معادل ELF در لینوکس و Mach-O در macOS است و در محیط‌های مبتنی بر UEFI نیز برای فایل‌های اجراپذیر کاربرد دارد. \cite{chapter6ref1}

فایل‌های PE نه تنها شامل فایل‌های اجرایی با پسوند .exe هستند، بلکه بسیاری از انواع فایل‌ها از این فرمت استفاده می‌کنند. کتابخانه‌های پویا ،(.dll) ماژول‌های هسته ،(.sys) برنامه‌های کنترل پنل ،(.cpl) فایل‌های شی (\lr{Object Files}) و حتی برخی فرمت‌های دیگر از جمله فایل‌های فونت نیز از این فرمت بهره می‌برند. \cite{chapter6ref2}


\label{sec:ch6-sec1}
\section{ساختار کلی فایل PE}

فایل PE دارای یک ساختار سلسله مراتبی ثابت است که با چندین هدر شروع شده و به دنبال آن جدول بخش‌ها و در نهایت محتوای واقعی بخش‌ها قرار می‌گیرند. این ساختار مجموعه‌ای از دستورالعمل‌ها را به لودر پویا (\lr{Dynamic Linker}) ارائه می‌دهد تا نحوه‌ی نقشه‌برداری صحیح فایل در حافظه را تعیین کند.\cite{chapter6ref1}
\begin{figure}[h]
\centering
\includegraphics[width=0.8\textwidth]{Portable_Executable_32_bit_Structure_in_SVG_fixed.jpg}
\caption{شمای کلی یک فایل اجرایی قابل حمل (۳۲ بیتی)}
\end{figure}

\noindent
فایل PE شامل قسمت‌های زیر می‌شود:

\noindent
\subsection{\lr{DOS Header}}
هر فایل PE با هدر DOS شروع می‌شود که ساختاری به طول ۶۴ بایت است. این هدر که برای سازگاری با سیستم‌های ۱۶ بیتی MS-DOS قدیمی طراحی شده، شامل یک امضای استاندارد و یک اشاره‌گر کلیدی است.
بایت‌های اولیه فایل همیشه شامل امضای جادویی MZ (معادل هگز \lr{0x5A4D} و نشان‌دهنده \lr{Mark Zbikowski}، یکی از توسعه‌دهندگان \lr{MS-DOS}) است که اولین نشانه‌ی شناسایی فرمت PE است.
فیلد 4 بایتی e\_lfanew در آفست ثابت \lr{0x3C} از شروع فایل قرار دارد و شامل آفست فایل (\lr{File Offset}) شروع هدرهای NT است.\cite{chapter6ref1} برای سیستم‌عامل‌های مدرن ویندوز، مهم‌ترین وظیفه \lr{DOS Header}، ارائه این اشاره‌گر است. بارگذار ویندوز بلافاصله به این آفست مراجعه کرده و از بقیه هدر DOS و برنامه \lr{DOS Stub} صرف نظر می‌کند تا به ساختار اصلی NT دست یابد.\cite{chapter6ref1}

\noindent
\subsection{\lr{DOS Stub}}
برنامه‌ای کوچک سازگار با \lr{MS-DOS 2.0} که صرفاً یک پیام خطا چاپ می‌کند: \lr{"This program cannot be run in DOS mode"}. این Stub اطمینان می‌دهد که اگر برنامه در محیط DOS اجرا شود، یک پیام معنادار نمایش داده شود.\cite{chapter6ref2}

\noindent
\subsection{\lr{NT Headers}}
هدرهای \lr{NT}، که موقعیت شروع آن توسط e\_lfanew تعیین شده، اطلاعاتی درباره ماشین هدف (\lr{Target Machine}) و ویژگی‌های فایل PE دارد و شامل سه بخش متوالی است:\cite{chapter6ref2}

\noindent
1. \lr{PE Signature}: یک امضای ۴ بایتی که مقدار ثابت \lr{PE00} (\lr{0x50450000}) است که فایل را به عنوان PE شناسایی می‌کند​.\cite{chapter6ref2}

\noindent
2. \lr{COFF File Header}: هدری که اطلاعات کلی درباره فایل مانند معماری هدف (\lr{Machine Type})، تعداد بخش‌ها (NumberOfSections)، timestamp، اندازه Optional Header و Characteristics را نگهداری می‌کند​.\cite{chapter6ref2}
\begin{itemize}
    \setlength{\itemsep}{-0.5em}
    \item \lr{Machine} - مشخص‌کننده معماری CPU هدف است.
    \item \lr{NumberOfSections} - تعداد بخش‌های موجود در فایل را نشان می‌دهد.
    \item \lr{TimeDateStamp} - زمان و تاریخ ایجاد فایل را مشخص می‌کند.
    \item \lr{PointerToSymbolTable} - آدرس (offset) جدول نمادها است، که معمولاً مقدار آن صفر است.
    \item \lr{NumberOfSymbols} - تعداد نمادهای موجود در جدول نمادها را مشخص می‌کند (معمولاً صفر است).
    \item \lr{SizeOfOptionalHeader} - اندازه هدر اختیاری را نشان می‌دهد.
    \item \lr{Characteristics} - شامل پرچم‌هایی است که ویژگی‌های فایل را مشخص می‌کنند (مثلاً فایل اجرایی، DLL و غیره).
\end{itemize}


\noindent
3. \lr{Optional Header}: 
علی‌رغم نام، برای فایل‌های PE الزامی است و شامل اطلاعات مهمی برای بارگذاری (Loading) و اجرای برنامه است. اطلاعات مهمی مانند نقطه ورود برنامه (\lr{Entry Point RVA})، \lr{ImageBase}، اطلاعات \lr{Alignment} و ... .\cite{chapter6ref2}
\begin{itemize}
    \setlength{\itemsep}{-0.5em}
    \item \lr{Magic} - نوع فایل اجرایی را مشخص می‌کند (مانند \lr{PE32} با مقدار \lr{0x010B} یا \lr{PE32+} با مقدار \lr{0x020B} برای ۶۴ بیت).
    \item \lr{MajorLinkerVersion / MinorLinkerVersion} - نسخه لینکِر (Linker) مورد استفاده را نشان می‌دهد.
    \item \lr{SizeOfCode} - اندازه بخش کد را مشخص می‌کند.
    \item \lr{SizeOfInitializedData} - اندازه بخش داده‌های مقداردهی‌شده را مشخص می‌کند.
    \item \lr{AddressOfEntryPoint} - آدرس مجازی نسبی (\lr{RVA}) نقطه ورود برنامه را نشان می‌دهد.
    \item \lr{BaseOfCode / BaseOfData} - آدرس‌های مجازی نسبی (\lr{RVA}) بخش‌های کد و داده را مشخص می‌کنند.
    \item \lr{ImageBase} - آدرس بارگذاری ترجیحی تصویر (Image) در حافظه را مشخص می‌کند.
    \item \lr{SectionAlignment / FileAlignment} - نحوه تراز (Alignment) بخش‌ها در حافظه و روی دیسک را تعیین می‌کند.
    \item \lr{SizeOfImage} - اندازه کل تصویر (Image) در حافظه را نشان می‌دهد.
    \item \lr{SizeOfHeader} - اندازه ترکیبی تمام سربرگ‌ها (Headers) را مشخص می‌کند.
    \item \lr{Subsystem} - زیرسامانه‌ای را مشخص می‌کند که برای اجرای فایل لازم است (مثلاً رابط کاربری گرافیکی ویندوز).
    \item \lr{DllCharacteristics} - شامل پرچم‌هایی است که ویژگی‌های فایل DLL را مشخص می‌کنند.
    \item \lr{SizeOfStackReserve / SizeOfStackCommit} - اندازه رزرو و تعهد (Commit) پشته (Stack) را مشخص می‌کند.
    \item \lr{SizeOfHeapReserve / SizeOfHeapCommit} - اندازه رزرو و تعهد (Commit) پشته حافظه (Heap) را مشخص می‌کند.
    \item \lr{NumberOfRvaAndSizes} - تعداد ورودی‌های موجود در جدول آدرس‌های مجازی نسبی را نشان می‌دهد.
\end{itemize}

\noindent
4. \lr{Data Directories}: 
این بخش به جداول و منابع مختلفی در فایل اشاره می‌کند؛ مانند جدول واردات (\lr{Imports})، صادرات (\lr{Exports})، منابع (\lr{Resources}) و موارد دیگر.\cite{chapter6ref2}
\begin{itemize}
    \setlength{\itemsep}{-0.5em}
    \item \lr{Export Table} - آدرس و اندازهٔ جدول صادرات.
    \item \lr{Import Table} - آدرس و اندازهٔ جدول واردات.
    \item \lr{Resource Table} - آدرس و اندازهٔ جدول منابع.
    \item \lr{Exception Table} - آدرس و اندازهٔ جدول استثناها (Exception).
    \item \lr{Certificate Table} - آدرس و اندازهٔ جدول گواهی امنیتی.
    \item \lr{Base Relocation Table} - آدرس و اندازهٔ جدول بازنشانی پایه (\lr{Relocation}).
    \item \lr{Debug Data} - آدرس و اندازهٔ داده‌های اشکال‌زدایی (\lr{Debug}).
    \item \lr{Architecture Data} - رزرو شده؛ مقدار آن باید صفر باشد.
    \item \lr{Global Pointer Register} - آدرس مجازی نسبی (\lr{RVA}) مقداری که در ثبات اشاره‌گر سراسری ذخیره می‌شود.
    \item \lr{TLS Table} - آدرس و اندازهٔ جدول حافظهٔ محلی رشته‌ها (\lr{Thread-Local Storage}).
    \item \lr{Load Configuration Table} - آدرس و اندازهٔ جدول پیکربندی بارگذاری.
    \item \lr{Bound Import Table} - آدرس و اندازهٔ جدول واردات مقید (\lr{Bound Import}).
    \item \lr{Import Address Table} - آدرس و اندازهٔ جدول آدرس‌های واردات.
    \item \lr{Delay Import Descriptor} - آدرس و اندازهٔ توصیف‌گر واردات تأخیری (\lr{Delay Import}).
    \item \lr{CLR Runtime Header} - آدرس و اندازهٔ هدر زمان‌اجرای \lr{CLR}.
    \item \lr{Reserved} - برای استفاده‌های آینده رزرو شده است.
\end{itemize}

\noindent
\subsection{\lr{Section Headers}}
جدولی از هدرهای بخش (\lr{Section Headers}) که برای هر بخش در فایل یک ورودی دارد. هر \lr{Section Header} شامل نام بخش، \lr{Virtual Address}، \lr{Virtual Size}، \lr{Pointer-To-Raw-Data} و سایر metadata است.\cite{chapter6ref2}

\noindent
\subsection{\lr{Sections}}
بخش‌های اصلی فایل که محتویات واقعی برنامه را شامل می‌شود. بخش‌های استاندارد شامل .text (کد)، .data (داده‌های مقداردهی‌شده)، .rdata (داده‌های فقط‌خوانده)، .rsrc (منابع) و دیگری هستند.\cite{chapter6ref2}


\begin{table}[h!]
    \centering
    \begin{tabular}{llll}
        \toprule
        \textbf{توضیح} & \textbf{حافظه} & \textbf{کاربرد اصلی} & \textbf{نام بخش} \\
        \midrule
        دستورالعمل‌های CPU و کد اصلی برنامه                 & R-X           & کد قابل اجرا       & .text   \\
        متغیرهای جهانی اولیه‌شده                            & RW-           & داده مقداردهی شده  & .data   \\
        رشته‌های ثابت و اطلاعات Import/Export               & R--            & داده فقط‌خوانده    & .rdata  \\
        اسامی DLL‌ها و توابع وارد‌شده                        & R--           & جدول Import        & .idata  \\
        اسامی توابع صادرشده                                & R--           & جدول Export        & .edata  \\
        آیکن‌ها، تصاویر، رشته‌ها، \lr{Dialog}‌های رابط کاربری & R--          & منابع برنامه       & .rsrc   \\
        آدرس‌های نیازمند بازنشانی برای ASLR                 & R--           & اطلاعات Relocation & .reloc  \\
        جداول \lr{exception handling} (فقط ۶۴ بیت)         & R--             & اطلاعات Exception  & .pdata  \\
        \bottomrule
    \end{tabular}
    \caption{بخش‌های مختلف فایل اجرایی (\lr{PE Sections}) و کاربرد آن‌ها}
    \label{tab:pe_sections}
\end{table}

\begin{figure}[h]
\centering
\includegraphics[width=0.8\textwidth]{images/PE Headers annotated.png}
\caption{نمونه کد باینری یک فایل PE}
\end{figure}

\label{sec:ch6-2-Pe}
\section{هربخش فایل PE چه اطلاعاتی دارد}

\subsection{هدر \lr{DOS}}
\label{subsec:ch6-2-1-dosheader}

بخش آغازین هر فایل اجرایی در ویندوز با ساختاری موسوم به \lr{DOS Header} مشخص می‌شود. این ساختار که با امضای دودویی \lr{MZ} (برگرفته از نام \lr{Mark Zbikowski}) آغاز می‌گردد، در اصل بازمانده‌ای از قالب‌های اجرایی \lr{MS-DOS} است و برای حفظ سازگاری عقب‌رو در فایل‌های اجرایی مدرن همچنان نگه داشته شده است. هدر \lr{DOS} معمولاً ۶۴ بایت نخست فایل را دربر می‌گیرد و شامل تعدادی فیلد از پیش تعریف‌شده است که اطلاعات پایه‌ای درباره‌ی تصویر اجرایی را ذخیره می‌کنند.

یکی از مهم‌ترین فیلدهای این ساختار، مقدار چهار‌بایتی \lr{e\_lfanew} است. این فیلد آفستِ محل شروع هدر اصلی فایل، یعنی \lr{PE (NT) Header} را نسبت به ابتدای فایل مشخص می‌کند. لودر ویندوز با اتکا به همین مقدار می‌تواند به نقطه‌ی دقیقی که ساختار \lr{PE} در آن قرار گرفته دسترسی پیدا کرده و فرایند بارگذاری تصویر را در حافظه آغاز کند. بدیهی است در صورتی‌که این مقدار نادرست باشد یا به ناحیه‌ای نامعتبر اشاره کند، سیستم‌عامل قادر نخواهد بود فایل را به‌عنوان یک تصویر اجرایی معتبر تشخیص دهد و بارگذاری آن متوقف می‌شود.

پس از فیلدهای هدر \lr{DOS}، بخشی کوتاه از کد اجرایی موسوم به \lr{DOS Stub} قرار می‌گیرد. هدف از این بخش، تضمین رفتار قابل‌قبول در محیط‌هایی است که از قالب \lr{PE} پشتیبانی نمی‌کنند. این کد معمولاً در صورت اجرای فایل در سیستم‌های قدیمی یا محیط‌های ناسازگار، تنها یک پیام ساده (مانند «این برنامه را باید در ویندوز اجرا کنید») نمایش داده و از ادامه‌ی اجرا جلوگیری می‌کند. در سیستم‌های امروزی این کد عملاً اجرا نمی‌شود، اما وجودش بخشی از قالب استاندارد فایل‌های اجرایی ویندوز است.

از منظر تحلیل معکوس و بررسی‌های امنیتی، مطالعه‌ی مقادیر موجود در \lr{DOS Header} می‌تواند نشانه‌هایی از دست‌کاری، پکر شدن فایل یا تلاش برای پنهان‌سازی ساختار واقعی تصویر را آشکار کند. هرچند این بخش در روند اجرای واقعی برنامه نقش کاربردی مستقیمی ندارد، اما برای آن‌که فایل توسط ویندوز به‌عنوان یک تصویر اجرایی معتبر در نظر گرفته شود باید وجود داشته باشد و مقادیر کلیدی آن (به‌ویژه \lr{e\_lfanew}) صحیح باشند. برای جزئیات بیشتر می‌توان به مستندات رسمی مایکروسافت درباره‌ی قالب \lr{PE} و همچنین منابع تحلیلی حوزه‌ی بدافزار مراجعه کرد \cite{MicrosoftPEFormatSpecification,TechZealotsPEStructureMalware}.

\subsection{هدر \lr{PE}}
\label{subsec:ch6-2-2-peheader}

پس از بخش ابتدایی \lr{DOS Header}، فایل اجرایی وارد مرحله‌ای می‌شود که ساختار واقعی قابل‌اجرای ویندوز را تشکیل می‌دهد. این بخش با امضای چهار بایتی \lr{PE\textbackslash0\textbackslash0} آغاز می‌شود و به عنوان نقطه‌ی شروع هدر اصلی یا همان \lr{NT Header} شناخته می‌شود. در این بخش، اطلاعات حیاتی درباره‌ی ماهیت فایل و نحوه‌ی بارگذاری آن در حافظه ذخیره شده است. سیستم‌عامل ویندوز در هنگام اجرای برنامه، ابتدا به این بخش مراجعه می‌کند تا بر اساس مقادیر موجود در آن، نقشه‌ی حافظه‌ی برنامه را ایجاد کرده و بخش‌های مختلف فایل را در موقعیت‌های مناسب بارگذاری کند.

ساختار کلی هدر \lr{PE} شامل سه بخش است: امضا، هدر فایل (\lr{COFF File Header}) و هدر اختیاری (\lr{Optional Header}). امضای \lr{PE\textbackslash0\textbackslash0} نشانه‌ای است که سیستم از طریق آن تشخیص می‌دهد فایل متعلق به قالب اجرایی مدرن ویندوز است. در ادامه، هدر فایل اطلاعات پایه‌ای نظیر نوع پردازنده‌ی هدف، تعداد بخش‌ها، زمان کامپایل و ویژگی‌های کلی فایل را در خود دارد. بخش اختیاری، علی‌رغم نام آن، جزئی حیاتی از این ساختار است و داده‌هایی مانند آدرس نقطه‌ی ورود برنامه، اندازه‌ی کلی تصویر در حافظه، نسخه‌ی زیرسیستم اجرایی، و مسیرهای جداول داده‌ای نظیر \lr{import}، \lr{export}، \lr{resources} و \lr{relocation} را تعریف می‌کند. این داده‌ها به لودر ویندوز امکان می‌دهند تا به‌صورت نظام‌مند و دقیق، برنامه را در فضای مجازی حافظه سازمان‌دهی کرده و ارتباط آن را با کتابخانه‌های اشتراکی برقرار کند.

اهمیت این ساختار فراتر از نقش فنی آن در اجرای برنامه‌هاست و می‌توان آن را نمونه‌ای از تکامل تدریجی معماری نرم‌افزار در سطح سیستم‌عامل دانست. قالب \lr{PE} در واقع حاصل فرگشت فرمت‌های اجرایی قدیمی‌تر مانند \lr{MZ} و \lr{NE} است که در دوران گذار از محیط خط فرمان \lr{DOS} به معماری چندوظیفه‌ای \lr{Windows NT} شکل گرفت. در این روند، مایکروسافت تلاش کرد قالبی ارائه دهد که در عین حفظ سازگاری با گذشته، بتواند پاسخ‌گوی نیازهای روزافزون در زمینه‌ی امنیت، چندمعماری بودن، و توسعه‌پذیری باشد. نتیجه، ساختاری بود که نه‌تنها اطلاعات لازم برای اجرای فایل را نگهداری می‌کند، بلکه بستری منعطف برای افزوده شدن قابلیت‌های جدید در گذر زمان فراهم می‌آورد.

از منظر تکامل نرم‌افزاری، پایداری و تداوم این ساختار در نسخه‌های مختلف ویندوز بیانگر نوعی تعهد به اصل «پایداری رابط‌ها» است. این اصل تضمین می‌کند که ابزارها، کتابخانه‌ها و حتی نرم‌افزارهایی که دهه‌ها پیش توسعه یافته‌اند، همچنان بتوانند با نسخه‌های جدید سیستم‌عامل تعامل داشته باشند. همین پایداری است که باعث شده قالب \lr{PE} طی بیش از سه دهه، بدون نیاز به بازنویسی بنیادی، بتواند تغییرات مهمی مانند پشتیبانی از معماری ۶۴ بیتی، امضای دیجیتال، حفاظت از فضاهای حافظه و بارگذاری تصادفی (\lr{ASLR}) را در خود جای دهد. به بیان دیگر، \lr{PE Header} نه‌تنها بخشی از ساختار فایل اجرایی است، بلکه نمود عینی مفهوم «تکامل پایدار» در مهندسی نرم‌افزار محسوب می‌شود؛ مفهومی که در آن، طراحی اولیه به اندازه‌ای منعطف و تعمیم‌پذیر است که امکان رشد و گسترش در گذر زمان را بدون از دست دادن سازگاری فراهم می‌کند.

در زمینه‌ی تحلیل بدافزار نیز، \lr{PE Header} به عنوان نقطه‌ای کلیدی برای شناسایی ویژگی‌های رفتاری فایل شناخته می‌شود. بررسی دقیق فیلدهای این بخش می‌تواند سرنخ‌هایی درباره‌ی نوع کامپایلر، مسیرهای کتابخانه‌های واردشده، و حتی وجود تغییرات غیرعادی ناشی از بسته‌بندی یا رمزگذاری ارائه دهد. به همین دلیل، این بخش نه‌تنها در اجرای فایل، بلکه در تحلیل و درک رفتار آن نیز نقشی بنیادین دارد.

در مجموع، \lr{PE Header} را می‌توان هسته‌ی منطقی فرمت اجرایی ویندوز دانست؛ بخشی که با وجود ظاهر فنی و ثابت خود، بازتابی از رویکرد تکاملی در طراحی نرم‌افزار است، رویکردی که به‌جای جایگزینی کامل ساختارها، آن‌ها را به‌تدریج غنی‌تر و پایدارتر می‌سازد.\cite{MicrosoftPEFormatSpecification}.


\subsection{جدول بخش‌ها (\lr{Section Table})}
\label{subsec:ch6-2-3-section-table}

پس از پایان هدر \lr{PE}، ساختاری با عنوان \lr{Section Table} یا جدول بخش‌ها قرار دارد که نقش آن تعریف اجزای اصلی فایل اجرایی است. این جدول بلافاصله پس از هدر اختیاری قرار می‌گیرد و شامل آرایه‌ای از ساختارهای تکرارشونده موسوم به \lr{IMAGE\_SECTION\_HEADER} است؛ هر یک از این ساختارها نماینده‌ی یکی از بخش‌های فایل (مانند \lr{.text}، \lr{.data}، \lr{.rdata}، \lr{.rsrc} و غیره) محسوب می‌شوند. تعداد ورودی‌های جدول برابر با مقداری است که در فیلد \lr{NumberOfSections} از \lr{File Header} مشخص شده است و ترتیب آن‌ها دقیقاً با ترتیب فیزیکی بخش‌ها در فایل هم‌خوانی دارد.

هر ورودی جدول شامل اطلاعات دقیق مربوط به نام بخش، اندازه‌ی مجازی آن در حافظه، اندازه‌ی واقعی آن در فایل، آدرس مجازی شروع بخش (\lr{VirtualAddress})، مکان شروع داده‌ها در فایل (\lr{PointerToRawData}) و مجموعه‌ای از پرچم‌ها (\lr{Characteristics}) است که نوع دسترسی آن بخش را تعیین می‌کنند. برای نمونه، بخش \lr{.text} معمولاً با پرچم‌های قابل‌اجرا و فقط‌خواندنی مشخص می‌شود، در حالی‌که \lr{.data} قابل‌نوشتن است. لودر ویندوز در زمان بارگذاری برنامه، بر اساس همین مقادیر تصمیم می‌گیرد که هر بخش را در کجای حافظه مستقر کرده و چه سطحی از دسترسی به آن اختصاص دهد. به این ترتیب، \lr{Section Table} را می‌توان نقشه‌ی دقیق تخصیص حافظه و نحوه‌ی سازمان‌دهی کد و داده در محیط اجرایی دانست.

در سطح ساختاری، ترتیب و چیدمان بخش‌ها انعکاس‌دهنده‌ی منطق کامپایلر و پیونددهنده (\lr{linker}) است؛ برای مثال، بخش \lr{.text} معمولاً در ابتدای تصویر قرار می‌گیرد زیرا شامل دستورالعمل‌های اجرایی است و پس از آن داده‌ها، منابع و اطلاعات باز‌اسکان‌پذیری جای می‌گیرند. در بسیاری از فایل‌های سیستمی یا بدافزارها، تغییر در توالی یا اندازه‌ی این بخش‌ها می‌تواند نشانه‌ای از فشرده‌سازی، رمزگذاری یا تزریق کد باشد. از این رو، تحلیل‌گران امنیتی همواره جدول بخش‌ها را یکی از نخستین نقاط بررسی خود قرار می‌دهند تا بتوانند تفاوت میان ساختار واقعی و ساختار مورد انتظار را شناسایی کنند.

از منظر تکامل نرم‌افزاری، مفهوم \lr{Section Table} بیانگر یکی از اصول بنیادی طراحی ماژولار در معماری سیستم‌های اجرایی است. تقسیم فایل به بخش‌های مستقل با کارکردهای مشخص، امکان توسعه و نگهداری تدریجی را فراهم کرده است. به همین دلیل، قالب \lr{PE} در طول دهه‌ها بدون تغییر اساسی در منطق خود، توانسته از افزوده شدن ویژگی‌های متنوعی مانند داده‌های منابع چندزبانه، متادیتاهای مدیریت‌شده در \lr{.NET}، و بخش‌های مخصوص به امضاهای دیجیتال پشتیبانی کند. این پایداری ساختاری، نشان‌دهنده‌ی درک عمیق طراحان از نیاز به انعطاف‌پذیری بلندمدت در طراحی قالب‌های اجرایی است؛ مفهومی که ارتباط مستقیمی با اصول تکامل پایدار نرم‌افزار دارد.

در مجموع، \lr{Section Table} نقطه‌ی اتصال میان ساختار منطقی برنامه و بازنمایی فیزیکی آن در حافظه است. بدون این جدول، لودر ویندوز قادر نخواهد بود بخش‌های مختلف فایل را به‌درستی از روی دیسک به حافظه منتقل کند یا دسترسی‌های لازم را برای اجرای ایمن فراهم آورد. این بخش نه‌تنها عنصر حیاتی در بارگذاری برنامه است، بلکه در تحلیل ساختار فایل و تشخیص تغییرات غیرمجاز نیز جایگاهی اساسی دارد؛ به گونه‌ای که کوچک‌ترین انحراف در مقادیر آن می‌تواند چهره‌ی واقعی یک فایل سالم یا مخرب را آشکار سازد.\cite{MicrosoftPEFormatSpecification}.
\begin{table}[H]
    \centering
    \begin{tabular}{c c l p{7cm}}
        \toprule
        \textbf{Offset} & \textbf{Size} & \textbf{Field} & \textbf{Description} \\
        \midrule
        0  & 8 & \lr{Name} & نام بخش به‌صورت رشتهٔ ۸ بایتی (اگر طول کمتر باشد با صفر پُر می‌شود). \\
        8  & 4 & \lr{VirtualSize} & اندازهٔ بخش هنگام بارگذاری در حافظه؛ اگر از \lr{SizeOfRawData} بزرگ‌تر باشد، انتها صفر می‌شود. \\
        12 & 4 & \lr{VirtualAddress} & آدرس مجازی شروع بخش نسبت به مبنای تصویر هنگام بارگذاری. \\
        16 & 4 & \lr{SizeOfRawData} & اندازهٔ دادهٔ بخش روی دیسک؛ باید با مقدار \lr{FileAlignment} در هدر اختیاری هم‌تراز باشد. \\
        20 & 4 & \lr{PointerToRawData} & اشاره‌گر فایل به ابتدای داده‌های این بخش در تصویر \lr{PE}/\lr{COFF}. \\
        24 & 4 & \lr{PointerToRelocations} & محل ورودی‌های جابجایی برای این بخش؛ در تصاویر اجرایی معمولاً صفر است. \\
        28 & 4 & \lr{PointerToLinenumbers} & محل ورودی‌های شماره‌خط \lr{COFF}؛ برای تصاویر اجرایی صفر است. \\
        32 & 2 & \lr{NumberOfRelocations} & تعداد ورودی‌های جابجایی در این بخش؛ برای تصاویر اجرایی صفر است. \\
        34 & 2 & \lr{NumberOfLinenumbers} & تعداد ورودی‌های شماره‌خط در این بخش؛ برای تصاویر اجرایی صفر است. \\
        36 & 4 & \lr{Characteristics} & پرچم‌های توصیف‌کنندهٔ ویژگی‌های بخش (قابل‌اجرا، قابل‌نوشتن، فقط‌خواندنی و ...). \\
        \bottomrule
    \end{tabular}
    \caption{ساختار کلی هر ورودی در جدول بخش‌ها (\lr{Section Table Entry}) در قالب فایل \lr{PE} \cite{MicrosoftPEFormatSpecification}}
    \label{tab:section-table-entry}
\end{table}
\subsection{بخش‌های \lr{.text}، \lr{.data} و \lr{.rdata}}
\label{subsec:ch6-2-4-text-data-rdata}

در ادامهٔ ساختار فایل اجرایی، بخش‌های مختلفی وجود دارند که هرکدام نقش خاصی در سازمان‌دهی کد و داده در حافظه ایفا می‌کنند. این بخش‌ها در جدول بخش‌ها تعریف می‌شوند و از طریق فیلدهای مشخص‌شده در هر ورودی جدول، لودر سیستم‌عامل می‌داند چگونه باید آن‌ها را در فضای مجازی حافظه نگاشت کند. از میان این بخش‌ها، سه بخش \lr{.text}، \lr{.data} و \lr{.rdata} تقریباً در تمام فایل‌های اجرایی ویندوز وجود دارند و پایه‌ی عملکردی برنامه را تشکیل می‌دهند.

بخش \lr{.text} به عنوان اصلی‌ترین بخش اجرایی، شامل تمامی دستورالعمل‌های ماشین و منطق برنامه است. کدهای ترجمه‌شده از زبان سطح بالا در زمان کامپایل در این بخش قرار می‌گیرند. آدرس نقطه‌ی ورود (\lr{Entry Point}) برنامه نیز در بیشتر موارد در همین بخش قرار دارد. از آن‌جا که محتوای این بخش باید توسط پردازنده به عنوان دستورالعمل اجرا شود، سیستم‌عامل آن را با سطح دسترسی «قابل اجرا و فقط‌خواندنی» (\lr{Execute / Read}) در حافظه نگاشت می‌کند تا از تغییر تصادفی یا عمدی کد در زمان اجرا جلوگیری شود. در معماری امنیتی ویندوز، همین تفکیک اجازه داده است که سازوکارهایی نظیر \lr{DEP (Data Execution Prevention)} مؤثر واقع شوند؛ به این معنا که فقط نواحی متعلق به \lr{.text} و بخش‌های مشخص‌شده مجاز به اجرا هستند. در نتیجه، دستکاری در این بخش یا تزریق کد جدید در نواحی داده‌ای توسط بدافزارها می‌تواند از طریق مقایسهٔ اندازه‌ها و هش این بخش شناسایی گردد. علاوه بر این، الگوی چینش توابع در \lr{.text} معمولاً سرنخ‌های مهمی درباره‌ی کامپایلر، بهینه‌سازی‌ها و حتی زبان برنامه‌نویسی به‌کاررفته فراهم می‌کند که در تحلیل مهندسی معکوس کاربرد دارد.

در مقابل، بخش \lr{.data} برای نگهداری داده‌های اولیه‌شده‌ای استفاده می‌شود که برنامه در طول اجرای خود نیاز به خواندن یا تغییر آن‌ها دارد. به بیان دیگر، هر متغیر سراسری یا ایستا که در زمان کامپایل مقدار مشخصی دریافت کرده باشد، در این بخش ذخیره می‌شود. سیستم‌عامل هنگام بارگذاری برنامه، مقادیر این بخش را به همان شکل در حافظه کپی می‌کند و اجازه‌ی خواندن و نوشتن (\lr{Read / Write}) به آن می‌دهد. تفاوت این بخش با \lr{.bss} (که برای داده‌های مقداردهی‌نشده است) در همین مقدار اولیه نهفته است. تحلیل‌گران امنیتی معمولاً این بخش را برای یافتن داده‌های حساس، کلیدهای رمزنگاری یا مقادیر پیکربندی بررسی می‌کنند. از منظر طراحی نرم‌افزار، وجود \lr{.data} نمادی از جداسازی داده و منطق در سطح ماشین است؛ تفکیکی که امکان بهینه‌سازی مصرف حافظه و اجرای امن‌تر را فراهم می‌کند.

در کنار این دو، بخش \lr{.rdata} یا \lr{Read-Only Data} جایگاهی میان کد و داده دارد. همان‌گونه که از نامش پیداست، داده‌های موجود در این بخش پس از بارگذاری در حافظه غیرقابل تغییر هستند. این داده‌ها معمولاً شامل رشته‌های ثابت، جداول ثابت برنامه، اشاره‌گرهای تابع، مقادیر ثابت کامپایل‌شده و همچنین جداول واردات (\lr{Import Tables}) و صادرات (\lr{Export Tables}) هستند. لودر ویندوز از اطلاعات این بخش برای پیوند دادن تابع‌های خارجی و کتابخانه‌های اشتراکی استفاده می‌کند. برای مثال، جدول واردات در \lr{.rdata} مشخص می‌کند که برنامه از چه توابعی در چه کتابخانه‌هایی مانند \lr{kernel32.dll} یا \lr{user32.dll} استفاده می‌کند. در نتیجه، \lr{.rdata} نه‌تنها محل داده‌های غیرقابل تغییر است، بلکه در عمل نقشه‌ی وابستگی‌های خارجی برنامه را نیز در خود دارد. به دلیل همین نقش دوگانه، دستکاری در این بخش یکی از روش‌های رایج در بدافزارها برای پنهان‌سازی رفتار واقعی یا تغییر مسیر فراخوانی توابع سیستمی است.

از دیدگاه تکامل نرم‌افزاری، تقسیم ساختار فایل اجرایی به بخش‌هایی با ماهیت متفاوت، مصداقی روشن از طراحی ماژولار و اصل \lr{Separation of Concerns} (تفکیک نگرانی‌ها) است. در نسخه‌های اولیهٔ سیستم‌عامل \lr{DOS}، تمام کد و داده در یک فضای خطی قرار می‌گرفت و مدیریت آن‌ها به شکل دستی صورت می‌گرفت. اما در معماری \lr{NT} و فرمت \lr{PE}، هر بخش دارای هدف، سطح دسترسی و محدودیت‌های تعریف‌شده است. این طراحی باعث شد قالب \lr{PE} بتواند بدون تغییر بنیادی، با پیشرفت معماری‌های پردازنده و افزوده شدن ویژگی‌های امنیتی جدید، همچنان پایدار بماند. برای مثال، افزودن بخش‌های خاص مانند \lr{.tls} (\lr{Thread Local Storage}) یا \lr{.pdata} در نسخه‌های جدیدتر، بدون تأثیر منفی بر ساختار موجود ممکن شد، زیرا هستهٔ طراحی بر پایه‌ی بخش‌بندی استاندارد استوار بود.

بدین ترتیب، بخش‌های \lr{.text}، \lr{.data} و \lr{.rdata} در کنار یکدیگر نمایانگر سه بُعد اصلی یک برنامهٔ اجرایی هستند: منطق، وضعیت و ثبات. منطق در \lr{.text} تجلی می‌یابد، وضعیت در \lr{.data} حفظ می‌شود، و ثبات در \lr{.rdata} تبلور می‌یابد. این سه‌گانه نه‌تنها اساس اجرای برنامه را شکل می‌دهند، بلکه بازتابی از روند تکامل مفهومی نرم‌افزار از ساختارهای یکنواخت به معماری‌های تفکیک‌شده و قابل نگهداری هستند.\cite{MicrosoftPEFormatSpecification}.


\label{sec:ch6-sec3}
\section{هر بخش از فایل PE چه مزایایی دارد و چه اطلاعاتی را در خود ذخیره می‌کند؟}

فایل‌های اجرایی در سیستم‌عامل ویندوز با قالب \en{PE (Portable Executable)} شناخته می‌شوند. این قالب، ساختار استاندارد تمام فایل‌های اجرایی مانند \en{.exe}، \en{.dll}، \en{.sys} و بسیاری از فایل‌های سیستم است. هدف از طراحی قالب \en{PE} ایجاد یک چارچوب قابل‌حمل میان نسخه‌های مختلف ویندوز است تا سیستم‌عامل بتواند برنامه‌ها را به‌صورت بهینه در حافظه بارگذاری، مدیریت و اجرا کند.

ساختار فایل \en{PE} از چندین بخش اصلی تشکیل شده است که هر کدام وظیفه‌ی خاصی دارند. این بخش‌ها اطلاعات ضروری مانند کد اجرایی، داده‌ها، منابع و تنظیمات بارگذاری را ذخیره می‌کنند. در ادامه، هر بخش فایل \en{PE} به همراه وظایف، مزایا و نوع اطلاعات ذخیره‌شده در آن توضیح داده می‌شود.

\subsection{\en{Header} (هدر فایل \en{PE})}
هدر فایل \en{PE} در ابتدای فایل قرار دارد و شامل اطلاعات کلی درباره‌ی ساختار، نسخه و نحوه‌ی بارگذاری برنامه در حافظه است. این بخش به سیستم‌عامل کمک می‌کند تا بداند فایل چگونه باید تفسیر و اجرا شود.

\textbf{زیرمجموعه‌های مهم هدر \en{PE} عبارت‌اند از:}
\begin{itemize}
    \item \textbf{\en{DOS Header}:} برای سازگاری با نسخه‌های قدیمی \en{DOS} استفاده می‌شود. اگر برنامه در محیط \en{DOS} اجرا شود، پیامی مانند \en{This program cannot be run in DOS mode} نمایش داده می‌شود. مزیت آن \textbf{سازگاری عقب‌رو} \en{(Backward Compatibility)} است.
    \item \textbf{\en{PE Signature}:} رشته‌ای با مقدار \code{PE\textbackslash0\textbackslash0} که آغاز ساختار واقعی \en{PE} را مشخص می‌کند.
    \item \textbf{\en{File Header}:} حاوی اطلاعات عمومی مانند نوع فایل (\en{exe} یا \en{dll})، تعداد بخش‌ها، تاریخ ساخت و اندازه‌ی داده‌ها است.
    \item \textbf{\en{Optional Header}:} شامل اطلاعات حیاتی برای بارگذاری در حافظه مانند آدرس نقطه‌ی ورود \en{(Entry Point)}، اندازه‌ی پشته و نوع پردازنده است.
\end{itemize}

\textbf{مزایا:}
\begin{itemize}
    \item فراهم‌کردن نقشه‌ی دقیق ساختار فایل برای \en{Loader}
    \item افزایش سازگاری با نسخه‌های مختلف ویندوز
    \item بهینه‌سازی فرآیند بارگذاری و تخصیص حافظه
\end{itemize}

\subsection{\en{.text}}
بخش \en{.text} شامل \textbf{کدهای اجرایی برنامه} (دستورالعمل‌های ماشین) است. این قسمت معمولاً به‌صورت فقط‌خواندنی در حافظه نگهداری می‌شود تا از تغییرات ناخواسته جلوگیری شود.

\textbf{اطلاعات موجود در این بخش:}
\begin{itemize}
    \item دستورالعمل‌های کامپایل‌شده
    \item آدرس و محل قرارگیری توابع
    \item داده‌های ثابت و ثابت‌های برنامه
\end{itemize}

\textbf{مزایا:}
\begin{itemize}
    \item افزایش امنیت از طریق جلوگیری از تغییر کدها در زمان اجرا
    \item صرفه‌جویی در حافظه به دلیل امکان اشتراک کد بین چند فرآیند
    \item تسهیل فرآیند اشکال‌زدایی \en{(Debugging)}
\end{itemize}

\subsection{\en{.data}}
این بخش شامل \textbf{داده‌های مقداردهی‌شده} \en{(Initialized Data)} است؛ یعنی متغیرهایی که در زمان کامپایل مقدار اولیه دارند. برای مثال، در زبان \en{C} اگر دستور \code{int a = 10;} وجود داشته باشد، مقدار آن در این بخش ذخیره می‌شود.

\textbf{مزایا:}
\begin{itemize}
    \item فراهم‌کردن دسترسی سریع به داده‌های مهم در زمان اجرا
    \item حفظ داده‌های پایدار در طول اجرای برنامه
    \item ساختاردهی مناسب حافظه و جلوگیری از تداخل متغیرها
\end{itemize}

\subsection{\en{.bss}}
بخش \en{.bss} برای \textbf{داده‌های بدون مقدار اولیه} \en{(Uninitialized Data)} استفاده می‌شود. این داده‌ها در زمان بارگذاری برنامه به‌صورت خودکار با مقدار صفر مقداردهی می‌شوند.

\textbf{نمونه:} در برنامه‌ای با دستور \code{int counter;}، متغیر \en{counter} در بخش \en{.bss} قرار می‌گیرد.

\textbf{مزایا:}
\begin{itemize}
    \item کاهش اندازه‌ی فایل اجرایی چون داده‌های بدون مقدار ذخیره نمی‌شوند
    \item تخصیص پویا و بهینه‌ی حافظه در زمان اجرا
    \item تفکیک واضح میان داده‌های مقداردهی‌شده و نشده
\end{itemize}

\subsection{\en{.rdata}}
بخش \en{.rdata} شامل \textbf{داده‌های فقط‌خواندنی} \en{(Read-Only Data)} است؛ مانند رشته‌های ثابت، جداول واردات و صادرات و برخی ساختارهای زمان اجرا.

\textbf{اطلاعات ذخیره‌شده در این بخش:}
\begin{itemize}
    \item نام توابع و کتابخانه‌های وارداتی
    \item داده‌های ثابت و اشاره‌گرهای ثابت
    \item اطلاعات موردنیاز زمان اجرا
\end{itemize}

\textbf{مزایا:}
\begin{itemize}
    \item جلوگیری از تغییر داده‌های حیاتی در زمان اجرا
    \item بهبود امنیت و پایداری برنامه
    \item افزایش کارایی حافظه با امکان اشتراک داده‌ها میان چند فرآیند
\end{itemize}

\subsection{\en{.idata}}
این بخش مربوط به \textbf{جدول واردات} \en{(Import Table)} است و فهرستی از کتابخانه‌ها و توابع خارجی که برنامه از آن‌ها استفاده می‌کند را ذخیره می‌کند.

\textbf{مزایا:}
\begin{itemize}
    \item تسهیل مدیریت وابستگی‌های خارجی
    \item پشتیبانی از بارگذاری پویا \en{(Dynamic Linking)}
    \item امکان به‌روزرسانی کتابخانه‌ها بدون نیاز به تغییر برنامه‌ی اصلی
\end{itemize}

\subsection{\en{.edata}}
بخش \en{.edata} مربوط به \textbf{جدول صادرات} \en{(Export Table)} است. این جدول فهرستی از توابع یا داده‌هایی را نگهداری می‌کند که برنامه برای استفاده‌ی سایر برنامه‌ها در دسترس قرار می‌دهد.

\textbf{مزایا:}
\begin{itemize}
    \item امکان اشتراک توابع و داده‌ها میان چند برنامه
    \item پشتیبانی از ساختار ماژولار نرم‌افزار
    \item کاهش حجم کد تکراری در سیستم
\end{itemize}

\subsection{\en{.rsrc}}
بخش \en{.rsrc} محل ذخیره‌ی \textbf{منابع} \en{(Resources)} مانند آیکون‌ها، منوها، تصاویر، فایل‌های زبان و داده‌های رابط کاربری است. این بخش به برنامه امکان می‌دهد بدون تغییر کد، ظاهر و محتوای خود را به‌روزرسانی کند.

\textbf{اطلاعات موجود در این بخش:}
\begin{itemize}
    \item آیکون‌ها، تصاویر و صداها
    \item اطلاعات نسخه‌ی نرم‌افزار
    \item داده‌های مربوط به زبان و محلی‌سازی \en{(Localization)}
\end{itemize}

\textbf{مزایا:}
\begin{itemize}
    \item جدا کردن منابع از منطق اصلی برنامه
    \item تسهیل تغییر ظاهر یا زبان نرم‌افزار
    \item پشتیبانی از چندزبانگی در نرم‌افزارهای بزرگ
\end{itemize}

\subsection{\en{.reloc}}
بخش \en{.reloc} برای \textbf{اطلاعات جابجایی} \en{(Relocation Information)} استفاده می‌شود. زمانی‌که فایل \en{PE} در آدرس مجازی پیش‌فرض خود بارگذاری نشود، این بخش به سیستم‌عامل کمک می‌کند تا آدرس‌های مطلق اصلاح شوند.

\textbf{مزایا:}
\begin{itemize}
    \item پشتیبانی از بارگذاری انعطاف‌پذیر در حافظه
    \item جلوگیری از تداخل آدرس‌ها میان چند ماژول
    \item بهبود قابلیت استفاده‌ی مجدد از کدها در محیط‌های مختلف
\end{itemize}

\subsection{جمع‌بندی}
هر بخش از فایل \en{PE} نقشی مشخص و حیاتی در عملکرد صحیح برنامه دارد. این تقسیم‌بندی باعث می‌شود سیستم‌عامل بتواند برنامه‌ها را سریع‌تر، ایمن‌تر و با مدیریت بهینه‌ی حافظه اجرا کند.

از بخش هدر که نقشه‌ی کلی فایل را در اختیار \en{Loader} قرار می‌دهد تا بخش‌های کد و داده که منطق برنامه را تشکیل می‌دهند، و همچنین بخش منابع که داده‌های ظاهری و رابط کاربری را نگهداری می‌کند، همگی در کنار هم ساختار منظم و کارآمدی برای اجرای برنامه‌ها در ویندوز ایجاد می‌کنند.

به‌طور کلی، مزیت اصلی ساختار \en{PE} در \textbf{قابلیت حمل، انعطاف‌پذیری، امنیت و ماژولار بودن} آن است؛ ویژگی‌هایی که باعث شده این قالب طی سال‌ها همچنان استاندارد اصلی فایل‌های اجرایی ویندوز باقی بماند.


\label{sec:ch6-sec4}
% chapter6/section4.tex
\section{مسیرهای تحقیقاتی و آموزشی آینده}
\label{sec:ch6-future-work}

با توجه به چالش‌ها و روندهای بررسی‌شده، به‌ویژه در فصول ۳ و ۴، مسیرهای زیر برای تحقیقات و آموزش‌های آتی در حوزه مهندسی نرم‌افزار پیشنهاد می‌شود:

\begin{enumerate}
  \item امنیت در چرخه‌های خودکار(DevSecOps): همان‌طور که در چالش‌های استقرار DevOps اشاره شد، افزایش سرعت استقرار می تواند نگرانی‌های امنیتی ایجاد کند. تحقیقات و آموزش‌های آینده باید بر ادغام یکپارچه امنیت در تمام مراحل چرخه عمر نرم افزار (رویکرد DevSecOps) و روش‌های مدیریت خودکار دسترسی‌ها و داده‌های حساس متمرکز شوند.
  \item کاربرد هوش مصنوعی در بازمهندسی نرم‌افزار: علاوه بر استفاده‌ی فعلی از AI در درک کد، مسیرهای تحقیقاتی آینده باید بر توسعه و ارزیابی مدل‌های هوش مصنوعی برای خودکارسازی فرآیند مهندسی معکوس، استخراج منطق تجاری از کدهای قدیمی ، شناسایی الگوهای ضد طراحی و تولید خودکار مستندات فنی برای سیستم‌های Legacy متمرکز شوند.
  \item مدیریت پیچیدگی ابزارها و زیرساخت‌های DevOps: یکی از موانع استقرار DevOps، پیچیدگی فنی ابزارهایی مانند Kubernetes و Terraform و بار آموزشی سنگین آن‌هاست. تحقیقات آتی می‌تواند بر «ساده‌سازی تعامل» با این ابزارها متمرکز باشد؛ چه از طریق توسعه‌ی پلتفرم‌های سطح بالاتر (PaaS) که به عنوان یک لایه‌ی انتزاعی عمل کرده و پیچیدگی‌های زیرساخت را از توسعه‌دهنده پنهان می‌کنند، و چه از طریق ایجاد روش‌های مدیریتی هوشمندتر و ابزارهای کمکی برای مدیریت بهینه‌ی خود این زیرساخت‌های پیچیده.
  \item توسعه‌ی چارچوب‌های آموزشی و تحقیقاتی برای جنبه‌های انسانی DevOps: با توجه به اینکه «مقاومت فرهنگی» یکی از مهم‌ترین موانع در استقرار موفق DevOps شناسایی شده است، یک خلاء تحقیقاتی و آموزشی آشکار وجود دارد. برنامه‌های آموزشی فعلی اغلب بیش از حد بر ابزارها متمرکز هستند. لذا، مسیرهای تحقیقاتی آینده باید بر توسعه و ارزیابی «مدل‌های مدیریت تغییر» و «تکنیک‌های روانشناسی سازمانی» متمرکز شوند که گذار فرهنگی به DevOps را تسهیل می‌کنند. همچنین، مسیرهای آموزشی آینده باید چارچوب‌هایی مدون برای آموزش مهارت‌های نرم (Soft Skills)، مانند ایجاد فرهنگ گزارش‌دهی بدون سرزنش (Blameless Postmortem) و همکاری بین‌تیمی ، در کنار آموزش‌های فنی ارائه دهند.
  \item الگوهای پیشرفته بازطراحی و مدیریت داده در مهاجرت: با توجه به اهمیت حیاتی سیستم‌های قدیمی (مانند سیستم‌های Core Banking) ، نیاز به الگوها و استراتژی‌های اثبات‌شده برای مدرن‌سازی آن‌ها وجود دارد. تحقیقات آینده می‌تواند بر توسعه‌ی مدل‌هایی برای ارزیابی دقیق ریسک، هزینه و زمان در سناریوهای مختلف بازطراحی ، و همچنین تحقیق بر روی الگوهای مدیریت سازگاری داده‌ها (مانند Saga و Idempotency) در طول مهاجرت افزایشی به معماری میکروسرویس تمرکز کند.
  \item بهبود فرآیندها مبتنی بر داده‌های مانیتورینگ (AIOps): با گسترش ابزارهای نظارت و بازخورد مانند Prometheus و Sentry ، فرصت‌های جدیدی برای استفاده از هوش مصنوعی و تحلیل داده‌های عملیاتی جهت بهبود مستمر فرآیندهای توسعه و تصمیم‌گیری‌های مبتنی بر شاخص‌های کمی (مانند SLOs) فراهم آمده است که نیازمند تحقیق و توسعه‌ی بیشتر است.
\end{enumerate}


\label{sec:ch6-sec5}
\subsection{تعریف آدرس (Address)}
\textbf{آدرس} به مکان منحصر به فردی در حافظه اصلی (RAM) اشاره دارد که یک واحد داده (معمولاً یک بایت) در آن ذخیره شده است. پردازنده (CPU) برای دسترسی به داده‌ها یا دستورالعمل‌ها، از این آدرس‌ها استفاده می‌کند.
\begin{itemize}
\item در معماری‌های قدیمی‌تر مانند \lr{Intel 8086}، آدرس‌ها به صورت \textbf{آدرس فیزیکی} (\lr{Physical Address}) و در معماری‌های مدرن‌تر، اغلب به صورت \textbf{آدرس مجازی} (\lr{Virtual Address}) به برنامه ارائه می‌شوند که توسط واحد مدیریت حافظه (\lr{MMU}) به آدرس فیزیکی ترجمه می‌شوند.
\item آدرس‌دهی مستقیم و مطلق، مستقیماً به یک مکان خاص در حافظه اشاره می‌کند.
\end{itemize}


\subsection{تعریف آفست (Offset)}
\textbf{آفست} (جابجایی یا فاصله) بیانگر \textbf{فاصله} یک بایت داده یا یک دستورالعمل، از یک نقطه شروع مشخص (معمولاً ابتدای یک ساختار، آرایه، رکورد، یا سگمنت در معماری‌های سگمنتی) است.
\begin{itemize}
\item در معماری‌های سگمنتی (مانند \lr{Real Mode پردازنده ۸۰۸۶})، آدرس کامل یک مکان حافظه (\textbf{آدرس فیزیکی}) از ترکیب یک آدرس پایه (\lr{Base Address}) که در ثبات سگمنت ذخیره شده، و آفست به دست می‌آید.
\item آدرس فیزیکی $P$ از طریق فرمول زیر محاسبه می‌شود (در معماری ۸۰۸۶):
$$P = (\text{\lr{Segment Base}} \times 16) + \text{Offset}$$
\item آفست، در واقع، یک آدرس \textbf{نسبی} (\lr{Relative Address}) است.
\end{itemize}

\subsection{کاربرد در تحلیل باینری و مهندسی معکوس}
در تحلیل باینری (\lr{Binary Analysis}) و مهندسی معکوس (\lr{Reverse Engineering})، درک آدرس و آفست از اهمیت حیاتی برخوردار است.
\subsubsection{آدرس‌ها}
\begin{itemize}
\item \textbf{نقشه‌برداری حافظه:} آدرس‌های مجازی و فیزیکی برای تعیین محل قرارگیری توابع، داده‌ها و متغیرها در هنگام اجرای برنامه ضروری هستند.
\item \textbf{\lr{EIP/RIP (Instruction Pointer)}:} ثبات اشاره‌گر دستورالعمل (\lr{EIP در ۳۲ بیت و RIP در ۶۴ بیت}) همواره آدرس دستوری را که CPU قرار است اجرا کند، نگهداری می‌کند. تحلیلگر با بررسی تغییرات این ثبات می‌تواند مسیر اجرای برنامه (\lr{Control Flow}) را دنبال کند.
\item \textbf{\lr{Breakpoints:}} برای توقف اجرای برنامه در یک نقطه خاص (مثلاً ابتدای یک تابع مشکوک)، نیاز است که آدرس دقیق آن مکان در حافظه مشخص شود.
\end{itemize}

\subsubsection{آفست‌ها}
\begin{itemize}
\item \textbf{آدرس مجازی نسبی (RVA):} در ساختار فایل‌های اجرایی مانند \lr{PE (Windows)} و \lr{ELF (Linux)}، بسیاری از آدرس‌ها به صورت \lr{RVA (Relative Virtual Address)} ذخیره می‌شوند که در واقع آفست نسبت به آدرس مبدأ بارگذاری فایل در حافظه (\lr{Image Base}) هستند. این امر جابه‌جایی فایل اجرایی را در حافظه (ASLR) آسان می‌کند.
\item \textbf{تجزیه ساختارها:} آفست‌ها برای دسترسی به فیلدهای یک ساختار داده (مانند ساختارهای ویندوز، یا شیءها در برنامه‌نویسی شیءگرا) ضروری هستند. برای مثال، برای دسترسی به فیلد دوم یک ساختار، باید آفست آن فیلد نسبت به ابتدای ساختار مشخص شود.
\item \textbf{\lr{Buffer Overflow:}} در تحلیل آسیب‌پذیری‌های سرریز بافر (\lr{Buffer Overflow})، تعیین آفست دقیق برای رونویسی آدرس بازگشت (\lr{Return Address}) یا سایر داده‌های حساس (مانند اشاره‌گرهای پشته) یک مرحله کلیدی است.
\end{itemize}

\subsection{مثال‌های عددی}
\label{sec:examples}
\subsubsection{مثال ۱: آدرس‌دهی سگمنتی (8086)}
فرض کنید ثبات سگمنت داده ($DS$) حاوی مقدار $1000h$ و ثبات اندیس مبدأ ($SI$) (که به عنوان آفست عمل می‌کند) حاوی مقدار $00A0h$ باشد.
\begin{itemize}
\item \textbf{\lr{Segment Base (شیفت‌یافته)}:} $1000h \times 10h = 10000h$
\item \textbf{Offset:} $00A0h$
\item \textbf{آدرس فیزیکی:}
$$P = 10000h + 00A0h = 100A0h$$
\end{itemize}
دستور اسمبلی مانند \texttt{MOV AL, [SI]} داده‌ای که در آدرس فیزیکی $100A0h$ قرار دارد را به ثبات $AL$ منتقل می‌کند.
\subsubsection{مثال ۲: محاسبه آفست در آرایه}
در زبان اسمبلی (یا در زبان‌های سطح بالا مانند C) فرض کنید یک آرایه از اعداد صحیح ۴ بایتی (\texttt{int}) به نام \texttt{my_array} در آدرس پایه $B=0x400000$ تعریف شده باشد. برای دسترسی به عنصر شماره $i$ (که اندیس آن از صفر شروع می‌شود)، از آفست استفاده می‌شود.
\begin{itemize}
\item \textbf{فرمول آفست:} $\text{Offset} = i \times \text{\lr{Size of Element}}$
\item \textbf{آدرس عنصر سوم ($i=2$):}
$$\text{Address}(2) = B + (2 \times 4) = 0x400000 + 8 = 0x400008$$
\item در اینجا، مقدار $8$ (بایت) آفست عنصر سوم نسبت به ابتدای آرایه است.
\end{itemize}

\subsection{نکات کاربردی در مهندسی معکوس}
\label{sec:re_tips}
\begin{enumerate}
\item \textbf{محاسبه (RVA):} در مهندسی معکوس، اغلب نیاز است که آدرس‌های مجازی (VA) را به آفست‌های فایل (\lr{File Offsets}) تبدیل کنید تا بتوانید داده‌ها را در فایل باینری روی دیسک مشاهده یا تغییر دهید. این کار با استفاده از جدول سِکشن‌ها (\lr{Section Table}) در هدر فایل \lr{PE/ELF} انجام می‌شود.
\item \textbf{آدرس‌دهی نسبی به (RIP):} در معماری‌های ۶۴ بیتی (x64)، آدرس‌دهی نسبی به ثبات $RIP$ رایج است: \texttt{\lr{MOV EAX, [RIP + offset]}}. تحلیلگر باید مقدار آفست را به آدرس فعلی $RIP$ اضافه کند تا آدرس مقصد را پیدا کند. این امر به ویژه در کدهای مستقل از موقعیت (PIC) بسیار مهم است.
\item \textbf{آفست‌های پشته:} در هنگام بررسی توابع، متغیرهای محلی و پارامترهای تابع با استفاده از آفست‌هایی نسبت به ثبات پایه پشته ($EBP$/$RBP$) یا اشاره‌گر پشته ($ESP$/$RSP$) آدرس‌دهی می‌شوند. پیدا کردن آفست متغیرها نسبت به $RBP$ (مانند \texttt{\lr{[RBP - 0x20]}}) برای درک منطق تابع حیاتی است.
\end{enumerate}



