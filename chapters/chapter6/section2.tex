% chapter6/section2.tex
\section{تأثیر DevOps و بازطراحی بر پایداری نرم‌افزار}
\label{sec:ch6-devops-reeng-sustainability}

پایداری نرم‌افزار، به معنای توانایی سیستم برای ادامه عملکرد صحیح و قابلیت تکامل در طول زمان، به شدت تحت تأثیر چالش‌های نگهداری است. بخش عمده‌ای از هزینه‌های چرخه عمر نرم‌افزار (حدود 60 تا 80 درصد) صرف نگهداری و تکامل می‌شود. همچنین، هزینه‌های پنهان ناشی از بدهی فنی، پایداری بلندمدت پروژه‌ها را تهدید می‌کند.

رویکردهای DevOps و بازطراحی، راهکارهای مستقیمی برای افزایش پایداری و کاهش این هزینه‌ها ارائه می‌دهند. DevOps با خودکارسازی فرآیندهای تست و ادغام (CI) ، باعث کشف سریع خطاها در مراحل اولیه توسعه می‌شود. این امر هزینه‌های نگهداری بلندمدت را به طور قابل توجهی کاهش می‌دهد. فرهنگ DevOps مبتنی بر چرخه‌های بازخورد سریع و مسئولیت مشترک است ؛ رویکردی مانند "You build it, you run it" در آمازون تضمین می‌کند که توسعه‌دهندگان به پایداری محصول پس از استقرار نیز متعهد باشند ، که این امر مستقیماً به ارتقای کیفیت و پایداری سیستم کمک می‌کند.

از سوی دیگر، بازطراحی برای پایداری سیستم‌های قدیمی (Legacy Systems) که اغلب پیچیده و فاقد مستندات هستند، حیاتی است. تکنیک‌هایی مانند Refactoring با بهبود ساختار داخلی کد و افزایش خوانایی، قابلیت نگهداری سیستم را افزایش داده و هزینه اصلاح بدهی فنی را در بلند مدت کاهش می‌دهد. مهاجرت (Migration) نیز به سازمان‌ها اجازه می‌دهد تا از فناوری‌های منسوخ که دارای ریسک‌های امنیتی و عملیاتی هستند، فاصله بگیرند.

در مجموع، DevOps فرآیند تکامل و تحول مداوم را ممکن می‌سازد، در حالی که بازطراحی تضمیمن می‌کند که پایه‌های فنی سیستم برای این تکامل مستمر، مستحکم و قابل نگهداری باقی بمانند.
