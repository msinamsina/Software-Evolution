\section{ساختار کلی فایل PE}

فایل PE دارای یک ساختار سلسله مراتبی ثابت است که با چندین هدر شروع شده و به دنبال آن جدول بخش‌ها و در نهایت محتوای واقعی بخش‌ها قرار می‌گیرند. این ساختار مجموعه‌ای از دستورالعمل‌ها را به لودر پویا (\lr{Dynamic Linker}) ارائه می‌دهد تا نحوه‌ی نقشه‌برداری صحیح فایل در حافظه را تعیین کند.\cite{chapter6ref1}
\begin{figure}[h]
\centering
\includegraphics[width=0.8\textwidth]{Portable_Executable_32_bit_Structure_in_SVG_fixed.jpg}
\caption{شمای کلی یک فایل اجرایی قابل حمل (۳۲ بیتی)}
\end{figure}

\noindent
فایل PE شامل قسمت‌های زیر می‌شود:

\noindent
\subsection{\lr{DOS Header}}
هر فایل PE با هدر DOS شروع می‌شود که ساختاری به طول ۶۴ بایت است. این هدر که برای سازگاری با سیستم‌های ۱۶ بیتی MS-DOS قدیمی طراحی شده، شامل یک امضای استاندارد و یک اشاره‌گر کلیدی است.
بایت‌های اولیه فایل همیشه شامل امضای جادویی MZ (معادل هگز \lr{0x5A4D} و نشان‌دهنده \lr{Mark Zbikowski}، یکی از توسعه‌دهندگان \lr{MS-DOS}) است که اولین نشانه‌ی شناسایی فرمت PE است.
فیلد 4 بایتی e\_lfanew در آفست ثابت \lr{0x3C} از شروع فایل قرار دارد و شامل آفست فایل (\lr{File Offset}) شروع هدرهای NT است.\cite{chapter6ref1} برای سیستم‌عامل‌های مدرن ویندوز، مهم‌ترین وظیفه \lr{DOS Header}، ارائه این اشاره‌گر است. بارگذار ویندوز بلافاصله به این آفست مراجعه کرده و از بقیه هدر DOS و برنامه \lr{DOS Stub} صرف نظر می‌کند تا به ساختار اصلی NT دست یابد.\cite{chapter6ref1}

\noindent
\subsection{\lr{DOS Stub}}
برنامه‌ای کوچک سازگار با \lr{MS-DOS 2.0} که صرفاً یک پیام خطا چاپ می‌کند: \lr{"This program cannot be run in DOS mode"}. این Stub اطمینان می‌دهد که اگر برنامه در محیط DOS اجرا شود، یک پیام معنادار نمایش داده شود.\cite{chapter6ref2}

\noindent
\subsection{\lr{NT Headers}}
هدرهای \lr{NT}، که موقعیت شروع آن توسط e\_lfanew تعیین شده، اطلاعاتی درباره ماشین هدف (\lr{Target Machine}) و ویژگی‌های فایل PE دارد و شامل سه بخش متوالی است:\cite{chapter6ref2}

\noindent
1. \lr{PE Signature}: یک امضای ۴ بایتی که مقدار ثابت \lr{PE00} (\lr{0x50450000}) است که فایل را به عنوان PE شناسایی می‌کند​.\cite{chapter6ref2}

\noindent
2. \lr{COFF File Header}: هدری که اطلاعات کلی درباره فایل مانند معماری هدف (\lr{Machine Type})، تعداد بخش‌ها (NumberOfSections)، timestamp، اندازه Optional Header و Characteristics را نگهداری می‌کند​.\cite{chapter6ref2}
\begin{itemize}
    \setlength{\itemsep}{-0.5em}
    \item \lr{Machine} - مشخص‌کننده معماری CPU هدف است.
    \item \lr{NumberOfSections} - تعداد بخش‌های موجود در فایل را نشان می‌دهد.
    \item \lr{TimeDateStamp} - زمان و تاریخ ایجاد فایل را مشخص می‌کند.
    \item \lr{PointerToSymbolTable} - آدرس (offset) جدول نمادها است، که معمولاً مقدار آن صفر است.
    \item \lr{NumberOfSymbols} - تعداد نمادهای موجود در جدول نمادها را مشخص می‌کند (معمولاً صفر است).
    \item \lr{SizeOfOptionalHeader} - اندازه هدر اختیاری را نشان می‌دهد.
    \item \lr{Characteristics} - شامل پرچم‌هایی است که ویژگی‌های فایل را مشخص می‌کنند (مثلاً فایل اجرایی، DLL و غیره).
\end{itemize}


\noindent
3. \lr{Optional Header}: 
علی‌رغم نام، برای فایل‌های PE الزامی است و شامل اطلاعات مهمی برای بارگذاری (Loading) و اجرای برنامه است. اطلاعات مهمی مانند نقطه ورود برنامه (\lr{Entry Point RVA})، \lr{ImageBase}، اطلاعات \lr{Alignment} و ... .\cite{chapter6ref2}
\begin{itemize}
    \setlength{\itemsep}{-0.5em}
    \item \lr{Magic} - نوع فایل اجرایی را مشخص می‌کند (مانند \lr{PE32} با مقدار \lr{0x010B} یا \lr{PE32+} با مقدار \lr{0x020B} برای ۶۴ بیت).
    \item \lr{MajorLinkerVersion / MinorLinkerVersion} - نسخه لینکِر (Linker) مورد استفاده را نشان می‌دهد.
    \item \lr{SizeOfCode} - اندازه بخش کد را مشخص می‌کند.
    \item \lr{SizeOfInitializedData} - اندازه بخش داده‌های مقداردهی‌شده را مشخص می‌کند.
    \item \lr{AddressOfEntryPoint} - آدرس مجازی نسبی (\lr{RVA}) نقطه ورود برنامه را نشان می‌دهد.
    \item \lr{BaseOfCode / BaseOfData} - آدرس‌های مجازی نسبی (\lr{RVA}) بخش‌های کد و داده را مشخص می‌کنند.
    \item \lr{ImageBase} - آدرس بارگذاری ترجیحی تصویر (Image) در حافظه را مشخص می‌کند.
    \item \lr{SectionAlignment / FileAlignment} - نحوه تراز (Alignment) بخش‌ها در حافظه و روی دیسک را تعیین می‌کند.
    \item \lr{SizeOfImage} - اندازه کل تصویر (Image) در حافظه را نشان می‌دهد.
    \item \lr{SizeOfHeader} - اندازه ترکیبی تمام سربرگ‌ها (Headers) را مشخص می‌کند.
    \item \lr{Subsystem} - زیرسامانه‌ای را مشخص می‌کند که برای اجرای فایل لازم است (مثلاً رابط کاربری گرافیکی ویندوز).
    \item \lr{DllCharacteristics} - شامل پرچم‌هایی است که ویژگی‌های فایل DLL را مشخص می‌کنند.
    \item \lr{SizeOfStackReserve / SizeOfStackCommit} - اندازه رزرو و تعهد (Commit) پشته (Stack) را مشخص می‌کند.
    \item \lr{SizeOfHeapReserve / SizeOfHeapCommit} - اندازه رزرو و تعهد (Commit) پشته حافظه (Heap) را مشخص می‌کند.
    \item \lr{NumberOfRvaAndSizes} - تعداد ورودی‌های موجود در جدول آدرس‌های مجازی نسبی را نشان می‌دهد.
\end{itemize}

\noindent
4. \lr{Data Directories}: 
این بخش به جداول و منابع مختلفی در فایل اشاره می‌کند؛ مانند جدول واردات (\lr{Imports})، صادرات (\lr{Exports})، منابع (\lr{Resources}) و موارد دیگر.\cite{chapter6ref2}
\begin{itemize}
    \setlength{\itemsep}{-0.5em}
    \item \lr{Export Table} - آدرس و اندازهٔ جدول صادرات.
    \item \lr{Import Table} - آدرس و اندازهٔ جدول واردات.
    \item \lr{Resource Table} - آدرس و اندازهٔ جدول منابع.
    \item \lr{Exception Table} - آدرس و اندازهٔ جدول استثناها (Exception).
    \item \lr{Certificate Table} - آدرس و اندازهٔ جدول گواهی امنیتی.
    \item \lr{Base Relocation Table} - آدرس و اندازهٔ جدول بازنشانی پایه (\lr{Relocation}).
    \item \lr{Debug Data} - آدرس و اندازهٔ داده‌های اشکال‌زدایی (\lr{Debug}).
    \item \lr{Architecture Data} - رزرو شده؛ مقدار آن باید صفر باشد.
    \item \lr{Global Pointer Register} - آدرس مجازی نسبی (\lr{RVA}) مقداری که در ثبات اشاره‌گر سراسری ذخیره می‌شود.
    \item \lr{TLS Table} - آدرس و اندازهٔ جدول حافظهٔ محلی رشته‌ها (\lr{Thread-Local Storage}).
    \item \lr{Load Configuration Table} - آدرس و اندازهٔ جدول پیکربندی بارگذاری.
    \item \lr{Bound Import Table} - آدرس و اندازهٔ جدول واردات مقید (\lr{Bound Import}).
    \item \lr{Import Address Table} - آدرس و اندازهٔ جدول آدرس‌های واردات.
    \item \lr{Delay Import Descriptor} - آدرس و اندازهٔ توصیف‌گر واردات تأخیری (\lr{Delay Import}).
    \item \lr{CLR Runtime Header} - آدرس و اندازهٔ هدر زمان‌اجرای \lr{CLR}.
    \item \lr{Reserved} - برای استفاده‌های آینده رزرو شده است.
\end{itemize}

\noindent
\subsection{\lr{Section Headers}}
جدولی از هدرهای بخش (\lr{Section Headers}) که برای هر بخش در فایل یک ورودی دارد. هر \lr{Section Header} شامل نام بخش، \lr{Virtual Address}، \lr{Virtual Size}، \lr{Pointer-To-Raw-Data} و سایر metadata است.\cite{chapter6ref2}

\noindent
\subsection{\lr{Sections}}
بخش‌های اصلی فایل که محتویات واقعی برنامه را شامل می‌شود. بخش‌های استاندارد شامل .text (کد)، .data (داده‌های مقداردهی‌شده)، .rdata (داده‌های فقط‌خوانده)، .rsrc (منابع) و دیگری هستند.\cite{chapter6ref2}


\begin{table}[h!]
    \centering
    \begin{tabular}{llll}
        \toprule
        \textbf{توضیح} & \textbf{حافظه} & \textbf{کاربرد اصلی} & \textbf{نام بخش} \\
        \midrule
        دستورالعمل‌های CPU و کد اصلی برنامه                 & R-X           & کد قابل اجرا       & .text   \\
        متغیرهای جهانی اولیه‌شده                            & RW-           & داده مقداردهی شده  & .data   \\
        رشته‌های ثابت و اطلاعات Import/Export               & R--            & داده فقط‌خوانده    & .rdata  \\
        اسامی DLL‌ها و توابع وارد‌شده                        & R--           & جدول Import        & .idata  \\
        اسامی توابع صادرشده                                & R--           & جدول Export        & .edata  \\
        آیکن‌ها، تصاویر، رشته‌ها، \lr{Dialog}‌های رابط کاربری & R--          & منابع برنامه       & .rsrc   \\
        آدرس‌های نیازمند بازنشانی برای ASLR                 & R--           & اطلاعات Relocation & .reloc  \\
        جداول \lr{exception handling} (فقط ۶۴ بیت)         & R--             & اطلاعات Exception  & .pdata  \\
        \bottomrule
    \end{tabular}
    \caption{بخش‌های مختلف فایل اجرایی (\lr{PE Sections}) و کاربرد آن‌ها}
    \label{tab:pe_sections}
\end{table}

\begin{figure}[h]
\centering
\includegraphics[width=0.8\textwidth]{images/PE Headers annotated.png}
\caption{نمونه کد باینری یک فایل PE}
\end{figure}