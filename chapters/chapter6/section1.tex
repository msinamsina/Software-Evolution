% chapter6/section1.tex
\section{مرور کلی یافته‌ها}
\label{sec:ch6-findings-overview}

در این گزارش، سیر تحول مهندسی نرم‌افزار از روش‌های ابتدایی و فاقد ساختار مانند «کد و فیکس»، به مدل‌های ساختارمند و خطی نظیر «آبشاری»، و در ادامه به رویکردهای تکرار شونده و تکاملی مانند مدل «افزایشی»، «مارپیچی» و در نهایت متدولوژی‌های «چابک» (Agile) و DevOps مورد بررسی قرار گرفت. مشخص شد که هدف اصلی این تکامل، تبدیل توسعه نرم‌افزار از یک فعالیت تجربی به فرآیندی نظام‌مند برای مدیریت پیچیدگی و تولید سیستم‌های نرم‌افزاری قابل اعتماد و با کیفیت بالا بوده است.
در فصل دوم، چالش‌های کلیدی در چرخه توسعه و تکامل نرم‌افزار شناسایی شدند. این مشکلات در سه دسته‌ی اصلی طبقه‌بندی شدند:

\begin{enumerate}
    \item مشکلات سازمانی: شامل ارتباطات ناکارآمد بین تیم‌ها ، مستندسازی ضعیف و مدیریت ناکارآمد تغییرات.
    \item مشکلات فنی: شامل انباشت بدهی فنی (Technical Debt) ، ناسازگاری با فناوری‌های جدید و چالش‌های ناشی از سیستم‌های قدیمی (Legacy Systems).
    \item مشکلات انسانی (شامل فرسودگی تیم و فقدان مهارت‌های جدید): مطالعات موردی پروژه‌های شکست‌خورده مانند LASCAD و (VCF) Virtual Case File نشان داد که عواملی چون تست ناکافی ، طراحی ضعیف و مدیریت تغییرات کنترل‌نشده نقش مستقیمی در شکست پروژه‌های بزرگ داشته‌اند.
\end{enumerate}

در فصل سوم، DevOps به‌عنوان یک فرهنگ ، فلسفه و مجموعه‌ای از ابزارها معرفی شد که با هدف یکپارچه‌سازی تیم‌های توسعه (Dev) و عملیات (Ops) و رفع شکاف میان آن‌ها پدید آمده است. نشان داده شد که DevOps با تکیه بر خودکارسازی فرآیندهای CI/CD و استفاده از ابزارهایی نظیر Docker ، Kubernetes و Jenkins ، منجر به افزایش سرعت تحویل نرم‌افزار ، بهبود پایداری استقرارها و ارتقای کیفیت محصول می‌شود. با این حال، چالش‌هایی نظیر مقاومت فرهنگی و پیچیدگی فنی ابزارها نیز در مسیر استقرار آن وجود دارد.

در نهایت، فصل چهارم به ضرورت باز طراحی (Reengineering) در چرخه عمر نرم‌افزار پرداخت. دلایل اصلی این نیاز، مواردی چون ضعف معماری اولیه (مانند الگوی ضد طراحی «توپ گلی بزرگ» (BBoM) ، منسوخ شدن فناوری‌ها و انباشت بدهی فنی است ؛ بدهی فنی می‌تواند هزینه‌ی فرصت سنگینی ایجاد کند، به‌طوری که توسعه‌دهندگان حدود ۴۲٪ از هفته کاری خود را صرف رسیدگی به آن می‌کنند. تکنیک‌هایی مانند بازآرایی (Refactoring) ، مهندسی معکوس (Reverse Engineering) و مهاجرت (Migration) به‌عنوان ابزارهای کلیدی معرفی شدند. همچنین مشخص شد که استفاده از ابزارهای نوین هوش مصنوعی می‌تواند فرآیند مهندسی معکوس سیستم‌های قدیمی را تسریع بخشد. در اجرای بازطراحی، استراتژی‌های مهاجرت افزایشی (مانند «الگوی انجیر خفه‌کننده») ریسک بسیار کمتری نسبت به رویکرد «انفجار بزرگ» دارند. مطالعات موردی نشان داد که بازطراحی می‌تواند اهداف استراتژیک متفاوتی داشته باشد؛ از بهبود تجربه کاربری و شخصی‌سازی (مانند اپلیکیشن PayPal) تا یکپارچه‌سازی خدمات بانکی و ابزارهای عملیاتی برای پاسخ به نیازهای بازار (مانند نئو بانک فوربیکس). تصمیم‌گیری نهایی برای باز طراحی نیازمند ارزیابی دقیق معيارهايي چون هزینه ، زمان ، ریسک و تاثیر بر کیفیت است.
