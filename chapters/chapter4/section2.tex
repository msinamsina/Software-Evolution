% chapter4/section2.tex
\section{تعریف بازطراحی (Redesign / Reengineering)}
\label{sec:ch4-definition}

بازمهندسی نرم‌افزار، فرآیند بررسی و تغییر یک سیستم موجود با هدف پیاده‌سازی آن در یک فرم جدید یا تطبیق داده شده است. این فرآیند از نرم‌افزار و مستندات موجود استفاده می‌کند تا نیازمندی‌ها و طراحی سیستم هدف را تولید کند.

\subsection{بازمهندسی در برابر مهندسی رو به جلو}
برخلاف مهندسی رو به جلو (Forward Engineering) که با یک سند مشخصات تعریف‌شده آغاز می‌شود، بازمهندسی با سیستم موجود به‌عنوان «مشخصات» خود آغاز شده و از طریق فرآیندهای درک و تبدیل، سیستم هدف را استخراج می‌کند.

\subsection{مهندسی معکوس (بازیابی طراحی)}
مرحله حیاتی که بازمهندسی را تعریف می‌کند، بازیابی طراحی یا مهندسی معکوس (Reverse Engineering) است. این مرحله برای بازیابی منطق و چرایی تصمیمات معماری از دست‌رفته که در طول پیاده‌سازی اولیه اتخاذ شده‌اند، ضروری است. قبل از شروع هرگونه کار فنی، زمینه و هدف بازمهندسی باید در چارچوب اهداف کلان سازمانی تعریف شود.
