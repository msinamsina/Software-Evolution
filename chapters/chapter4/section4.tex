% chapter4/section4.tex
\section{مراحل بازطراحی نرم‌افزار}
\label{sec:ch4-process}

فرآیند بازمهندسی نرم‌افزار یک روش‌شناسی ساختاریافته است که از تحلیل سیستم موجود آغاز شده و تا پیاده‌سازی استراتژی‌های مهاجرت با کمترین ریسک ادامه می‌یابد.

\subsection{تحلیل سیستم فعلی}
ارزیابی جامع سیستم فعلی گام اول است تا مشخص شود کدام بخش‌ها ارزش حفظ کردن دارند. این تحلیل شامل سنجش شاخص‌هایی مانند زمان پاسخ، درصد در دسترس بودن (uptime) و آزمون بار اوج (Peak Load Testing) است. همچنین ارزیابی هزینه کل مالکیت (TCO) برای تحلیل اقتصادی سیستم ضروری است.

\subsection{شناسایی نقاط ضعف}
در غیاب مستندات کامل، مهندسی معکوس برای درک منطق و بازیابی طراحی به‌کار می‌رود. ابزارهای هوش مصنوعی مانند GitHub Copilot می‌توانند وابستگی‌ها، فراخوانی‌ها و ساختارهای منطقی را استخراج کرده و مستندات به‌روز تولید کنند.

\subsection{طراحی مجدد معماری و پیاده‌سازی}
بازطراحی معماری معمولاً شامل گذار از ساختارهای یکپارچه به معماری‌های توزیع‌شده مانند میکروسرویس‌ها است.  
چالش‌های کلیدی این گذار عبارتند از:
\begin{itemize}
    \item افزایش هزینه‌های زیرساختی و تست برای هر سرویس جدید.
    \item از دست رفتن تضمین‌های ACID و نیاز به مدیریت ثبات نهایی (Eventual Consistency).
    \item استفاده از الگوهایی مانند ساگا (Saga) و تضمین هم‌توانایی (Idempotency) برای هماهنگی تراکنش‌ها.
\end{itemize}

\subsection{استراتژی‌های مهاجرت}
انتخاب روش مهاجرت بستگی به میزان تحمل ریسک سازمان دارد:
\begin{itemize}
    \item \textbf{مهاجرت انفجار بزرگ (Big Bang Migration):} سوئیچ فوری از سیستم قدیمی به سیستم جدید؛ پرریسک و مناسب سیستم‌های غیر بحرانی.
    \item \textbf{مهاجرت افزایشی (Incremental Migration) یا الگوی انجیر خفه‌کننده:} جایگزینی تدریجی بخش‌ها و اجرای همزمان سیستم‌های قدیم و جدید برای کاهش ریسک.
\end{itemize}
