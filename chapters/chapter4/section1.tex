% chapter4/section1.tex
\section{مقدمه}
\label{sec:ch4-intro}

سیستم‌های نرم‌افزاری، برخلاف دارایی‌های فیزیکی که دچار فرسایش مکانیکی می‌شوند، به مرور زمان کارایی خود را در انطباق با واقعیت‌های تجاری و بستر فناورانه از دست می‌دهند. این پدیده، که اغلب به آن «کهنگی نرم‌افزاری» گفته می‌شود، منجر به افزایش فزاینده در هزینه‌های عملیاتی و نگهداری می‌گردد. برآوردها نشان می‌دهد که نگهداری نرم‌افزار به‌عنوان پرهزینه‌ترین فاز چرخه حیات نرم‌افزار، تقریباً ۶۰ درصد از کل تلاش‌های صورت گرفته در این چرخه را به خود اختصاص می‌دهد.

سازمان‌ها در محیط‌های رقابتی و نظارتی امروز، تحت فشار مستمر برای افزایش چابکی و پاسخگویی به تغییرات بازار، مقررات جدید، و نیازهای در حال تحول کاربران قرار دارند. زمانی که سیستم‌های قدیمی (Legacy Systems) به مانعی برای نوآوری تبدیل می‌شوند و بخش نامتناسبی از بودجه را مصرف می‌کنند، بازمهندسی (Reengineering) به یک ضرورت استراتژیک تبدیل می‌گردد. هدف از بازمهندسی، نه صرفاً تولید ویژگی‌های جدید، بلکه بازیابی و طولانی کردن عمر سیستم‌های حیاتی است، ضمن کاهش هزینه‌های بالای نگهداری.

بازمهندسی، اساساً یک سرمایه‌گذاری در مدیریت ریسک و بهینه‌سازی مالی محسوب می‌شود. زمانی که هزینه‌های نگهداری بیش از نیمی از بودجه توسعه را می‌بلعد، این هزینه عملاً منابعی را که می‌توانست صرف نوآوری شود، از بین می‌برد (هزینه فرصت). بنابراین، بازمهندسی به عنوان راهکاری برای تثبیت سازمانی، کاهش ریسک‌های شکست سیستمی، و تضمین انطباق با قوانین، بر تحویل ویژگی‌های فوری اولویت می‌یابد.
