\section{مطالعه موردی}
\label{section:7}

بر اساس نتایج جستجو، اطلاعات مربوط به بازطراحی پی‌پال بیشتر بر به‌روزرسانی تجربه کاربری اپلیکیشن و هویت بصری متمرکز است، در حالی که سیستم بانکی به معرفی یک نئوبانک کسب‌وکاری داخلی می‌پردازد. در ادامه، این دو مورد به صورت جداگانه ارائه شده‌اند:

\subsection{بازطراحی اپلیکیشن PayPal}
هدف اصلی از بازطراحی پی‌پال، تبدیل آن از یک ابزار ساده برای انتقال پول به یک "راهنمای سلامتی مالی" شخصی‌شده برای کاربران بود. مشکلات اصلی که این بازطراحی به دنبال رفع آن‌ها بود، شامل صفحه اصلی غیرجذاب، سردرگمی کاربران در پیمایش و عدم اطلاع‌رسانی شفاف بود. راه‌حل‌های کلیدی اجرا شده در این بازطراحی عبارتند از:

\begin{itemize}
    \item \textbf{تجربه شخصی‌سازی‌شده}: صفحه اصلی به فضایی شخصی برای مدیریت امور مالی کاربر تبدیل شد. با نمایش فعالیت‌های حساب و بینش‌های مالی مفید، حس تعلق کاربر به اپلیکیشن تقویت شد.
    \item \textbf{پیمایش ساده‌شده}: نوار پیمایش پایین اپلیکیشن با آیکون‌ها و برچسب‌های قابل درک طراحی شد تا کاربران به راحتی به ویژگی‌های مورد نظر خود دسترسی پیدا کنند.
    \item \textbf{شفافیت اطلاعاتی}: اطلاعات حیاتی مانند موجودی حساب و کارت در حال استفاده، به وضوح و در صفحه اصلی نمایش داده می‌شوند تا از سردرگمی کاربر کاسته شود.
    \item \textbf{تجمیع عملکردها}: عملکردهای مرتبط با پول و حساب در یک بخش گروه‌بندی شدند تا قابلیت کشف و استفاده از آن‌ها برای کاربر آسان‌تر شود.
\end{itemize}

این تغییرات منجر به ایجاد تجربه‌ای شد که با انتظارات کاربران مطابقت بیشتری دارد، قابلیت کشف ویژگی‌ها را بهبود بخشید و از طریق شفافیت، اعتماد کاربران را افزایش داد. علاوه بر این، پی‌پال هویت بصری برند خود را نیز به‌روز کرد که شامل طراحی قلم سفارشی "PayPal Pro" و ساده‌سازی پالت رنگی برای نمایشی مدرن‌تر و خوش‌بینانه‌تر بود.

\subsection{بازطراحی بانکداری برای کسب‌وکارهای کوچک (فوربیکس)}
در ایران، نمونه بارز بازطراحی در سیستم بانکی، ظهور "نئوبانک‌های کسب‌وکاری" مانند فوربیکس است. هدف فوربیکس، بازطراحی خدمات بانکی برای پاسخگویی به نیازهای خاص کسب‌وکارهای کوچک و متوسط بود که اغلب توسط سیستم بانکی سنتی نادیده گرفته می‌شوند. ویژگی‌های کلیدی این بازطراحی شامل:

\begin{itemize}
    \item \textbf{یکپارچگی خدمات}: فوربیکس خدمات بانکی (مانند افتتاح حساب و درگاه پرداخت) را با ابزارهای عملیاتی کسب‌وکار (مانند سیستم حسابداری، صدور فاکتور، CRM و مدیریت انبار) در یک پلتفرم واحد ادغام کرد.
    \item \textbf{تمرکز بر کاربرپسندی}: این پلتفرم با ارائه یک اپلیکیشن موبایل با رابط کاربری ساده، تجربه مالی ساده‌ای را برای صاحبان کسب‌وکارها فراهم می‌کند.
    \item \textbf{اتوماسیون برای صرفه‌جویی در زمان}: با اتصال خودکار تراکنش‌های بانکی به سیستم حسابداری، فرآیندهای دستی کاهش یافته و تا ۴۰ درصد در زمان صرفه‌جویی می‌شود.
\end{itemize}

این رویکرد یکپارچه، چالش‌های کسب‌وکارها در استفاده همزمان از سیستم‌های ناهمگون بانکی و حسابداری را برطرف کرده و مدیریت امور مالی و عملیاتی را برای آن‌ها بسیار کارآمدتر کرده است.
