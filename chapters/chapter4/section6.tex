\section{معیارهای تصمیم‌گیری برای بازطراحی}
\label{section:6}

تصمیم‌گیری برای انجام فرآیند بازطراحی یک سیستم نرم‌افزاری، نیازمند سنجش و ارزیابی دقیق چندین معیار حیاتی است تا بتوان توجیه فنی و اقتصادی آن را به درستی بررسی کرد. مهم‌ترین این معیارها عبارتند از:

\subsection{هزینه (Cost)}
برآورد دقیق تمامی هزینه‌های مستقیم و غیرمستقیم پروژه بازطراحی امری ضروری است. این هزینه‌ها شامل دستمزد تیم توسعه، هزینه‌های مربوط به خرید یا اجاره ابزارها و زیرساخت‌های جدید، هزینه‌های آموزش پرسنل و همچنین هزینه‌های احتمالی توقف یا کاهش عملکرد سیستم در حین اجرای پروژه می‌شود. این معیار باید در مقابل هزینه‌های ادامه کار با سیستم قدیمی (مانند هزینه‌های بالای نگهداری و رفع نقص) سنجیده شود.

\subsection{زمان (Time)}
تخمین مدت زمان مورد نیاز برای تکمیل فرآیند بازطراحی از اهمیت بالایی برخوردار است. یک برنامه‌ریزی واقع‌بینانه باید شامل مراحل تحلیل، طراحی، پیاده‌سازی، تست و استقرار باشد. زمان‌بندی طولانی می‌تواند منجر به منسوخ شدن فناوری‌های به کار رفته در طول اجرای پروژه شود، در حالی که زمان‌بندی بسیار فشرده نیز کیفیت نهایی را به خطر می‌اندازد.

\subsection{ریسک (Risk)}
ارزیابی ریسک‌های بالقوه در موفقیت پروژه بازطراحی یک گام کلیدی است. این ریسک‌ها می‌توانند شامل پیچیدگی فنی بالای سیستم legacy، از دست دادن مهارت‌های تخصصی مورد نیاز، بروز مشکلات غیرمنتظره در حین مهاجرت داده‌ها، و مقاومت کاربران در برابر پذیرش سیستم جدید باشد. شناسایی این ریسک‌ها و برنامه‌ریزی برای مدیریت آن‌ها شانس موفقیت پروژه را افزایش می‌دهد.

\subsection{اثر بر کیفیت (Effect on Quality)}
در نهایت، باید تأثیر مثبت بازطراحی بر کیفیت محصول نهایی به وضوح تعریف و اندازه‌گیری شود. این بهبود کیفیت می‌تواند به صورت افزایش کارایی (Performance)، افزایش قابلیت اطمینان (Reliability)، افزایش امنیت، بهبود قابلیت نگهداری (Maintainability) و افزایش قابلیت گسترش (Scalability) سیستم ظاهر شود. این معیار نهایی، توجیه اصلی برای سرمایه‌گذاری روی پروژه بازطراحی محسوب می‌شود.
