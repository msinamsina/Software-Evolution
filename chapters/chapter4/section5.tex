\section{ابزارها و تکنیک‌های بازطراحی}
\label{section:5}

فرآیند بازطراحی و نگهداری سیستم‌های قدیمی معمولاً با به کارگیری مجموعه‌ای از ابزارها و تکنیک‌های تخصصی انجام می‌شود که هدف نهایی آن‌ها افزایش کارایی، قابلیت نگهداری و طول عمر نرم‌افزار است. سه مورد کلیدی در این حوزه عبارتند از:

\subsection{بازآرایی (Refactoring)}
بازآرایی فرآیند بازسازی (restructuring) ساختار داخلی کد بدون تغییر رفتار خارجی آن است. هدف اصلی بهبود طراحی، خوانایی و قابلیت نگهداری کد است که در نهایت توسعه‌ی ویژگی‌های جدید را آسان‌تر می‌کند. این کار اغلب با انجام تغییرات کوچک و مطمئن مانند تغییر نام متغیرها، استخراج متدها و ساده‌سازی شرط‌ها انجام می‌شود. ابزارهای مدرن یکپارچه در محیط‌های توسعه (IDE) مانند قابلیت‌های بازآرایی در IntelliJ IDEA یا ReSharper برای \#C، این فرآیند را به صورت خودکار و ایمن انجام می‌دهند \cite{ibm-refactoring}.

\subsection{مهندسی معکوس (Reverse Engineering)}
مهندسی معکوس فرآیند استخراج طراحی، معماری و مشخصات یک سیستم از کد منبع موجود است. این تکنیک به ویژه برای درک سیستم‌های legacy که فاقد مستندات کافی هستند، حیاتی است. ابزارهای این حوزه مانند ابزارهای تولید نمودارهای UML از کد (مانند Enterprise Architect یا ابزارهای موجود در Visual Studio) یا دیس اسمبلرها، لایه‌های مختلف سیستم را آشکار کرده و درک آن را برای تیم‌های توسعه ممکن می‌سازند \cite{geeks-reverse-engineering}.

\subsection{مهاجرت (Migration)}
مهاجرت به فرآیند انتقال یک سیستم نرم‌افزاری از یک محیط تکنولوژیکی قدیمی به یک محیط جدیدتر و قدرتمندتر اطلاق می‌شود. این امر می‌تواند شامل مهاجرت پایگاه داده (مانند انتقال از Oracle به PostgreSQL)، مهاجرت سکو (مانند انتقال یک برنامه از ویندوز به وب) یا حتی مهاجرت زبان برنامه‌نویسی باشد. ابزارهای اتوماسیون این فرآیند، ریسک و تلاش انسانی مورد نیاز را به شدت کاهش می‌دهند. موفقیت این فرآیند وابسته به برنامه‌ریزی دقیق، اجرای مرحله‌ای و تست گسترده برای اطمینان از حفظ یکپارچگی داده‌ها و عملکرد سیستم است \cite{geeks-migration}.
