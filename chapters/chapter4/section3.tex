% chapter4/section3.tex
\section{دلایل اصلی نیاز به بازطراحی}
\label{sec:ch4-reasons}

\subsection{تغییر نیازمندی‌ها}
تغییر در نیازمندی‌ها (ناشی از تکامل نیازهای کاربر یا تغییر بازار و مقررات) اجتناب‌ناپذیر است و ایجاد تغییرات دیرهنگام می‌تواند هزینه‌های توسعه را تا ۳۰ درصد افزایش دهد. استراتژی دفاعی، طراحی معماری بر اساس تجزیه مبتنی بر نوسان (Volatility-Based Decomposition) است. این اصل مستلزم کپسوله‌سازی اجزای مستعد تغییر برای جلوگیری از نشت تغییرات در سراسر سیستم و افزایش مقاومت در برابر انحراف ویژگی است. در غیر این صورت، سیستم با بدهی معماری روبه‌رو می‌شود.

\subsection{فناوری‌های جدید}
پذیرش فناوری‌های جدید نیروی محرکه قوی برای بازمهندسی است و اغلب برای رفع محدودیت‌های عملکردی و مقیاس‌پذیری سیستم‌های قدیمی ضروری است. این شامل گذار به برنامه‌های ابرمحور (Cloud-Native) و معماری میکروسرویس‌ها می‌شود. زیرساخت‌های قدیمی IT اغلب نیازمند سفارشی‌سازی‌های گسترده و راه‌حل‌های میان‌افزار پیچیده هستند تا قابلیت همکاری با ابزارهای دیجیتال جدید تضمین شود.

\subsection{ضعف معماری اولیه}
نیاز به بازطراحی اغلب ریشه در پذیرش الگوهای ضدطراحی (Anti-Patterns) دارد. بدنام‌ترین آن، الگوی «توپ گلی بزرگ» (Big Ball of Mud - BBoM) است که در آن ساختار سیستم فاقد تفکیک مسئولیت‌ها و سازماندهی مشخص است. این وضعیت باعث انباشت بدهی فنی شده و اصلاح سیستم را دشوار می‌کند.

\subsection{انباشت بدهی فنی}
بدهی فنی، هزینه استعاری تصمیم‌های کوتاه‌مدت است. همانند بدهی مالی، بهره‌مند است و با گذشت زمان رشد می‌کند. بر اساس گزارش‌ها، حدود ۴۲٪ از زمان توسعه‌دهندگان صرف مقابله با بدهی فنی می‌شود. زمانی که بدهی فنی از ۵۰٪ ارزش فناوری یک سیستم فراتر رود، بازمهندسی کامل از نظر اقتصادی توجیه‌پذیر است.
