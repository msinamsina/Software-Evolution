\section{مشکلات انسانی}

عوامل انسانی شاید کم‌اهمیت‌تر از مشکلات فنی یا سازمانی به نظر برسند، اما در واقع یکی از تأثیرگذارترین عوامل بر موفقیت پروژه‌های نرم‌افزاری محسوب می‌شوند. نرم‌افزار توسط انسان‌ها طراحی، توسعه و نگهداری می‌شود، و کیفیت عملکرد انسانی مستقیماً بر کیفیت محصول نهایی اثر می‌گذارد.

\subsection{فرسودگی تیم}

تحویل‌های پی‌درپی، ساعات کاری طولانی، فشار زمان و عدم توازن میان زندگی شخصی و کاری باعث فرسودگی شغلی می‌شود. فرسودگی منجر به کاهش تمرکز، افت کیفیت کد و افزایش نرخ خروج کارکنان می‌شود. فرهنگ کاری سالم نقش اساسی در حفظ پایداری نیروی انسانی دارد.

\subsection{فقدان مهارت‌های جدید}

سرعت بالای پیشرفت فناوری باعث می‌شود سازمان‌هایی که در آموزش نیروی انسانی سرمایه‌گذاری نکنند، دیر یا زود با بحران مهارت مواجه شوند. کارگاه‌های آموزشی، اشتراک دانش و استفاده از مشاورهای فنی به حفظ انگیزه و رشد مهارتی کمک می‌کنند.

\section*{مشکلات در نگهداری و تکامل سیستم‌های قدیمی \en{(Legacy Systems)}}
Legacy Systems به نرم‌افزارها یا سیستم‌های قدیمی گفته می‌شود که هنوز در سازمان‌ها و شرکت‌ها مورد استفاده قرار می‌گیرند، اما با فناوری‌های مدرن سازگار نیستند یا از استانداردهای فعلی فاصله دارند؛ به عبارت ساده‌تر، \en{Legacy Systems} سیستم‌هایی هستند که عملکردشان هنوز ادامه دارد، اما به دلیل قدیمی بودن فناوری، پیچیدگی کد یا کمبود مستندات، نگهداری و تکامل آن‌ها با چالش‌های قابل توجهی همراه است. به دلیل اینکه بسیاری از زیرساخت ها و نرم افزارهای سازمان ها و شرکت ها را تشکیل می‌دهند، از اهمیت بالایی برخوردار هستند. در ادامه به چند مورد از دلایل اهمیت آن می‌پردازیم:

\textbf{پایداری عملیاتی سازمان:} بسیاری از فرآیندهای حیاتی سازمان‌ها به \en{Legacy Systems} وابسته هستند. توقف یا خرابی این سیستم‌ها می‌تواند باعث اختلال جدی در عملکرد سازمان شود.

\textbf{حفظ سرمایه‌گذاری‌های گذشته:} توسعه و پیاده‌سازی سیستم‌های بزرگ و پیچیده نیازمند هزینه و زمان زیادی بوده است. نگهداری این سیستم‌ها به سازمان‌ها اجازه می‌دهد سرمایه‌گذاری‌های قبلی خود را حفظ کنند و از دوباره‌کاری جلوگیری شود.

\textbf{تداوم خدمات به کاربران:} سیستم‌های قدیمی غالباً به‌صورت مستقیم با کاربران نهایی یا مشتریان سروکار دارند. نگهداری این سیستم‌ها باعث می‌شود کیفیت خدمات و رضایت کاربران حفظ شود.

\textbf{تسهیل تکامل تدریجی:} Legacy Systems می‌توانند پایه‌ای برای تکامل نرم‌افزار و افزودن قابلیت‌های جدید باشند. با نگهداری و بازسازی تدریجی، می‌توان آن‌ها را با فناوری‌های جدید یکپارچه کرد.

\textbf{کاهش ریسک‌های عملیاتی:} جایگزینی یک سیستم قدیمی با نرم‌افزار جدید همیشه پرریسک است. نگهداری و بهبود سیستم‌های موجود، ریسک‌های مربوط به مهاجرت و بازنویسی کامل را کاهش می‌دهد.

\subsection*{مشکلات نگهداری و تکامل سیستم های قدیمی \en{(Legacy Systems)}}
\begin{itemize}
\item \textbf{پیچیدگی و ساختار قدیمی کد:} نرم‌افزارهای Legacy اغلب با فناوری‌ها و الگوهای قدیمی توسعه یافته‌اند که ساختار کد را پیچیده می‌کند و درک و اصلاح آن‌ها برای توسعه‌دهندگان جدید دشوار است.
\item \textbf{وابستگی به فناوری‌های منسوخ:} زبان‌های برنامه‌نویسی، پایگاه داده‌ها و سیستم‌های عامل قدیمی دیگر پشتیبانی نمی‌شوند.
\item \textbf{کمبود مستندات و دانش فنی:} مستندات کامل و به‌روز اغلب وجود ندارد و خروج افراد با تجربه باعث از بین رفتن دانش کلیدی می‌شود.
\item \textbf{سازگاری محدود با سیستم‌های جدید:} سیستم‌ها معمولاً طراحی نشده‌اند تا با فناوری‌های جدید کار کنند.
\item \textbf{هزینه و زمان بالا برای نگهداری و تکامل:} هر تغییر یا بهبود در سیستم‌های Legacy نیازمند تست گسترده و زمان طولانی است.
\end{itemize}

\textbf{نمونه:} سیستم‌های Core Banking در بانک‌های بزرگ ایران مثل بانک ملی که در دهه ۷۰ و ۸۰ با COBOL و Delphi توسعه پیدا کرده‌اند و هنوز در عملیات پایه بانکی پایدار هستند.

\subsection*{هزینه‌های نگهداری و تکامل}
بخش نگهداری و تکامل حدود ۶۰ تا ۸۰ درصد از کل چرخه نرم‌افزار را به خود اختصاص می‌دهد:
\begin{itemize}
\item \textbf{هزینه‌های نیروی انسانی:} نیازمند متخصصان با تجربه در فناوری‌های منسوخ.
\item \textbf{هزینه‌های تست و تضمین کیفیت:} طراحی و اجرای تست‌های عملکردی، امنیتی و یکپارچگی سیستم.
\item \textbf{هزینه‌های ابزار و فناوری:} استفاده از ابزارهای مدیریت نگهداری، پایگاه داده و نسخه‌بندی.
\item \textbf{هزینه‌های بدهی فنی (Technical Debt):} ناشی از تصمیمات کوتاه‌مدت توسعه.
\item \textbf{هزینه‌های یکپارچه‌سازی و مهاجرت:} افزودن قابلیت‌های جدید یا مهاجرت سیستم‌های قدیمی به مدرن.
\end{itemize}
