\section{مشکلات انسانی}

عوامل انسانی شاید کم‌اهمیت‌تر از مشکلات فنی یا سازمانی به نظر برسند، اما در واقع یکی از تأثیرگذارترین عوامل بر موفقیت پروژه‌های نرم‌افزاری محسوب می‌شوند. نرم‌افزار توسط انسان‌ها طراحی، توسعه و نگهداری می‌شود، و کیفیت عملکرد انسانی مستقیماً بر کیفیت محصول نهایی اثر می‌گذارد.

\subsection{فرسودگی تیم}

تحویل‌های پی‌درپی، ساعات کاری طولانی، فشار زمان و عدم توازن میان زندگی شخصی و کاری باعث فرسودگی شغلی می‌شود. فرسودگی منجر به کاهش تمرکز، افت کیفیت کد و افزایش نرخ خروج کارکنان می‌شود. فرهنگ کاری سالم نقش اساسی در حفظ پایداری نیروی انسانی دارد.

\subsection{فقدان مهارت‌های جدید}

سرعت بالای پیشرفت فناوری باعث می‌شود سازمان‌هایی که در آموزش نیروی انسانی سرمایه‌گذاری نکنند، دیر یا زود با بحران مهارت مواجه شوند. کارگاه‌های آموزشی، اشتراک دانش و استفاده از مشاورهای فنی به حفظ انگیزه و رشد مهارتی کمک می‌کنند.
