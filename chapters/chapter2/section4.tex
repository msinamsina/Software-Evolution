
\section*{مشکلات در نگهداری و تکامل سیستم‌های قدیمی (Legacy Systems)}
Legacy Systems به نرم‌افزارها یا سیستم‌های قدیمی گفته می‌شود که هنوز در سازمان‌ها و شرکت‌ها مورد استفاده قرار می‌گیرند، اما با فناوری‌های مدرن سازگار نیستند یا از استانداردهای فعلی فاصله دارند؛ به عبارت ساده‌تر، Legacy Systems سیستم‌هایی هستند که عملکردشان هنوز ادامه دارد، اما به دلیل قدیمی بودن فناوری، پیچیدگی کد یا کمبود مستندات، نگهداری و تکامل آن‌ها با چالش‌های قابل توجهی همراه است اما به دلیل اینکه بسیاری از زیرساخت ها و نرم افزار های سازمان ها و شرکت ها را در تشکیل می دهد بنابراین از اهمیت بالایی برخوردار است در ادامه به چند مورد از دلایل اهمیت آن می پردازیم:

\textbf{پایداری عملیاتی سازمان:} بسیاری از فرآیندهای حیاتی سازمان‌ها به Legacy Systems وابسته هستند. توقف یا خرابی این سیستم‌ها می‌تواند باعث اختلال جدی در عملکرد سازمان شود.


\textbf{حفظ سرمایه‌گذاری‌های گذشته:} توسعه و پیاده‌سازی سیستم‌های بزرگ و پیچیده نیازمند هزینه و زمان زیادی بوده است. نگهداری این سیستم‌ها به سازمان‌ها اجازه می‌دهد سرمایه‌گذاری‌های قبلی خود را حفظ کنند و از دوباره‌کاری جلوگیری شود.

\textbf{تداوم خدمات به کاربران:} سیستم‌های قدیمی غالباً به‌صورت مستقیم با کاربران نهایی یا مشتریان سروکار دارند. نگهداری این سیستم‌ها باعث می‌شود کیفیت خدمات و رضایت کاربران حفظ شود.

\textbf{تسهیل تکامل تدریجی:} Legacy Systems می‌توانند پایه‌ای برای تکامل نرم‌افزار و افزودن قابلیت‌های جدید باشند. با نگهداری و بازسازی تدریجی، می‌توان آن‌ها را با فناوری‌های جدید یکپارچه کرد و از ایجاد یک سیستم کاملاً جدید جلوگیری نمود.

\textbf{کاهش ریسک‌های عملیاتی:} جایگزینی یک سیستم قدیمی با نرم‌افزار جدید همیشه پرریسک است. نگهداری و بهبود سیستم‌های موجود، ریسک‌های مربوط به مهاجرت و بازنویسی کامل را کاهش می‌دهد.

\subsection*{مشکلات نگهداری و تکامل سیستم های قدیمی (Legacy Systems)}
\begin{itemize}
\item \textbf{پیچیدگی و ساختار قدیمی کد:} نرم‌افزارهای Legacy اغلب با فناوری‌ها و الگوهای قدیمی توسعه یافته‌اند که ساختار کد را پیچیده می‌کند و درک و اصلاح این کدها برای توسعه‌دهندگان جدید دشوار است و تغییرات کوچک ممکن است باعث ایجاد خطاهای غیر منتظره شود.

\item \textbf{وابستگی به فناوری‌های منسوخ:} زبان‌های برنامه‌نویسی، پایگاه داده‌ها و سیستم‌های عامل قدیمی دیگر پشتیبانی نمی‌شوند و این مسئله مانع از به‌کارگیری ابزارها و تکنولوژی‌های مدرن برای توسعه، تست و مستندسازی می‌شود.

\item \textbf{کمبود مستندات و دانش فنی:} مستندات کامل و به‌روز اغلب وجود ندارد یا ناقص است و در بسیاری از موارد، تنها توسعه‌دهندگان اولیه یا افراد با تجربه می‌توانند سیستم را درک و نگهداری کنند و خروج این افراد از سازمان باعث از بین رفتن دانش کلیدی و دشوار شدن نگهداری می‌شود.

\item \textbf{سازگاری محدود با سیستم‌های جدید:} Legacy System معمولاً طراحی نشده‌اند تا با سیستم‌ها یا فناوری‌های جدید هم‌زمان کار کنند و این محدودیت باعث می‌شود افزودن قابلیت‌های جدید یا یکپارچه‌سازی با نرم‌افزارهای مدرن، بسیار پیچیده و پرهزینه باشد.
\item \textbf{هزینه و زمان بالا برای نگهداری و تکامل:} هر تغییر یا بهبود در سیستم‌های Legacy نیازمند تست گسترده و زمان طولانی برای اطمینان از عدم تأثیر بر بخش‌های دیگر سیستم است و این مسئله باعث افزایش هزینه‌های نگهداری و کاهش انعطاف‌پذیری سازمان می‌شود.
یک نمونه برای سیستم های قدیمی که در ایران است سیستم های core banking که در بانک های بزرگ ایران مثل بانک ملی است که در دهه 70 و 80 با استفاده از زبان هایی مثل COBOL و Delphi توسعه پیدا کردن و هنوز هم در عملیات پایه بانکی مثل صورتحساب کار میکند و پایدار است.

\end{itemize}



\subsection*{هزینه‌های نگهداری و تکامل}
به صورت کلی بخش نگهداری و تکامل حدودا بین 60 تا 80 درصد از کل چرخه ی نرم افزار رو به خودش اختصاص می دهد و به صورت کلی به بخش های زیر تقسیم می شود:

\begin{itemize}
\item \textbf{هزینه‌های نیروی انسانی:} توسعه‌دهندگان و مهندسان نرم‌افزار برای نگهداری، اصلاح خطاها و افزودن قابلیت‌ها نیازمند تخصص هستند همچنین سیستم‌های قدیمی (Legacy Systems) نیازمند متخصصان با تجربه در فناوری‌های منسوخ هستند که دسترسی به آن‌ها محدود و هزینه‌بر است.

\item \textbf{هزینه‌های تست و تضمین کیفیت:} هر تغییر در نرم‌افزار نیازمند تست گسترده برای اطمینان از عدم تأثیر منفی بر سایر بخش‌هاست و این فرآیند شامل طراحی و اجرای تست‌های عملکردی، امنیتی و یکپارچگی سیستم است و هزینه و زمان قابل توجهی می‌طلبد.

\item \textbf{هزینه‌های ابزار و فناوری:} استفاده از ابزارهای مدیریت نگهداری، پایگاه داده، سیستم‌های نسخه‌بندی و پایش نرم‌افزار هزینه‌بر است ؛ به‌ویژه در سیستم‌های قدیمی، هزینه یکپارچه‌سازی با ابزارهای مدرن و فناوری‌های جدید بالا است.
\item \textbf{هزینه‌های بدهی فنی (Technical Debt):} بدهی فنی ناشی از تصمیمات توسعه‌ای کوتاه‌مدت است که در بلندمدت باعث افزایش پیچیدگی و هزینه نگهداری می‌شود. اصلاح بدهی فنی ممکن است شامل بازنویسی بخش‌هایی از سیستم یا بهبود ساختار کد باشد که هزینه و زمان بالایی دارد
\item \textbf{هزینه‌های یکپارچه‌سازی و مهاجرت:} افزودن قابلیت‌های جدید یا اتصال نرم‌افزار به سیستم‌های مدرن نیازمند یکپارچه‌سازی پیچیده است و در برخی موارد، سازمان‌ها مجبورند سیستم‌های قدیمی را به تدریج با سیستم‌های جدید جایگزین کنند که این فرایند پرهزینه است.

\end{itemize}
