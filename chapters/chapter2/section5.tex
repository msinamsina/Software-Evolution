
\section{روش‌های کاهش مشکلات}
\subsection*{مدیریت تغییرات}
مدیریت تغییرات (Change Management) به مجموعه فرآیندهایی گفته می‌شود که هدف آن کنترل، مستندسازی و پیگیری تغییرات نرم‌افزار است.

\textbf{اهمیت مدیریت تغییرات:}
\begin{itemize}
\item جلوگیری از خطاهای ناشی از تغییرات غیرمستند یا غیرکنترل‌شده.
\item تضمین سازگاری تغییرات با سیستم‌های موجود و فرآیندهای سازمان.
\item امکان پیگیری و بازگشت به نسخه‌های قبلی در صورت بروز مشکل.
\item کاهش زمان و هزینه نگهداری با شناسایی سریع مشکلات.
\end{itemize}

\textbf{اصول مدیریت تغییرات:}
\begin{itemize}
\item ثبت و مستندسازی تغییرات.
\item بررسی و تأیید تغییرات قبل از اعمال.
\item تست پیش از اجرا (واحد، یکپارچگی، عملکرد).
\item پیگیری و گزارش‌دهی پس از اعمال تغییرات.
\item بازگشت به نسخه قبلی (Rollback Plan).
\end{itemize}

\subsection{Refactoring و بازطراحی جزئی}
Refactoring به فرآیند بازسازی کد نرم‌افزار بدون تغییر رفتار خارجی آن گفته می‌شود.

\textbf{اهمیت Refactoring:}
\begin{itemize}
\item کاهش پیچیدگی و افزایش خوانایی کد.
\item کاهش خطاهای نرم‌افزاری.
\item افزایش انعطاف‌پذیری سیستم.
\item کاهش هزینه نگهداری در بلندمدت.
\end{itemize}

\textbf{اصول Refactoring:}
\begin{itemize}
\item تغییر تدریجی و بخش‌بخش.
\item تست مستمر قبل و بعد از هر تغییر.
\item مستندسازی تغییرات.
\item استفاده از الگوهای طراحی و استانداردهای کدنویسی.
\end{itemize}

\subsection{Continuous Integration (ادغام مداوم)}
ادغام مداوم یک رویکرد در مهندسی نرم‌افزار است که توسعه‌دهندگان به‌طور مکرر تغییرات خود را در مخزن اصلی کد منبع ادغام می‌کنند.
\textbf{اهمیت:}
\begin{itemize}
\item کشف سریع خطاها.
\item کاهش هزینه‌های نگهداری.
\item افزایش کیفیت نرم‌افزار.
\item سهولت در تکامل نرم‌افزار.
\end{itemize}

\textbf{اصول و ابزارهای ادغام مداوم:}
\begin{itemize}
\item مخزن مشترک کد منبع (GitHub/GitLab).
\item تست خودکار پس از هر Commit.
\item استفاده از سرور CI (Jenkins, GitLab CI/CD, Travis CI, GitHub Actions).
\item بازخورد سریع به توسعه‌دهندگان.
\end{itemize}
