\section{روش‌های کاهش مشکلات در تکامل نرم‌افزار}
\label{sec:solution-methods}

\subsection{مدیریت تغییرات (Change Management)}
مدیریت تغییرات به مجموعه فرآیندهایی گفته می‌شود که هدف آن کنترل، مستندسازی و پیگیری تغییرات نرم‌افزار است. این روش نقش کلیدی در کاهش مشکلات و چالش‌های نگهداری و تکامل نرم‌افزار دارد، به ویژه در سیستم‌های پیچیده و قدیمی.

\textbf{اهمیت مدیریت تغییرات:}
\begin{itemize}
    \item جلوگیری از خطاهای ناشی از تغییرات غیر مستند یا غیرکنترل‌شده.
    \item تضمین سازگاری تغییرات با سیستم‌های موجود و فرآیندهای سازمان.
    \item امکان پیگیری و بازگشت به نسخه‌های قبلی در صورت بروز مشکل.
    \item کاهش زمان و هزینه نگهداری با شناسایی سریع مشکلات ناشی از تغییرات.
\end{itemize}

\textbf{اصول مدیریت تغییرات:}
\begin{itemize}
    \item \textbf{ثبت و مستندسازی تغییرات:} هر تغییر باید به صورت کامل ثبت شود، شامل هدف تغییر، بخش‌های تأثیرپذیر و روش اجرای آن.
    \item \textbf{بررسی و تأیید تغییرات:} تغییرات باید قبل از اعمال، توسط تیم فنی و مدیران پروژه بررسی و تأیید شوند تا از تداخل با بخش‌های دیگر سیستم جلوگیری شود.
    \item \textbf{تست پیش از اجرا:} قبل از اعمال تغییرات در محیط تولید، تست‌های لازم (واحد، یکپارچگی، عملکرد) انجام شود تا خطاها پیش از مواجهه با کاربران شناسایی شوند.
    \item \textbf{پیگیری و گزارش‌دهی:} پس از اعمال تغییرات، باید پیگیری عملکرد سیستم و ثبت مشکلات احتمالی انجام شود تا تجربه برای تغییرات آینده ذخیره شود.
    \item \textbf{بازگشت به نسخه قبلی (Rollback Plan):} هر تغییر باید قابلیت بازگشت سریع به نسخه پایدار قبلی را داشته باشد تا در صورت بروز خطا، سیستم از کار نیفتد.
\end{itemize}


[Image of the change management process lifecycle]


\subsection{بازسازی کد (Refactoring) و بازطراحی جزئی}
\label{ssec:refactoring}
$\text{Refactoring}$ به فرآیند بازسازی کد نرم‌افزار بدون تغییر رفتار خارجی آن گفته می‌شود. هدف اصلی این روش، بهبود ساختار داخلی کد، کاهش پیچیدگی و افزایش قابلیت نگهداری است.

\textbf{اهمیت $\text{Refactoring}$:}
\begin{itemize}
    \item کاهش پیچیدگی و افزایش خوانایی کد: با ساده‌سازی ساختار کد، توسعه‌دهندگان می‌توانند تغییرات را سریع‌تر و با ریسک کمتر اعمال کنند.
    \item کاهش خطاهای نرم‌افزاری: کد تمیزتر و منظم‌تر باعث می‌شود احتمال ایجاد خطا در هنگام تغییرات کاهش یابد. 
    \item افزایش انعطاف‌پذیری سیستم: سیستم‌های $\text{Refactored}$ راحت‌تر با قابلیت‌های جدید توسعه و با فناوری‌های مدرن یکپارچه می‌شوند.
    \item کاهش هزینه نگهداری در بلندمدت: هرچند $\text{Refactoring}$ هزینه و زمان اولیه دارد، اما باعث کاهش هزینه‌های نگهداری و اصلاح خطا در آینده می‌شود.
\end{itemize}

\textbf{اصول Refactoring و بازطراحی جزئی:}
\begin{itemize}
    \item \textbf{تغییر تدریجی:} بازسازی کد به صورت بخش‌بخش انجام شود تا ریسک خطا کاهش یابد.
    \item \textbf{تست مستمر:} قبل و بعد از هر تغییر، کد باید تست شود تا اطمینان حاصل شود که رفتار نرم‌افزار تغییر نکرده است.
    \item \textbf{مستندسازی تغییرات:} هر تغییر ساختاری باید مستند شود تا توسعه‌دهندگان آینده راحت‌تر آن را درک کنند.
    \item \textbf{استفاده از الگوهای طراحی و استانداردهای کدنویسی:} این کار باعث می‌شود $\text{Refactoring}$ موثرتر و پایدارتر باشد.
\end{itemize}

\subsection{ادغام مداوم (Continuous Integration - CI)}
ادغام مداوم ($\text{Continuous Integration}$) یک رویکرد در مهندسی نرم‌افزار است که در آن توسعه‌دهندگان به طور مکرر (معمولاً چند بار در روز) تغییرات خود را در مخزن اصلی کد منبع ($\text{Main Repository}$) ادغام می‌کنند. هر بار که تغییری اعمال می‌شود، سیستم به طور خودکار فرآیند ساخت ($\text{Build}$) و تست نرم‌افزار را اجرا می‌کند تا اطمینان حاصل شود که هیچ خطا یا ناسازگاری جدیدی به سیستم اضافه نشده است.

\textbf{اهمیت $\text{Continuous Integration}$:}
\begin{itemize}
    \item کشف سریع خطاها: با تست خودکار پس از هر ادغام، خطاها در همان مراحل اولیه توسعه شناسایی می‌شوند و از انباشته شدن مشکلات جلوگیری می‌شود.
    \item کاهش هزینه‌های نگهداری: رفع خطاهای کوچک در مراحل ابتدایی، هزینه و زمان نگهداری را در بلندمدت به طور قابل توجهی کاهش می‌دهد.
    \item افزایش کیفیت نرم‌افزار: تست‌های خودکار مداوم باعث می‌شود کد نهایی پایدارتر و با کیفیت‌تر باشد.
    \item سهولت در تکامل نرم‌افزار: با وجود یک سیستم ادغام مداوم، افزودن قابلیت‌های جدید یا تغییرات بزرگ در آینده با ریسک بسیار کمتری انجام می‌شود.
\end{itemize}

\textbf{اصول و ابزارهای ادغام مداوم:}
\begin{itemize}
    \item \textbf{مخزن مشترک کد منبع:} همه اعضای تیم تغییرات خود را در یک مخزن مشترک (مانند $\text{GitHub}$ یا $\text{GitLab}$) ذخیره می‌کنند.
    \item \textbf{تست خودکار پس از هر $\text{Commit}$:} هر بار که کدی به مخزن افزوده می‌شود، مجموعه‌ای از تست‌های خودکار اجرا می‌شود.
    \item \textbf{استفاده از سرور $\text{CI}$:} ابزارهایی مانند $\text{Jenkins}$، $\text{GitLab CI/CD}$، $\text{Travis CI}$ یا $\text{GitHub Actions}$ فرآیند ادغام و تست را به صورت خودکار مدیریت می‌کنند.
    \item \textbf{بازخورد سریع:} سیستم $\text{CI}$ در صورت بروز خطا، بلافاصله توسعه‌دهندگان را مطلع می‌کند تا اصلاحات سریع انجام شود.
\end{itemize}


[Image of the Continuous Integration CI process flow diagram]