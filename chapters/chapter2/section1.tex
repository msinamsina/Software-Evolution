\section{مقدمه}

چرخه‌ی توسعه و تکامل نرم‌افزار یک فرآیند پویا و چندبعدی است که از مرحله‌ی تحلیل نیازمندی‌ها تا نگهداری و به‌روزرسانی مداوم نرم‌افزار را شامل می‌شود. در این چرخه، تعامل میان عوامل انسانی، فنی و سازمانی نقش تعیین‌کننده‌ای در موفقیت یا شکست پروژه‌ها دارد. با وجود پیشرفت چشمگیر روش‌های مهندسی نرم‌افزار و ظهور مدل‌های چابک (Agile) و DevOps، همچنان مشکلات متعددی در مسیر توسعه و تکامل نرم‌افزار وجود دارد که باعث کاهش بهره‌وری، افزایش هزینه‌ها و افت کیفیت محصولات نرم‌افزاری می‌شود. این مشکلات معمولاً در سه دسته‌ی اصلی سازمانی، فنی و انسانی قرار می‌گیرند. در بخش‌های بعدی، مهم‌ترین چالش‌های هر دسته مورد بررسی قرار می‌گیرند تا درک روشن‌تری از دلایل شکست یا کندی پیشرفت در پروژه‌های نرم‌افزاری به دست آید.

چرخه‌ی توسعه و تکامل نرم‌افزار یک فرآیند چندمرحله‌ای و تعاملی است که از شناسایی نیازمندی‌ها آغاز شده و با تحلیل، طراحی، پیاده‌سازی، آزمون، استقرار، بهره‌برداری و در نهایت نگهداری و ارتقا ادامه می‌یابد. از آن‌جا که نرم‌افزار موجودیتی ایستا نیست و همواره در معرض تغییرات محیطی، فنی و رفتاری کاربران قرار دارد، این چرخه باید پویا، قابل یادگیری و انعطاف‌پذیر طراحی شود.

اما در عمل، بسیاری از پروژه‌های نرم‌افزاری از مسیر برنامه‌ریزی‌شده منحرف می‌شوند. دلایل این انحراف متنوع‌اند؛ نبود ارتباط بین تیم‌ها، ضعف ساختار مدیریتی و تأخیر در تصمیم‌گیری و یا مشکلات فنی ناشی از بدهی فنی، ناسازگاری ابزارها و کمبود دانش به‌روز توسعه‌دهندگان می‌توانند دلیل این انحراف باشند. با وجود تکامل رویکردهای مدرن مانند Agile، Scrum، DevOps و CI/CD، هنوز هم درصد قابل‌توجهی از پروژه‌ها با شکست یا تأخیر مواجه می‌شوند.

برخی از پروژه‌های نرم‌افزاری در جهان به‌واسطه‌ی ضعف در ارتباطات، مستندسازی ناقص یا تغییرات کنترل‌نشده‌ی نیازمندی‌ها آسیب می‌بینند. موفقیت واقعی زمانی به دست می‌آید که سه بعد انسانی، فنی و سازمانی به‌صورت هم‌زمان مورد مدیریت و بهبود قرار گیرند. در ادامه، هر یک از این دسته چالش‌ها با جزئیات گسترده‌تر بررسی می‌شود تا درک جامعی از موانع واقعی توسعه و تکامل نرم‌افزار حاصل گردد.
