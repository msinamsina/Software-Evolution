\section{جمع‌بندی فصل}
\label{sec:conclusion}

چرخه‌ی توسعه و تکامل نرم‌افزار، فرآیندی پویا و مداوم است که از طراحی و پیاده‌سازی اولیه آغاز شده و تا پایان عمر مفید سیستم ادامه دارد. با این حال، تجربه نشان داده است که بخش عمده‌ای از چالش‌ها و شکست‌های نرم‌افزاری نه در مرحله‌ی تولید، بلکه در مراحل نگهداری و تکامل رخ می‌دهد.

یکی از مهم‌ترین دلایل این مسئله، ماهیت پیچیده و در حال تغییر نرم‌افزارها است. با گذشت زمان، نیازمندی‌های کاربران دگرگون می‌شوند، فناوری‌ها تغییر می‌کنند و محیط‌های اجرایی جدید به‌وجود می‌آیند. در چنین شرایطی، نرم‌افزاری که در ابتدا به‌خوبی کار می‌کرد، ممکن است دیگر پاسخ‌گوی نیازهای کنونی نباشد و نیاز به اصلاح، بازطراحی یا بازنویسی پیدا کند.

\textbf{مشکلات رایج در این چرخه شامل موارد زیر است:}
\begin{itemize}
    \item کدهای پیچیده و غیرمستند که درک و تغییر آن‌ها دشوار است.
    \item نبود مدیریت تغییرات مؤثر که منجر به ناسازگاری میان اجزای سیستم می‌شود.
    \item افزایش هزینه‌های نگهداری در نتیجه‌ی ساختار نامناسب، وابستگی زیاد و فناوری‌های منسوخ.
    \item مشکلات فنی ناشی از Legacy Systems که مانع از یکپارچه‌سازی با فناوری‌های جدید می‌شوند.
    \item کمبود تست‌های مداوم و خودکار ($\text{Continuous Testing}$) که باعث می‌شود خطاها در مراحل بعدی آشکار شوند و هزینه‌ی رفع آن‌ها بیشتر شود.
    \item ضعف ارتباط میان تیم‌های توسعه، پشتیبانی و کاربران نهایی که درک نیازهای واقعی را دشوار می‌کند.
\end{itemize}

در مجموع، می‌توان گفت که نگهداری و تکامل نرم‌افزار نه یک فعالیت فرعی، بلکه بخش اصلی از چرخه‌ی حیات نرم‌افزار است. سازمان‌هایی که از ابتدا به طراحی منعطف، مستندسازی دقیق، مدیریت تغییرات، تست مستمر و نوسازی سیستم‌های قدیمی توجه کنند، قادر خواهند بود هزینه‌ها را کاهش داده و طول عمر سیستم‌های نرم‌افزاری خود را افزایش دهند. در مقابل، بی‌توجهی به این جنبه‌ها می‌تواند منجر به افزایش هزینه، کاهش پایداری، و در نهایت شکست کامل پروژه شود.