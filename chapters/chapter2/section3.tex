\section{مشکلات فنی}

چالش‌های فنی از مهم‌ترین عوامل تأخیر در توسعه و افزایش هزینه‌های نگهداری نرم‌افزار به شمار می‌روند. این مشکلات معمولاً ریشه در ساختار پیچیده‌ی کد، فناوری‌های قدیمی، یا ضعف در طراحی اولیه دارند. هرچه پروژه بزرگ‌تر و طولانی‌تر باشد، احتمال بروز چنین مشکلاتی بیشتر می‌شود.

\subsection{بدهی فنی (\lr{Technical Debt})}

بدهی فنی به تصمیمات کوتاه‌مدتی اشاره دارد که برای سرعت‌بخشیدن به تحویل محصول گرفته می‌شوند، اما در آینده هزینه‌های سنگینی به پروژه تحمیل می‌کنند. این بدهی می‌تواند شامل کدهای تکراری، طراحی ناقص، تست‌های ناکافی، یا معماری غیراستاندارد باشد. هر بار که تیمی به‌جای اصلاح ریشه‌ای مشکلات، راه‌حل موقتی انتخاب می‌کند، بدهی فنی افزایش می‌یابد.

\subsection{ناسازگاری با فناوری‌های جدید}

سیستم‌هایی که بر پایه فناوری‌های منسوخ بنا شده‌اند، به‌تدریج دچار محدودیت عملکردی و امنیتی می‌شوند. مهاجرت به فناوری‌های جدید اغلب دشوار، هزینه‌بر و زمان‌گیر است. راهکار مؤثر، استراتژی «مدرن‌سازی تدریجی» است؛ یعنی انتقال گام‌به‌گام بخش‌های حیاتی سیستم به فناوری‌های جدید بدون توقف کامل سیستم.

\subsection{خطاهای طراحی و ماژول‌های ناسازگار}

طراحی ناپایدار و غیرماژولار باعث بروز خطاهای میان‌ماژولی، افت کارایی، اختلال در عملکرد و دشواری شدید در افزودن ویژگی‌های جدید می‌شود. استفاده از معماری مدرن مانند microservices و طراحی مبتنی بر API می‌تواند از بروز تضادها جلوگیری کند.
