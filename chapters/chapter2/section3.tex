\section{مشکلات فنی}

چالش‌های فنی از مهم‌ترین عوامل تأخیر در توسعه و افزایش هزینه‌های نگهداری نرم‌افزار به شمار می‌روند. این مشکلات معمولاً ریشه در ساختار پیچیده‌ی کد، فناوری‌های قدیمی، یا ضعف در طراحی اولیه دارند. هرچه پروژه بزرگ‌تر و طولانی‌تر باشد، احتمال بروز چنین مشکلاتی بیشتر می‌شود.

\subsection{بدهی فنی (Technical Debt)}
بدهی فنی اصطلاحی است که به تصمیمات کوتاه‌مدتی اشاره دارد که در زمان توسعه برای سرعت‌بخشیدن به تحویل محصول گرفته می‌شوند، اما در آینده هزینه‌های سنگینی به پروژه تحمیل می‌کنند. این بدهی می‌تواند شامل کدهای تکراری، طراحی ناقص، تست‌های ناکافی، یا معماری غیراستاندارد باشد. هر بار که تیمی به‌جای اصلاح ریشه‌ای مشکلات، راه‌حل موقتی انتخاب می‌کند، بدهی فنی افزایش می‌یابد. انباشت بدهی فنی به‌مرور باعث کاهش سرعت تیم، افزایش احتمال خطا و کاهش انعطاف‌پذیری سیستم می‌شود. مانند بدهی مالی، هرچه بازپرداخت آن به تأخیر بیفتد، هزینه‌ی اصلاح بیشتر می‌گردد.

برای مثال، زمانی که تیم توسعه برای تحویل سریع‌تر، تست‌های خودکار را حذف یا طراحی را ساده‌سازی می‌کند، در واقع «بدهی فنی» انباشته می‌کند. بازپرداخت این بدهی در آینده ممکن است شامل بازنویسی بخش‌های بزرگی از کد یا اصلاح ساختار معماری سیستم باشد. هرچه این بدهی بیشتر شود، سرعت توسعه در آینده کمتر و هزینه‌ی نگهداری بیشتر خواهد شد.

برای مدیریت این مشکل، تیم توسعه باید برنامه‌ی منظمی برای بازبینی کد، حذف بخش‌های ناکارآمد، و بازطراحی اجزای کلیدی داشته باشد.

\subsection{ناسازگاری با فناوری‌های جدید}
عمر مفید فناوری‌های نرم‌افزاری کوتاه است. زبان‌ها، فریم‌ورک‌ها و کتابخانه‌ها با سرعتی زیاد به‌روزرسانی می‌شوند و سیستم‌هایی که بر پایه فناوری‌های منسوخ بنا شده‌اند، به‌تدریج دچار محدودیت عملکردی و امنیتی می‌شوند.

برای مثال، برنامه‌هایی که روی پلتفرم‌های قدیمی COBOL یا Delphi ایجاد شده‌اند، با سیستم‌های ابری یا معماری‌های مدرن سازگار نیستند. مهاجرت از این بسترها اغلب دشوار، هزینه‌بر و زمان‌گیر است. در مواردی که زیرساخت‌ها به‌روز نمی‌شوند، سازمان در معرض خطرات امنیتی، نبود پشتیبانی و محدودیت در توسعه قابلیت‌های جدید قرار می‌گیرد.

راهکار مؤثر برای مقابله با این چالش، استراتژی «مدرن‌سازی تدریجی» است؛ یعنی انتقال گام‌به‌گام بخش‌های حیاتی سیستم به فناوری‌های جدید، بدون اینکه عملکرد سیستم موجود به طور کامل متوقف شود. ارزیابی دوره‌ای زیرساخت‌ها و به‌روزرسانی مستمر کتابخانه‌ها نیز از الزامات حیاتی طراحی سیستم‌های پایدار محسوب می‌شود.

\subsection{خطاهای طراحی و ماژول‌های ناسازگار}
طراحی ناپایدار و غیرماژولار یکی از ریشه‌های مهم مشکلات فنی است. در پروژه‌هایی که معماری سیستم به‌درستی تعریف نشده یا در طول زمان با تغییرات کنترل‌نشده مواجه شده است، تعامل میان اجزا به مشکل برمی‌خورد.

این ناسازگاری‌ها باعث بروز خطاهای میان‌ماژولی، افت کارایی، اختلال در عملکرد و دشواری شدید در افزودن ویژگی‌های جدید می‌شوند. در پروژه‌های توزیع‌شده و بین‌المللی، نبود طراحی یکپارچه ممکن است هر تیم را به سمت پیاده‌سازی متفاوتی از یک ماژول سوق دهد.

به‌کارگیری اصول معماری مدرن مانند microservices، طراحی مبتنی بر API و استفاده از استانداردهای تعامل (مانند OpenAPI) می‌تواند از بروز چنین تضادهایی جلوگیری کند. علاوه بر این، بازبینی‌های فنی و جلسات منظم برای بررسی طراحی و معماری، تضمین می‌کنند که تمامی تغییرات با اهداف کلان معماری هم‌سو باقی بمانند.
