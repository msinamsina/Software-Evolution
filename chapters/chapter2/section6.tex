
\section{مطالعه‌ی موردی از شکست پروژه‌ها}
\subsection{شکست سیستم LASCAD (London Ambulance Service Computer Aided Dispatch)}
LASCAD پروژه‌ای بود که در اوایل دهه ۱۹۹۰ توسط خدمات آمبولانس لندن (LAS) اجرا شد. هدف از این پروژه، خودکارسازی فرآیند دریافت تماس‌های اضطراری، تخصیص آمبولانس و پیگیری عملیات امداد بود تا پاسخ‌دهی سریع‌تر و کارآمدتری ارائه شود؛ اما این پروژه به یکی از بارزترین شکست‌های تاریخ فناوری اطلاعات بریتانیا تبدیل شد. تنها چند ساعت پس از راه‌اندازی سیستم در اکتبر ۱۹۹۲، مشکلات فنی گسترده‌ای بروز کرد و سیستم عملاً از کار افتاد. در نتیجه، ده‌ها تماس اضطراری بدون پاسخ ماند و گزارش شد که چندین نفر جان خود را از دست دادند.

\textbf{دلایل شکست:}تحقیقات بعدی چند عامل کلیدی را در شکست این پروژه شناسایی کردند:

\begin{itemize}
\item طراحی ضعیف و غیرقابل نگهداری: ساختار نرم‌افزار به‌گونه‌ای بود که تغییر در یک بخش باعث بروز خطا در سایر بخش‌ها می‌شد. قابلیت تکامل (Evolvability) در طراحی لحاظ نشده بود.

\item تست ناکافی: سیستم بدون انجام آزمون‌های واقعی و تست بار (Load Testing) در محیط عملیاتی راه‌اندازی شد و نتوانست حجم زیاد داده‌ها و تماس‌ها را تحمل کند.
\item نبود مدیریت تغییرات: تغییرات پی‌درپی در نیازمندی‌ها و طراحی، بدون کنترل و مستندسازی مناسب انجام می‌شد که موجب ناسازگاری داخلی سیستم گردید.
\item مستندسازی و آموزش ضعیف: کاربران سیستم آموزش کافی ندیده بودند و مستندات فنی ناقص بود، در نتیجه امکان نگهداری مؤثر وجود نداشت.
\item فشار زمانی و مدیریتی: مدیران پروژه به‌دلیل فشارهای سیاسی و رسانه‌ای، سیستم را بدون آمادگی کامل به بهره‌برداری رساندند.

\end{itemize}

\textbf{نتایج و پیامدها:} سیستم تنها چند ساعت پس از راه‌اندازی به‌طور کامل از کار افتاد. خدمات اورژانس لندن برای چند روز به حالت دستی بازگشت. اعتبار سازمان خدمات آمبولانس لندن به‌شدت آسیب دید و اعتماد عمومی کاهش یافت. در نهایت پروژه لغو شد و برای طراحی مجدد آن میلیون‌ها پوند هزینه شد..
\textbf{مهم‌ترین درس‌های حاصل از این پروژه}نگهداری و تکامل‌پذیری نرم‌افزار باید از مراحل اولیه‌ی طراحی در نظر گرفته شود. وجود مدیریت تغییرات، مستندسازی دقیق و تست مستمر برای تضمین پایداری سیستم حیاتی است. فشارهای زمانی و تصمیم‌گیری‌های مدیریتی بدون توجه به آمادگی فنی می‌تواند پروژه را با شکست کامل روبه‌رو کند. نرم‌افزارهای حیاتی (مانند سیستم‌های اورژانس) باید پیش از بهره‌برداری، در محیط واقعی و تحت بار عملیاتی واقعی تست شوند.

\subsection{شکست پروژه‌ی Virtual Case File در سازمان FBI}در سال ۲۰۰۰، سازمان FBI تصمیم گرفت تا سیستم‌های قدیمی خود را که بر پایه‌ی فناوری‌های دهه‌ی ۱۹۸۰ ساخته شده بودند، با یک سیستم مدرن دیجیتال جایگزین کند. این پروژه با نام Virtual Case File آغاز شد و هدف آن ایجاد یک سامانه‌ی یکپارچه برای مدیریت پرونده‌های جنایی، اسناد، مدارک و جریان کاری مأموران بود.
سیستم جدید قرار بود فرآیندهای دستی و پراکنده‌ی موجود را خودکار کند و از نظر امنیت، دقت و سرعت، عملکرد FBI را ارتقا دهد.


\textbf{دلایل شکست:}
\begin{itemize}
\item زیرساخت قدیمی و ناسازگار: سیستم‌های قبلی FBI بسیار قدیمی بودند و هیچ مستندات دقیقی از ساختار آن‌ها وجود نداشت. این موضوع باعث شد تبدیل و مهاجرت داده‌ها (Data Migration) به سیستم جدید با خطا و پیچیدگی بالا همراه شود.

\item مدیریت تغییرات ضعیف: در طول پروژه، نیازمندی‌های نرم‌افزار بارها تغییر کردند، اما هیچ سازوکار رسمی برای کنترل و پیگیری این تغییرات وجود نداشت. نتیجه این شد که بخش‌های مختلف نرم‌افزار با هم ناسازگار شدند.

\item طراحی غیرقابل نگهداری: تیم توسعه بدون داشتن چشم‌انداز تکامل بلندمدت، سیستم را به‌شکل متمرکز و سخت‌افزاری وابسته طراحی کرد. هر گونه تغییر یا افزودن قابلیت جدید، نیاز به بازنویسی بخش‌های زیادی از کد داشت.

\item فقدان ارتباط مؤثر میان ذینفعان: ارتباط میان مدیران، تحلیل‌گران، توسعه‌دهندگان و مأموران FBI ضعیف بود. نیازهای واقعی کاربران نهایی به‌درستی منتقل یا درک نمی‌شد.
\item فشار زمانی و مدیریتی: پس از حملات ۱۱ سپتامبر، فشار زیادی برای تسریع در تحویل سیستم وجود داشت. این تصمیم باعث شد توسعه‌ی نرم‌افزار بدون تست و بررسی‌های کیفی لازم ادامه یابد.
\end{itemize}

\textbf{نتایج و پیامدها:} پروژه پس از صرف حدود ۱۷۰ میلیون دلار هزینه و چهار سال زمان، به‌طور کامل کنار گذاشته شد.
هیچ‌یک از قابلیت‌های کلیدی مورد انتظار (جست‌وجوی هوشمند، اشتراک‌گذاری پرونده‌ها، تحلیل خودکار داده‌ها) به مرحله‌ی استفاده نرسید. FBI مجبور شد پروژه‌ی جدیدی به نام Sentinel را از ابتدا و با درس‌گرفتن از شکست VCF آغاز کند.

\textbf{مهم‌ترین درس‌های حاصل از این پروژه:} سیستم‌های قدیمی بدون مستندات مناسب، ریسک بالایی برای تکامل دارند و پیش از مهاجرت باید تحلیل عمیق روی آن‌ها انجام شود. مدیریت تغییرات و نیازمندی‌ها باید از روز اول پروژه برقرار باشد تا از بروز ناسازگاری جلوگیری شود. طراحی سیستم باید قابلیت نگهداری (Maintainability) و تکامل‌پذیری (Evolvability) را در خود داشته باشد. فشار برای تحویل سریع در پروژه‌های حیاتی، معمولاً منجر به کاهش کیفیت و شکست می‌شود.
