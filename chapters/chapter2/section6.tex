
\section*{مطالعه‌ی موردی از شکست پروژه‌ها}
\subsection*{سیستم LASCAD (London Ambulance Service Computer Aided Dispatch)}
LASCAD پروژه‌ای بود که در اوایل دهه ۱۹۹۰ توسط خدمات آمبولانس لندن اجرا شد. هدف خودکارسازی فرآیند دریافت تماس‌های اضطراری، تخصیص آمبولانس و پیگیری عملیات امداد بود؛ اما سیستم عملاً از کار افتاد.

\textbf{دلایل شکست:}
\begin{itemize}
\item طراحی ضعیف و غیرقابل نگهداری.
\item تست ناکافی.
\item نبود مدیریت تغییرات.
\item مستندسازی و آموزش ضعیف.
\item فشار زمانی و مدیریتی.
\end{itemize}

\textbf{نتایج و پیامدها:} سیستم چند ساعت پس از راه‌اندازی به‌طور کامل از کار افتاد و اعتبار سازمان کاهش یافت.

\subsection*{پروژه‌ی Virtual Case File در FBI}
در سال ۲۰۰۰، سازمان FBI تصمیم گرفت سیستم‌های قدیمی خود را با یک سیستم مدرن جایگزین کند.

\textbf{دلایل شکست:}
\begin{itemize}
\item زیرساخت قدیمی و ناسازگار.
\item مدیریت تغییرات ضعیف.
\item طراحی غیرقابل نگهداری.
\item فقدان ارتباط مؤثر میان ذینفعان.
\item فشار زمانی و مدیریتی.
\end{itemize}

\textbf{نتایج و پیامدها:} پروژه پس از صرف حدود ۱۷۰ میلیون دلار و چهار سال، کنار گذاشته شد و FBI پروژه جدیدی آغاز کرد.

\textbf{درس‌ها:} سیستم‌های قدیمی بدون مستندات مناسب، ریسک بالایی دارند. مدیریت تغییرات و نیازمندی‌ها باید از روز اول برقرار باشد.
