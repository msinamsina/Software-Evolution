فلسفه DevOps بر پایه اصولی استوار است که فرهنگ همکاری، خودکارسازی و بهبود مستمر را ترویج می‌دهد. این فلسفه را می‌توان در "حلقه بی‌پایان" عملیات DevOps (که شامل مراحل برنامه‌ریزی، توسعه، استقرار و نظارت است) و همچنین در "سه راهی" معروف آن (جریان (Flow)، بازخورد (Feedback) و یادگیری مستمر (\lr{Continues Learning}) خلاصه کرد.
ارتباط DevOps با متدولوژی Agile بسیار عمیق است. Agile بر انعطاف‌پذیری، تحویل تدریجی و پاسخگویی به تغییرات در طول فرآیند توسعه تأکید دارد. DevOps این فلسفه را گسترش می‌دهد و آن را به فرآیند استقرار و عملیات پس از توسعه تسری می‌بخشد. در حقیقت، DevOps مکمل Agile است؛ در حالی که Agile سرعت و کیفیت توسعه را افزایش می‌دهد، DevOps تضمین می‌کند که این تغییرات سریع می‌توانند به صورت ایمن و پایدار در محیط تولید مستقر شوند. بنابراین، می‌توان DevOps را به عنوان ادامه طبیعی و ضروری جنبش Agile در نظر گرفت که تمرکز آن بر روی کل چرخه عمر نرم‌افزار است.

