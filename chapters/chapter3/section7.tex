% chapters/chapter3/section7.tex
% این فایل را فایل اصلی فصل با \input وارد می‌کند، پس نباید \section یا \subsection داشته باشد.

شرکت \lr{Netflix} با میلیون‌ها کاربر در سراسر جهان، یکی از پیشگامان در به‌کارگیری رویکرد \lr{DevOps} است. مقیاس بسیار بزرگ سامانه و نیاز به ارائهٔ مداوم محتوا، این شرکت را بر آن داشت تا از شیوه‌های سنتی توسعه فاصله بگیرد و معماری‌ای پویا و مبتنی بر خودکارسازی ایجاد کند. همان‌گونه که در پژوهش \cite{Jha2023} نیز تأکید شده، موفقیت در مقیاس گسترده تنها زمانی ممکن است که فرهنگ سازمانی، ابزارها و فرآیندها هم‌زمان دگرگون شوند.

\subsubsection*{چالش‌های اولیه}

در سال‌های ابتدایی فعالیت، \lr{Netflix} با چند چالش اساسی روبه‌رو بود:
\begin{itemize}
    \item استقرارهای نرم‌افزاری به‌صورت دستی انجام می‌شد و احتمال خطاهای انسانی بالا بود.
    \item هرگونه تغییر کوچک در سیستم می‌توانست موجب اختلال در پخش محتوا شود.
    \item سرورها در مراکز دادهٔ داخلی نگهداری می‌شدند و مقیاس‌پذیری آن‌ها محدود بود.
\end{itemize}

این چالش‌ها سبب شدند که \lr{Netflix} در سال ۲۰۰۸ تصمیم بگیرد به زیرساخت ابری مهاجرت کند و هم‌زمان فلسفهٔ \lr{DevOps} را در سازمان پیاده‌سازی کند. این تصمیم، نقطهٔ عطفی در مسیر تکامل فنی و فرهنگی شرکت بود.

\subsubsection*{معماری و ابزارهای مورد استفاده}

برای تحقق اصول \lr{DevOps}، \lr{Netflix} مجموعه‌ای از ابزارها و فرآیندهای خودکار را توسعه داد. برخی از مهم‌ترین آن‌ها عبارت‌اند از:
\begin{itemize}
    \item \textbf{\lr{Spinnaker}:} سیستم متن‌باز ویژهٔ \lr{Netflix} برای خودکارسازی خط لوله‌های \lr{CI/CD}. این ابزار امکان استقرار مکرر، سریع و بدون وقفهٔ سرویس‌ها را فراهم می‌کند.
    \item \textbf{\lr{Chaos Monkey}:} ابزاری برای آزمایش پایداری سیستم از طریق ایجاد خطاهای تصادفی در سرورها؛ هدف آن ارزیابی مقاومت سامانه در برابر شکست است.
    \item \textbf{\lr{Atlas}} و \textbf{\lr{Vector}:} ابزارهای پایش و تحلیل عملکرد سرویس‌ها که داده‌ها را به‌صورت لحظه‌ای جمع‌آوری و بررسی می‌کنند.
\end{itemize}

با این زیرساخت‌ها، \lr{Netflix} قادر است روزانه صدها استقرار جدید انجام دهد، بدون آن‌که کاربران هیچ‌گونه اختلالی در سرویس احساس کنند.

\subsubsection*{فرهنگ سازمانی \lr{DevOps} در \lr{Netflix}}

مطابق با دیدگاه مطرح‌شده در \cite{Jha2023}، یکی از عوامل کلیدی موفقیت \lr{DevOps} در \lr{Netflix}، نهادینه‌سازی آن در فرهنگ سازمانی است. اصول فرهنگی مهم در این شرکت شامل موارد زیر است:
\begin{itemize}
    \item \textbf{اعتماد به تیم‌ها:} هر تیم مسئول استقرار و نگهداری سرویس‌های خود است.
    \item \textbf{آزادی همراه با مسئولیت:} توسعه‌دهندگان در انتخاب ابزار و روش‌ها آزادی کامل دارند، اما مسئولیت عملکرد سرویس نیز با خود آنان است.
    \item \textbf{بازخورد سریع:} داده‌های واقعی کاربران به‌صورت لحظه‌ای تحلیل می‌شود و تصمیم‌گیری‌ها بر پایهٔ شواهد انجام می‌گیرد.
\end{itemize}

\subsubsection*{نتایج پیاده‌سازی}

اجرای اصول \lr{DevOps} در \lr{Netflix} منجر به بهبود چشمگیر در جنبه‌های مختلف توسعه و بهره‌برداری از سامانه شده است:
\begin{itemize}
    \item کاهش محسوس خطاهای استقرار،
    \item افزایش سرعت ارائهٔ قابلیت‌های جدید،
    \item مقیاس‌پذیری بسیار بالا در پاسخ به رشد کاربران،
    \item ارتقای تجربهٔ کاربری و کاهش زمان قطعی سرویس.
\end{itemize}

به‌عنوان نمونه، در زمان اوج مصرف، سامانه‌های \lr{Netflix} قادرند میلیون‌ها درخواست هم‌زمان را بدون افت کیفیت پاسخ دهند؛ قابلیتی که بدون زیرساخت خودکار و فرهنگ همکاری \lr{DevOps} امکان‌پذیر نبود.
