\section{مزایای DevOps در تکامل نرم‌افزار}

مهم‌ترین مزایای به‌کارگیری \lr{DevOps} در فرایند توسعه و تکامل نرم‌افزار عبارت‌اند از:
\begin{itemize}
    \item افزایش سرعت تحویل نرم‌افزار
    \item بهبود پایداری و اطمینان در استقرارها
    \item ارتقای کیفیت محصول
    \item افزایش بهره‌وری و هماهنگی تیم‌ها
    \item توانایی پاسخ سریع به تغییرات بازار و نیازهای کاربران
\end{itemize}
همان‌طور که در \cite{Jha2023} نیز اشاره شده است، این مزایا زمانی به‌طور کامل به دست می‌آیند که فرهنگ همکاری میان تیم‌های توسعه و عملیات که در بخش \ref{subsec:ch3-5-collaboration} توضیح داده شد، در سازمان نهادینه شده باشد.

\subsubsection*{افزایش سرعت تحویل نرم‌افزار}
\lr{DevOps} موجب می‌شود چرخه‌ی توسعه از ایده تا تحویل نهایی کوتاه‌تر شود. با خودکارسازی مراحلی مانند ساخت، تست و استقرار، تیم‌ها می‌توانند در بازه‌های زمانی بسیار کوتاه نسخه‌های جدید ارائه دهند. به‌عنوان نمونه، شرکت \lr{Amazon} روزانه هزاران استقرار جدید در زیرساخت خود انجام می‌دهد. این حجم از به‌روزرسانی تنها به لطف استفاده از خطوط خودکار \lr{CI/CD} ممکن است.

\subsubsection*{بهبود پایداری و اطمینان در استقرارها}
در روش‌های سنتی، استقرار نرم‌افزار اغلب با اضطراب و خطا همراه بود، زیرا تغییرات به‌صورت گسترده و یک‌باره اعمال می‌شد. \lr{DevOps} این مشکل را با اعمال تغییرات کوچک و مکرر حل کرده است. نمونه‌ی شناخته‌شده، تجربهٔ \lr{Etsy} است که پس از خودکارسازی استقرارها، توانست بدون توقف سرویس، استقرارهای متعدد روزانه انجام دهد.

\subsubsection*{ارتقای کیفیت محصول}
تست‌های خودکار و مانیتورینگ مستمر از ارکان \lr{DevOps} هستند و کمک می‌کنند خطاها در مراحل ابتدایی شناسایی و اصلاح شوند. شرکت‌هایی مانند \lr{Google} با تکیه بر پایش مداوم، نرخ خرابی را کاهش داده‌اند.

\subsubsection*{افزایش بهره‌وری و هماهنگی تیم‌ها}
\lr{DevOps} باعث می‌شود تیم‌های توسعه، عملیات، آزمون و حتی امنیت در یک چرخه‌ی واحد کار کنند و کارهای دستی و تکراری حذف شود.

\subsubsection*{پاسخ سریع به تغییرات بازار و نیازهای کاربران}
در محیط‌های پویا، چرخه‌ی بازخورد سریع که در \cite{Jha2023} بر آن تأکید شده، امکان انتشار و بازگردانی سریع ویژگی‌ها را فراهم می‌کند.
