\section{چرخه عمر DevOps}
چرخه عمر DevOps یک فرآیند تکراری و مستمر است که مراحل مختلفی از ایده تا تحویل نرم‌افزار و نظارت بر آن را در بر می‌گیرد. این چرخه با استفاده از ابزارهای خودکار به هم پیوسته، جریان ارزش را سریع و کارآمد می‌کند.

\begin{itemize}
    \item \textbf{برنامه‌ریزی (Plan)} \\
    در این فاز اولیه، اهداف پروژه تعریف، وظایف زمان‌بندی و پیشرفت کار رهگیری می‌شود. این مرحله تضمین می‌کند که همه اعضای تیم از اهداف کسب‌وکار و برنامه‌های فنی آگاه هستند. \\
    \textbf{ابزارها:} از ابزارهایی مانند Jira برای ردیابی Issues و مدیریت پروژه و Confluence برای مستندسازی و همکاری استفاده می‌شود.

    \item \textbf{توسعه (Code)} \\
    توسعه‌دهندگان در این مرحله نرم‌افزار را می‌نویسند. برای اطمینان از سازگاری و قابلیت تکرار محیط‌های توسعه، از ابزارهای خاصی استفاده می‌شود. \\
    \textbf{ابزارها:} Docker برای بسته‌بندی نرم‌افزار در کانتینرهای سبک و قابل حمل، Kubernetes برای مدیریت و خودکارسازی این کانتینرها، و Ansible \& Puppet برای مدیریت پیکربندی و خودکارسازی زیرساخت به کار می‌روند.

    \item \textbf{یکپارچه‌سازی مستمر (\lr{Continuous Integration})} \\
    این تمرین شامل ادغام مکرر کد نوشته‌شده توسط تمام توسعه‌دهندگان به یک ریپازیتوری مشترک است. پس از هر ادغام، فرآیندهای ساخت و تست به طور خودکار اجرا می‌شوند تا خطاها در اسرع وقت شناسایی شوند. CI تضمین می‌کند که کدها به طور مداوم با یکدیگر یکپارچه شده و از بروز تعارضات بزرگ در آینده جلوگیری می‌کند.

    \item \textbf{تحویل مستمر (\lr{Continuous Delivery})} \\
    CD گام بعدی پس از CI است. این تمرین تضمین می‌کند که پس از هر ادغام موفقیت‌آمیز کد، می‌توان نرم‌افزار را در هر لحظه و با کمترین تلاش به صورت دستی در محیط تولید منتشر کرد. در تحویل مستمر، فرآیند استقرار تا مرحله نهایی خودکار است، اما انتشار نهایی در محیط تولید به صورت دستی و با تأیید یک انسان انجام می‌شود.

    \item \textbf{استقرار مستمر (\lr{Continuous Deployment})} \\
    این پیشرفته‌ترین مرحله است که در آن، هر تغییری که از تست‌ها در مراحل CI/CD موفقیت‌آمیز عبور کند، به طور خودکار در محیط تولید مستقر می‌شود. در این مدل، هیچ مداخله دستی در فرآیند استقرار وجود ندارد و انتشار نرم‌افزار به یک رویداد عادی و روزمره تبدیل می‌شود. این امر سرعت ارائه ارزش به کاربر نهایی را به حداکثر می‌رساند. \\
    \textbf{ابزارهای CI/CD:} از ابزارهایی مانند Jenkins, GitHub Actions/GitLab CI/CD و CircleCI برای خودکارسازی کامل خط لوله از یکپارچه‌سازی تا استقرار استفاده می‌شود.

    \item \textbf{نظارت و بازخورد (\lr{Monitoring \& Feedback})} \\
    پس از استقرار نرم‌افزار در محیط تولید، عملکرد آن تحت نظارت دقیق قرار می‌گیرد تا از پایداری و سلامت سرویس اطمینان حاصل شود. داده‌های مربوط به عملکرد برنامه، زیرساخت و تجربه کاربر جمع‌آوری و تجزیه و تحلیل می‌شوند. این داده‌ها به صورت یک حلقه بازخورد (Feedback Loop) به تیم‌های توسعه و برنامه‌ریزی بازمی‌گردند تا برای بهبود مستمر محصول و رفع مشکلات در چرخه‌های بعدی مورد استفاده قرار گیرند. \\
    \textbf{ابزارها:} Grafana برای تجسم متریک‌ها، Elastic Stack برای مدیریت و تحلیل لاگ‌ها، Prometheus برای مانیتورینگ و هشدار و Sentry برای ردیابی خطاها در لحظه استفاده می‌شوند.
\end{itemize}
