% chapters/chapter3/section8.tex
% این فایل را فایل اصلی فصل با \input وارد می‌کند، پس نباید \section یا \subsection داشته باشد.

هرچند \lr{DevOps} در سال‌های اخیر به‌عنوان یکی از مؤثرترین رویکردها در توسعهٔ نرم‌افزار شناخته شده است، اما پیاده‌سازی موفق آن کار ساده‌ای نیست. همان‌گونه که در پژوهش \cite{Jha2023} نیز اشاره شده، سازمان‌ها در مسیر استقرار \lr{DevOps} با موانع فنی و فرهنگی متعددی روبه‌رو می‌شوند که در صورت مدیریت‌نشدن صحیح، می‌توانند موجب کندی یا حتی شکست کل فرآیند شوند. در ادامه، مهم‌ترین چالش‌های پیاده‌سازی این رویکرد بررسی می‌شود.

\subsubsection*{مسائل امنیتی و حفظ اعتماد}

با خودکار شدن فرآیندها و افزایش سرعت استقرار، امنیت به یکی از دغدغه‌های اصلی در محیط‌های \lr{DevOps} تبدیل شده است. در روش‌های سنتی، بررسی‌های امنیتی معمولاً در انتهای چرخهٔ توسعه انجام می‌شد، اما در \lr{DevOps} انتشارهای سریع و مکرر ممکن است سبب نادیده‌گرفتن برخی کنترل‌های حیاتی شود. برای نمونه، زمانی که تیم توسعه به‌صورت روزانه کد جدید را با شاخهٔ اصلی ادغام می‌کند، یک آسیب‌پذیری کوچک می‌تواند بلافاصله وارد محیط تولید شود.
برای رفع این مشکل، رویکرد \lr{DevSecOps} پیشنهاد می‌شود که در آن، امنیت از مراحل اولیهٔ توسعه در چرخهٔ عمر نرم‌افزار ادغام می‌شود. همچنین کنترل دسترسی، مدیریت کلیدها و محافظت از داده‌های حساس از مسئولیت‌های مهمی هستند که نیاز به نظارت مداوم دارند.

\subsubsection*{پیچیدگی زیرساخت و وابستگی به ابزارها}

یکی دیگر از چالش‌های جدی، افزایش پیچیدگی فنی در اثر استفاده از ابزارهای متنوع است. سازمان‌ها برای پیاده‌سازی \lr{DevOps} اغلب از ترکیب ابزارهایی چون \lr{Docker}، \lr{Kubernetes}، \lr{Jenkins} و \lr{Terraform} استفاده می‌کنند. هرچند این ابزارها قدرت و انعطاف بالایی دارند، اما برای تیم‌هایی که تجربهٔ کافی ندارند، می‌توانند موجب سردرگمی و کاهش بهره‌وری شوند.
مطالعهٔ \cite{Jha2023} نشان می‌دهد تمرکز بیش از حد بر ابزارها ممکن است هدف اصلی \lr{DevOps} یعنی همکاری مؤثر و تحویل سریع ارزش به مشتری را تحت‌الشعاع قرار دهد. مستندسازی دقیق، آموزش منظم و طراحی زیرساخت ساده و پایدار از مهم‌ترین راهکارهای مقابله با این چالش هستند.

\subsubsection*{مقاومت فرهنگی و تغییر در شیوهٔ کار}

مهم‌ترین مانع در مسیر اجرای \lr{DevOps}، چالش فرهنگی درون سازمان است. برخلاف تصور رایج، \lr{DevOps} صرفاً تغییر در ابزارها نیست، بلکه تحولی در نگرش، ساختار و مسئولیت‌پذیری اعضاست. در مدل سنتی، تیم‌های توسعه و عملیات معمولاً به‌صورت مجزا عمل می‌کردند و هرکدام تنها بخشی از مسئولیت را بر عهده داشتند؛ اما در \lr{DevOps} مرزها از میان برداشته می‌شوند و موفقیت کل محصول، مسئولیتی جمعی است.
در بسیاری از سازمان‌ها، این تغییر ذهنیت با مقاومت مواجه می‌شود — به‌ویژه در ساختارهای سلسله‌مراتبی که عادت به تفکیک نقش‌ها دارند. تجربهٔ گزارش‌شده در \cite{Jha2023} نشان می‌دهد آموزش مستمر، شفاف‌سازی اهداف و مشارکت فعال کارکنان در تصمیم‌گیری، از مؤثرترین راهکارها برای غلبه بر این مقاومت فرهنگی است.

در مجموع، استقرار موفق \lr{DevOps} مستلزم آمادگی فنی و فرهنگی توأمان است. بی‌توجهی به یکی از این ابعاد می‌تواند موجب کندی در تحول سازمانی و کاهش اثربخشی کل چرخهٔ توسعه شود.
