% chapter6/section4.tex
\section{مسیرهای تحقیقاتی و آموزشی آینده}
\label{sec:ch6-future-work}

با توجه به چالش‌ها و روندهای بررسی‌شده، به‌ویژه در فصول ۳ و ۴، مسیرهای زیر برای تحقیقات و آموزش‌های آتی در حوزه مهندسی نرم‌افزار پیشنهاد می‌شود:

\begin{enumerate}
  \item \textbf{امنیت در چرخه‌های خودکار(\lr{DevSecOps}):} همان‌طور که در چالش‌های استقرار DevOps اشاره شد، افزایش سرعت استقرار می تواند نگرانی‌های امنیتی ایجاد کند. تحقیقات و آموزش‌های آینده باید بر ادغام یکپارچه امنیت در تمام مراحل چرخه عمر نرم افزار (رویکرد \lr{DevSecOps}) و روش‌های مدیریت خودکار دسترسی‌ها و داده‌های حساس متمرکز شوند.
  \item \textbf{کاربرد هوش مصنوعی در بازمهندسی نرم‌افزار:} علاوه بر استفاده‌ی فعلی از AI در درک کد، مسیرهای تحقیقاتی آینده باید بر توسعه و ارزیابی مدل‌های هوش مصنوعی برای خودکارسازی فرآیند مهندسی معکوس، استخراج منطق تجاری از کدهای قدیمی ، شناسایی الگوهای ضد طراحی و تولید خودکار مستندات فنی برای سیستم‌های \lr{Legacy} متمرکز شوند.
  \item \textbf{مدیریت پیچیدگی ابزارها و زیرساخت‌های DevOps:} یکی از موانع استقرار DevOps، پیچیدگی فنی ابزارهایی مانند Kubernetes و Terraform و بار آموزشی سنگین آن‌هاست. تحقیقات آتی می‌تواند بر \textbf{«ساده‌سازی تعامل»} با این ابزارها متمرکز باشد؛ چه از طریق توسعه‌ی \textbf{پلتفرم‌های سطح بالاتر (\lr{PaaS})} که به عنوان یک لایه‌ی انتزاعی عمل کرده و پیچیدگی‌های زیرساخت را از توسعه‌دهنده پنهان می‌کنند، و چه از طریق ایجاد \textbf{روش‌های مدیریتی هوشمندتر }و ابزارهای کمکی برای مدیریت بهینه‌ی خود این زیرساخت‌های پیچیده.
  \item \textbf{توسعه‌ی چارچوب‌های آموزشی و تحقیقاتی برای جنبه‌های انسانی DevOps:} با توجه به اینکه «مقاومت فرهنگی» یکی از مهم‌ترین موانع در استقرار موفق DevOps شناسایی شده است، یک خلاء تحقیقاتی و آموزشی آشکار وجود دارد. برنامه‌های آموزشی فعلی اغلب بیش از حد بر ابزارها متمرکز هستند. لذا، مسیرهای تحقیقاتی آینده باید بر توسعه و ارزیابی «مدل‌های مدیریت تغییر» و «تکنیک‌های روانشناسی سازمانی» متمرکز شوند که گذار فرهنگی به DevOps را تسهیل می‌کنند. همچنین، مسیرهای آموزشی آینده باید چارچوب‌هایی مدون برای آموزش مهارت‌های نرم (\lr{Soft Skills})، مانند ایجاد فرهنگ گزارش‌دهی بدون سرزنش (\lr{Blameless Postmortem}) و همکاری بین‌تیمی ، در کنار آموزش‌های فنی ارائه دهند.
  \item \textbf{الگوهای پیشرفته بازطراحی و مدیریت داده در مهاجرت:} با توجه به اهمیت حیاتی سیستم‌های قدیمی (مانند سیستم‌های \lr{Core Banking}) ، نیاز به الگوها و استراتژی‌های اثبات‌شده برای مدرن‌سازی آن‌ها وجود دارد. تحقیقات آینده می‌تواند بر توسعه‌ی مدل‌هایی برای ارزیابی دقیق ریسک، هزینه و زمان در سناریوهای مختلف بازطراحی ، و همچنین تحقیق بر روی الگوهای مدیریت سازگاری داده‌ها (مانند \lr{Saga} و \lr{Idempotency}) در طول مهاجرت افزایشی به معماری میکروسرویس تمرکز کند.
  \item \textbf{بهبود فرآیندها مبتنی بر داده‌های مانیتورینگ (\lr{AIOps}):} با گسترش ابزارهای نظارت و بازخورد مانند Prometheus و Sentry ، فرصت‌های جدیدی برای استفاده از هوش مصنوعی و تحلیل داده‌های عملیاتی جهت بهبود مستمر فرآیندهای توسعه و تصمیم‌گیری‌های مبتنی بر شاخص‌های کمی (مانند SLOs) فراهم آمده است که نیازمند تحقیق و توسعه‌ی بیشتر است.
\end{enumerate}
