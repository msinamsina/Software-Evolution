\section{تحلیل عمیق معماری و کارکرد ابزارهای IDA Pro و x64dbg}

در دنیای مهندسی معکوس و تحلیل بدافزار، دو ابزار IDA Pro و x64dbg به عنوان استانداردهای صنعتی شناخته می‌شوند. اگرچه هر دو ابزار برای درک عملکرد نرم‌افزار بدون دسترسی به کد منبع استفاده می‌شوند، اما فلسفه‌های طراحی و معماری داخلی آن‌ها کاملاً متفاوت است. IDA Pro در وهله اول یک ابزار \textit{«تحلیل ایستا»} (Static-First) است که بر پایه ایجاد یک مدل جامع و آفلاین از برنامه کار می‌کند. در مقابل، x64dbg یک ابزار \textit{«تحلیل دینامیک»} (Dynamic-First) است که بر روی دستکاری و نظارت بر اجرای زنده برنامه در حافظه تمرکز دارد \cite{stackexchange6911}. درک عمیق معماری این دو ابزار، برای ترکیب مؤثر آن‌ها در سناریوهای پیچیده ضروری است.

\subsection{معماری تحلیل ایستا در IDA Pro: فراتر از دیس‌اسمبلی ساده}

قدرت اصلی IDA Pro در موتور دیس‌اسمبلر پیشرفته آن نهفته است که از الگوریتم \textit{«تجزیه نزولی بازگشتی»} (Recursive Descent Disassembly) بهره می‌برد. برخلاف دیس‌اسمبلرهای خطی (Linear Sweep) مانند \texttt{objdump} که بایت‌ها را به صورت متوالی ترجمه می‌کنند، IDA تلاش می‌کند تا منطق اجرای برنامه را شبیه‌سازی کند.

\subsubsection{الگوریتم تجزیه نزولی بازگشتی}
این الگوریتم با فرض اینکه کد از نقطه ورود (Entry Point) آغاز می‌شود، دستورات را دنبال می‌کند. هنگامی که به دستورات تغییر جریان کنترل مانند \texttt{JMP} یا \texttt{CALL} می‌رسد، آدرس مقصد را به عنوان یک نقطه شروع جدید برای تحلیل در نظر می‌گیرد. این روش باعث می‌شود که IDA بتواند کدهای واقعی را از داده‌های جاسازی شده در بین کدها (مانند جداول رشته یا ساختارهای هم‌ترازسازی) تشخیص دهد \cite{devopsschool2025}.

با این حال، این روش در مواجهه با \textit{«پرش‌های غیرمستقیم»} (Indirect Jumps) که مقصد آن‌ها در زمان اجرا محاسبه می‌شود (مانند \texttt{JMP EAX})، دچار چالش می‌شود. IDA برای حل این مشکل از تکنیک‌های اکتشافی (Heuristics) پیشرفته و تحلیل جریان داده (Data Flow Analysis) استفاده می‌کند تا مقادیر احتمالی رجیسترها را حدس بزند و جداول پرش (Jump Tables) مربوط به ساختارهای \texttt{switch-case} را بازسازی کند \cite{rahulsingh_staticdynamic}.

\subsubsection{تکنولوژی FLIRT و شناسایی کتابخانه‌ها}
یکی دیگر از ویژگی‌های متمایز IDA، تکنولوژی \textit{«شناسایی سریع توابع کتابخانه‌ای»} (FLIRT) است. این سیستم با استفاده از امضاهای دیجیتال (Signatures)، توابع استاندارد کتابخانه‌ها (مانند \texttt{printf} یا \texttt{memcpy}) را شناسایی کرده و نام‌گذاری می‌کند. این امر باعث می‌شود تحلیلگر وقت خود را صرف تحلیل کدهای تکراری و شناخته‌شده نکند و مستقیماً بر روی منطق اختصاصی برنامه تمرکز نماید.

\subsection{دیکامپایلر Hex-Rays: بازسازی سطح بالا}

افزونه Hex-Rays Decompiler، خروجی دیس‌اسمبلی IDA را به یک نمایش سطح بالا شبیه به زبان C تبدیل می‌کند. این فرآیند شامل مراحل پیچیده‌ای است:
\begin{enumerate}
    \item \textbf{تولید میکروکد (Microcode Generation):} تبدیل دستورات اسمبلی وابسته به پردازنده به یک زبان میانی (IL) مستقل از پلتفرم.
    \item \textbf{بهینه‌سازی و تحلیل جریان داده:} حذف کدهای مرده، انتشار مقادیر ثابت و شناسایی متغیرهای پشته.
    \item \textbf{تحلیل ساختاری (Structural Analysis):} شناسایی الگوهای کنترلی سطح بالا مانند حلقه‌های \texttt{while}، شرط‌های \texttt{if-else} و بلوک‌های \texttt{try-catch}.
    \item \textbf{تولید کد C:} تبدیل درخت نحو مجرد (AST) نهایی به متن شبه‌کد C خوانا \cite{trailofbits2022}.
\end{enumerate}

\subsection{معماری دیباگر x64dbg: کنترل کامل در زمان اجرا}

ابزار x64dbg یک دیباگر سطح کاربر (User-Mode) برای ویندوز است که با هدف جایگزینی OllyDbg و پشتیبانی از معماری ۶۴ بیتی توسعه یافته است. هسته اصلی این ابزار بر پایه موتور \texttt{TitanEngine} بنا شده است که وظیفه تعامل با APIهای دیباگ ویندوز را بر عهده دارد.

\subsubsection{مدل چندنخی (Threading Model)}
برای جلوگیری از فریز شدن رابط کاربری در هنگام توقف برنامه هدف، x64dbg از یک معماری چندنخی استفاده می‌کند:
\begin{itemize}
    \item \textbf{نخ رابط کاربری (GUI Thread):} مسئول رسم پنجره‌ها و پاسخ به ورودی‌های کاربر است.
    \item \textbf{نخ دیباگ (Debug Thread):} این نخ در یک حلقه بی‌پایان، تابع \texttt{WaitForDebugEvent} را فراخوانی می‌کند و منتظر رخدادهایی مانند استثناها (Exceptions) یا نقاط توقف می‌ماند. تمام تعاملات با فرآیند هدف در این نخ انجام می‌شود \cite{elastic_hexrays}.
\end{itemize}

\subsubsection{انواع نقاط توقف (Breakpoints)}
x64dbg از مکانیزم‌های متنوعی برای توقف اجرا استفاده می‌کند:
\begin{itemize}
    \item \textbf{نقاط توقف نرم‌افزاری (INT 3):} با جایگزینی بایت اول دستور با \texttt{0xCC} کار می‌کند. تعداد نامحدود دارد اما کد را تغییر می‌دهد.
    \item \textbf{نقاط توقف سخت‌افزاری (Hardware Breakpoints):} از رجیسترهای دیباگ پردازنده (\texttt{DR0-DR7}) استفاده می‌کند. محدود به ۴ عدد است اما می‌تواند دسترسی (خواندن/نوشتن) به حافظه را نیز ردیابی کند که برای یافتن محل تغییر متغیرها بسیار حیاتی است \cite{hexrays_ida}.
    \item \textbf{نقاط توقف حافظه (Memory Breakpoints):} با تغییر مجوزهای صفحات حافظه (مانند \texttt{PAGE\_NOACCESS}) پیاده‌سازی می‌شود و برای ردیابی دسترسی به نواحی بزرگ حافظه کاربرد دارد.
\end{itemize}

\subsection{جریان کاری ترکیبی (Hybrid Workflow)}

تحلیلگران حرفه‌ای به ندرت تنها از یک ابزار استفاده می‌کنند. یک سناریوی رایج در تحلیل بدافزار به شرح زیر است:

\begin{enumerate}
    \item \textbf{تحلیل اولیه با IDA:} شناسایی ساختار کلی برنامه و یافتن نقاط مشکوک (مانند روتین‌های رمزگشایی).
    \item \textbf{دیباگ با x64dbg:} اجرای برنامه تا رسیدن به نقطه مشکوک و عبور از لایه‌های مبهم‌سازی (Unpacking) که تحلیل ایستا را غیرممکن کرده‌اند.
    \item \textbf{استخراج (Dumping):} کپی کردن کد رمزگشایی شده از حافظه به دیسک توسط x64dbg.
    \item \textbf{تحلیل مجدد با IDA:} بارگذاری فایل استخراج شده در IDA برای تحلیل دقیق منطق اصلی بدافزار که اکنون آشکار شده است \cite{oreilly_malware}.
\end{enumerate}

این چرخه نشان می‌دهد که چگونه IDA Pro (دید کلی و منطقی) و x64dbg (دید دقیق و لحظه‌ای) یکدیگر را تکمیل می‌کنند.

\begin{table}[h!]
\centering
\caption{مقایسه ویژگی‌های کلیدی IDA Pro و x64dbg}
\label{tab:ida-vs-x64dbg}
\begin{tabular}{|l|l|l|}
\hline
\textbf{ویژگی} & \textbf{IDA Pro} & \textbf{x64dbg} \\ \hline
\textbf{تمرکز اصلی} & تحلیل ایستا و دیس‌اسمبلی جامع & دیباگ پویا و دستکاری حافظه \\ \hline
\textbf{موتور تحلیل} & Recursive Descent & TitanEngine (Windows Debug API) \\ \hline
\textbf{نمایش کد} & گراف جریان کنترل (CFG) و شبه‌کد & نمای خطی اسمبلی و وضعیت رجیسترها \\ \hline
\textbf{قابلیت توسعه} & IDAPython و پلاگین‌های C++ & سیستم اسکریپت‌نویسی داخلی و پلاگین‌ها \\ \hline
\textbf{هزینه} & تجاری (گران‌قیمت) & متن‌باز و رایگان \\ \hline
\end{tabular}
\end{table}
