\section{7.7 نمونهٔ هک (مطالعهٔ موردی — شرح فنی و نحوهٔ کشف)}
\addcontentsline{toc}{section}{7.7 نمونهٔ هک (مطالعهٔ موردی — شرح فنی و نحوهٔ کشف)}
در این مطالعهٔ موردی، یک حملهٔ واقعی/شبیه‌سازی‌شده به‌صورت مرحله‌ای شرح داده می‌شود؛ تمرکز بر جزئیات فنی، چگونگی بهره‌برداری، و نحوهٔ کشف و تحلیل با ابزارهای دیباگ است.

\subsection{سناریو و پیش‌نیازها}
\begin{itemize}
  \item برنامه: یک REST API نوشته‌شده در جاوا که عملیات سریال‌سازی/دسریال‌سازی اشیا را انجام می‌دهد (مثلاً برای ذخیرهٔ سِشن یا ارسالِ پیام بین سرویس‌ها).
  \item محیط: Tomcat روی لینوکس، نسخهٔ قدیمیِ commons-collections در classpath، لاگِ سطح INFO موجود.
  \item ابزار مهاجم: ysoserial برای تولید gadget chain، netcat/curl برای ارسال payload، و یک سرور کنترل برای دریافت callback (outbound).
\end{itemize}

\subsection{شرح مرحله‌به‌مرحلهٔ حمله}
\begin{enumerate}
  \item 	extbf{شناسایی نقطهٔ ورودی:} مهاجم یک endpoint را که پارامتر cookie یا فیلدِ POST با محتوای باینری می‌پذیرد، شناسایی می‌کند.
  \item 	extbf{ساخت Payload:} با استفاده از ysoserial و یک gadget chain شناخته‌شده برای نسخهٔ آسیب‌پذیرِ commons-collections، payload تولید می‌شود که فرمانِ شل اجرا کند (مثلاً ایجاد یک reverse shell به 10.0.0.7:4444).
  \item 	extbf{ایجاد بستهٔ حمله:} payload به‌صورت BASE64 یا مستقیم در بدنهٔ درخواست فرستاده می‌شود:
  \begin{lstlisting}
  POST /api/deserialize HTTP/1.1
  Host: vulnerable.example
  Content-Type: application/octet-stream
  Content-Length: 512
  
  <binary serialized payload>
  \end{lstlisting}
  \item 	extbf{اجرای حمله و بهره‌برداری:} سرور payload را دسریالایز می‌کند؛ gadget chain موجب اجرای همان دستورِ تنظیم‌شده می‌شود و مهاجم دسترسیِ تعاملی به شل به‌دست می‌آورد.
\end{enumerate}

\subsection{چگونگیِ آشکارسازی توسط ابزارها}
\paragraph{1. DAST (Burp Suite + fuzzer):}
\begin{itemize}
  \item Burp Intruder با مجموعه‌ای از payloadهای باینری (نسخۀ ysoserial) به endpoint ارسال می‌کند؛ رفتار غیرعادی مانند افزایش زمان پاسخ یا بازشدن اتصال خروجی دیده می‌شود.
  \item قابل‌تشخیص: وجود رشته‌هایی در پاسخ HTML که محل اجرای دستور را لو می‌دهند یا headerهای غیرعادی.
\end{itemize}

\paragraph{2. لاگ‌ها و مانیتورینگ:}
\begin{itemize}
  \item لاگ‌های سرور نشان‌دهندهٔ فراخوانیِ کلاس‌های مشکوک هستند (مثلاً ورود کلاس org.apache.commons.collections.functors.InvokerTransformer).
  \item سیستم مانیتورینگ شبکه (IDS) هشدار برای اتصال خروجی غیرمعمول سرور به آدرس attacker را صادر می‌کند.
\end{itemize}

\paragraph{3. تحلیل داینامیک و دیباگینگ}
\begin{itemize}
  \item راه‌اندازی یک محیط staging که در آن با debugger (مثلاً attaching با jdb یا IDE) قبل از فراخوانی readObject یک breakpoint قرار می‌گیرد.
  \item هنگام هیت شدن breakpoint، مشاهدهٔ استک‌ترِس نشان می‌دهد که جریان از مسیر کلاسْ گجت عبور می‌کند؛ بررسی فیلدها و مقدارِ byte array محتوا را تأیید می‌کند.
  \item در سطح سیستم‌عامل، با strace یا auditd می‌توان خروجی‌های فراخوانی سیستم مانند fork/exec را دید که تأیید می‌کند یک فرایند جدید ایجاد شده است.
\end{itemize}

\subsection{نمونهٔ لاگِ تحلیلی (توضیحی)}
\begin{lstlisting}
[INFO] Received serialized payload, length=512
[WARN] Deserialization of untrusted data detected: class=org.apache.commons.collections.functors.InvokerTransformer
[ALERT] Outbound connection to 10.0.0.7:4444 established
[DEBUG] Stack trace (attached): java.lang.RuntimeException: ... -> org.apache.commons.collections...
\end{lstlisting}

\subsection{تجزیه و تحلیل فورنزیک و راهبردِ پاسخ}
\begin{itemize}
  \item 	extbf{ایزوله‌سازی سریع:} جداسازی سرویس از شبکه جهت جلوگیری از پخش آسیب.
  \item 	extbf{جمع‌آوری شواهد:} ذخیرهٔ core dump، لاگ‌های JVM، فهرستِ کلاس‌های بارگذاری‌شده و snapshot از حافظه JVM.
  \item 	extbf{ریمدییشن:} حذف یا به‌روزرسانی کتابخانهٔ آسیب‌پذیر، اجرای پچ، اعمال firewall rule برای جلوگیری از outbound به آدرسِ مهاجم.
  \item 	extbf{تست مجدد:} اجرای فازیینگ و اسکن‌های DAST روی محیط آپدیت‌شده برای تأیید رفع آسیب.
\end{itemize}

\subsection{درس‌های آموخته‌شده}
\begin{itemize}
  \item دسریالایز کردن اشیاء از ورودیِ غیرمطمئن ریسک بالایی دارد؛ در صورت نیاز، از فرمت‌های متنی امن و whitelist کلاس‌ها استفاده کنید.
  \item ترکیب لاگِ غنی، مانیتورینگِ شبکه و تست داینامیک شانس کشف زودهنگام را افزایش می‌دهد.
  \item داشتن Playbook پاسخ به حادثه و محیط staging برای بازتولید حمله بحرانی است.
\end{itemize}

\vspace{3mm}

\section{نتیجه‌گیری و جمع‌بندی}

ترکیب ابزارهای استاتیک و داینامیک، همراهِ فازیینگ و مانیتورینگ زمانِ اجرا، بهترین رویکرد برای کاهش ریسک است. رفع کاملِ مشکلات در همهٔ شرایط عملی نیست، بنابراین هدف باید حرکت به سمت کاهش ریسک و افزایش سرعت کشف و پاسخ باشد.

\section{ضمیمه: مثال‌هایی از قواعد Semgrep / Snippet}
\begin{lstlisting}[language=Java]
// Semgrep rule (simplified)
// look for use of readObject without validation
if pattern: |
  ObjectInputStream ois = new ObjectInputStream($X);
  ois.readObject();
message: "Possible insecure deserialization: validate input or avoid readObject"

actions:
  - look: $X
\end{lstlisting}

\vfill
\noindent \textit{پیشنهاد: برای چاپ و گرفتن PDF با پشتیبانی کاملِ فارسی از XeLaTeX استفاده کنید.}

\end{document}
