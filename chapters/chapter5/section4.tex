\section{۷.X تهیهٔ دیاگرام‌ها و مثال‌ها در مهندسی معکوس نرم‌افزار}
\addcontentsline{toc}{section}{۷.X تهیهٔ دیاگرام‌ها و مثال‌ها در مهندسی معکوس نرم‌افزار}

در این بخش، نقش دیاگرام‌ها، نمودارهای ساختاری، گراف‌های جریان کنترل و نمونه‌های استخراج‌شده از ابزارهای مهندسی معکوس مورد بررسی قرار می‌گیرد. هدف، تبیین این است که چگونه نمایش تصویری ساختار داخلی نرم‌افزار می‌تواند درک مهندس معکوس را به‌طور چشمگیری بهبود دهد، روند تحلیل را تسریع کند، و نقاط کلیدی اجرای برنامه را آشکار سازد. 

یکی از چالش‌های اصلی در تحلیل فایل‌های دودویی و کدهای کامپایل‌شده، نبود مستندات رسمی و فقدان دانش از طراحی اولیهٔ سیستم است. ابزارهای مهندسی معکوس با تولید دیاگرام‌هایی همچون \lr{Control Flow Graph (CFG)}، \lr{Call Graph}، و \lr{Data Flow Graph} کمک می‌کنند تا مهندس معکوس از سطح دستورهای اسمبلی به سطح ساختاری و منطقی برنامه حرکت کند.

\subsection{نقش دیاگرام‌ها در تحلیل ساختار برنامه}

\subsubsection{۱. گراف جریان کنترل (CFG)}
گراف جریان کنترل، یکی از بنیادی‌ترین خروجی‌های ابزارهایی مانند IDA Pro و Ghidra است. این گراف مسیرهای احتمالی اجرای برنامه را نمایش داده و نقاط حساس مانند حلقه‌ها، تصمیم‌گیری‌ها، و بلاک‌های بحرانی را نشان می‌دهد.
\begin{itemize}
    \item استفاده از CFG برای تحلیل رفتار بدافزارها.
    \item شناسایی بلاک‌هایی که احتمال وجود backdoor در آن‌ها زیاد است.
    \item استخراج مسیرهای اجرا برای تست فازیینگ هدفمند.
\end{itemize}

\subsubsection{۲. گراف فراخوانی توابع (Call Graph)}
این گراف نشان می‌دهد که چه توابعی توسط چه بخش‌هایی از برنامه فراخوانی می‌شوند. این مدل برای تحلیل سیستم‌های بزرگ و پیچیده ضروری است.
\begin{itemize}
    \item کشف وابستگی‌های پنهان بین ماژول‌ها.
    \item شناسایی توابعی که رفتار مخفی دارند و از نقاط غیرمنتظره فراخوانی می‌شوند.
    \item کمک به بازسازی معماری اصلی نرم‌افزار.
\end{itemize}

\subsubsection{۳. دیاگرام‌های سطح بالا}
برخی ابزارها (مانند Ghidra یا JADX در اندروید) می‌توانند شبه‌کد تولید کرده و دیاگرام‌های شبه‌معماری ارائه دهند.  
این نمودارها برای مستندسازی مجدد سیستم‌های قدیمی بسیار مفیدند.

\subsection{مثال‌های عملی استخراج‌شده از ابزارهای رایج}

\subsubsection{۱. تحلیل یک برنامهٔ ساده با کمک Ghidra}
در یک نمونهٔ واقعی، فایل اجرایی کوچکی مانند \lr{calc.exe} با Ghidra بارگذاری شده و ابزار به‌طور خودکار:
\begin{itemize}
    \item توابع قابل اجرا را شناسایی می‌کند،
    \item بدنهٔ آن‌ها را دی‌کامپایل کرده و نسخهٔ شبه‌کد ارائه می‌دهد،
    \item و گراف جریان کنترل هر تابع را تولید می‌کند.
\end{itemize}

در گزارش نهایی، دیاگرام‌ها به مهندس معکوس کمک می‌کنند که مسیرهای کلی اجرا را بدون نیاز به تحلیل تک‌به‌تک دستورهای اسمبلی شناسایی کند.



\subsubsection{۲. استخراج ساختار APK با JADX}
در تحلیل اپلیکیشن‌های اندروید، ابزار JADX قادر است:
\begin{itemize}
    \item ساختار پوشه‌ها و کلاس‌های دالویک را بازسازی کند،
    \item دیاگرام فراخوانی متدهای کلیدی را نمایش دهد،
    \item و رفتارهایی مانند دسترسی به شبکه یا فایل را برجسته کند.
\end{itemize}

برای نمونه، دیاگرام دسترسی شبکه‌ای یک بدافزار اندرویدی می‌تواند مسیر فراخوانی متدی که اطلاعات را به سرور مهاجم ارسال می‌کند نشان دهد.

\subsection{مراحل استاندارد تولید دیاگرام‌ها در فرایند مهندسی معکوس}

\subsubsection{۱. جداسازی بخش‌های مهم باینری}
ابزارها ابتدا بخش‌های قابل‌اجرای برنامه، جدول واردات/صادرات، رشته‌ها و منابع را شناسایی می‌کنند.

\subsubsection{۲. بازسازی توابع و تولید پویای گراف‌ها}
بر پایهٔ الگوهای فراخوانی و دستورهای پرش، بلاک‌های اساسی و سپس گراف جریان کنترل ساخته می‌شود.

\subsubsection{۳. تولید شبه‌کد و هم‌ترازی آن با نمودارها}
دیاگرام‌ها معمولاً همراه با شبه‌کد هستند تا بتوان ارتباط visual و متنی را درک کرد.

\subsubsection{۴. بازسازی معماری و تولید مستندات}
از روی دیاگرام‌ها می‌توان:
\begin{itemize}
    \item معماری لایه‌ای سیستم را بازسازی کرد،
    \item مسیرهای ورودی/خروجی داده را تعیین کرد،
    \item و رفتارهای پنهان را مستند نمود.
\end{itemize}

\subsection{اهمیت دیاگرام‌ها در پروژه‌های واقعی}

در سیستم‌های قدیمی که کد منبع در دسترس نیست، دیاگرام‌ها تقریباً تنها ابزار درک عملکرد داخلی هستند.  
در تحلیل بدافزار نیز بدون داشتن نمایش تصویری از جریان اجرا، شناسایی زنجیرهٔ حمله و رفتار مخرب بسیار دشوار خواهد بود.

\subsubsection{جمع‌بندی}
دیاگرام‌ها و مثال‌های مهندسی معکوس، بخش جدایی‌ناپذیری از تحلیل نرم‌افزارهای پیچیده‌اند. این نمایش‌های تصویری، مسیرهای منطقی و ساختاری برنامه را آشکار کرده و امکان می‌دهند که بدون فهم کامل اسمبلی، ساختار درونی نرم‌افزار فهمیده شود. ترکیب شبه‌کد، گراف‌ها و دیاگرام‌های سطح بالا، سریع‌ترین و دقیق‌ترین راه برای بازسازی منطق یک برنامه و مستندسازی رفتارهای آن است.
