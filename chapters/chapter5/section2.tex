\section{دلایل نیاز به مهندسی معکوس}

\subsection{فقدان مستندات یا مستندات ناقص}
بسیاری از سیستم‌های نرم‌افزاری بدون مستندات کافی توسعه یافته‌اند یا مستندات آن‌ها در طول زمان از بین رفته است \cite{pressman2020software}. در این شرایط، مهندسی معکوس به تیم توسعه کمک می‌کند تا از روی نرم‌افزار، مستندات طراحی و نمودارهای سیستم را بازسازی کند.

\subsection{تحلیل سیستم‌های قدیمی (\lr{Legacy Systems})}
در سازمان‌ها هنوز از سیستم‌هایی استفاده می‌شود که بر پایه فناوری‌های قدیمی ساخته شده‌اند. مهندسی معکوس به توسعه‌دهندگان کمک می‌کند تا ساختار کلی این سیستم‌ها را درک کنند و در صورت نیاز آن‌ها را به فناوری‌های جدید منتقل نمایند.

\subsection{درک ساختار و منطق سیستم‌های موجود}
گاهی نرم‌افزار توسط تیم‌های مختلف توسعه یافته و در نتیجه کدها پیچیده و نامنظم شده‌اند. مهندسی معکوس ابزاری برای درک ارتباط بین ماژول‌ها، کلاس‌ها و داده‌ها فراهم می‌کند و درک درستی از منطق سیستم به تیم توسعه می‌دهد.

\subsection{تسهیل مهاجرت به فناوری‌های جدید}
تغییر پلتفرم‌ها و ابزارها اجتناب‌ناپذیر است. برای مثال، ممکن است سازمانی بخواهد نرم‌افزار خود را از نسخه‌ی دسکتاپ به تحت وب منتقل کند. مهندسی معکوس امکان تحلیل دقیق سیستم فعلی را فراهم می‌کند تا مهاجرت بدون خطا و از دست دادن داده انجام گیرد \cite{pressman2020software}.
