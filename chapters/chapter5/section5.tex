

در این فصل، به بررسی چالش‌ها و محدودیت‌های موجود در به‌کارگیری رویکرد \en{DevOps} در کنار مهندسی معکوس نرم‌افزار پرداخته می‌شود. هدف از این بخش، شناسایی عواملی است که می‌توانند بر اثربخشی، اعتبار و قابلیت تعمیم نتایج حاصل از اجرای این رویکردها تاثیرگذار باشند. محدودیت‌های شناسایی‌شده در چهار محور اصلی شامل مسائل حقوقی و مالکیت فکری، هزینه و زمان‌بر بودن فرآیند، تفسیر نادرست منطق کد، و ریسک‌های امنیتی مورد تحلیل قرار گرفته‌اند. هر یک از این عوامل، به‌صورت مستقیم یا غیرمستقیم می‌توانند مانعی در مسیر پیاده‌سازی مؤثر \en{DevOps} و بازمهندسی سیستم‌های نرم‌افزاری در محیط‌های واقعی ایجاد کنند و ضرورت به‌کارگیری رویکردی میان‌رشته‌ای و تصمیم‌گیری دقیق در این زمینه را نشان می‌دهند.

\subsection*{مشکلات حقوقی و مالکیت فکری}
مهندسی معکوس نرم‌افزار معمولاً با محدودیت‌های قانونی و حقوقی گسترده‌ای همراه است. بسیاری از سیستم‌های نرم‌افزاری موجود در شرکت‌های بزرگ، تحت مجوزهای اختصاصی، قراردادهای توسعه یا توافق‌نامه‌های عدم افشا (\en{NDA}) طراحی و نگهداری می‌شوند. در چنین شرایطی، هرگونه تلاش برای تحلیل معکوس، استخراج کد منبع یا بازطراحی اجزای نرم‌افزار می‌تواند نقض حق مالکیت فکری محسوب شود \cite{nrc1991}. به‌ویژه در کشورهایی با نظام حقوقی سختگیرانه مانند ایالات متحده، قوانین کپی‌رایت و پتنت می‌توانند مانع هرگونه مهندسی معکوس حتی با هدف بهبود سازگاری یا پایداری سیستم شوند \cite{ttc2024}.
از سوی دیگر، محدودیت‌های بین‌المللی نیز موجب پیچیدگی بیشتر می‌شوند. در محیط‌های چندملیتی، تعریف و اجرای حقوق مالکیت نرم‌افزار می‌تواند میان کشورها متفاوت باشد و در نتیجه، دسترسی به کد یا داده‌های واقعی جهت تحلیل علمی محدود گردد \cite{quarkslab2023}. 
همچنین، تضاد میان نوآوری و حفاظت از مالکیت فکری چالشی فلسفی در این حوزه ایجاد کرده است. برخی محققان معتقدند که قوانین سختگیرانه‌ی \en{IP}، پیشرفت فناوری را کند می‌کنند زیرا مانع از یادگیری از سیستم‌های پیشین می‌شوند \cite{ipwatchdog2021}. در مقابل، گروهی دیگر بر این باورند که مهندسی معکوس بی‌رویه بدون رعایت مجوزها، ریسک سرقت فناوری و نقض حقوق مولف را افزایش می‌دهد. در این تحقیق نیز، به دلیل عدم دسترسی به داده‌های واقعی شرکت‌ها و محدودیت حقوقی تحلیل نمونه‌های صنعتی، بررسی‌های تجربی محدود به جنبه‌های نظری باقی مانده است.

\subsection*{هزینه و زمان‌بر بودن}
از منظر اقتصادی و اجرایی، بازمهندسی نرم‌افزار فرآیندی پرهزینه و زمان‌بر است که نیازمند منابع انسانی، زیرساختی و مالی قابل‌توجهی است \cite{recordpoint2024}. در گزارش‌های صنعتی آمده است که شرکت‌ها سالانه میلیاردها دلار صرف نگهداری و بازطراحی سیستم‌های قدیمی خود می‌کنند و در بسیاری از موارد، هزینه‌ی بازمهندسی از هزینه‌ی توسعه‌ی مجدد سیستم جدید نیز بیشتر است \cite{vfunction2022}. 
عوامل متعددی در افزایش این هزینه موثرند. برای نمونه، نبود مستندات کافی، نیاز به تحلیل وابستگی‌ها، بازسازی مدل داده‌ها، باز طراحی معماری و آزمون‌های مکرر همه باعث افزایش زمان اجرای پروژه می‌شوند. هرچه ساختار کد قدیمی‌تر و پیچیده‌تر باشد، زمان تحلیل و اصلاح آن به‌صورت تصاعدی افزایش می‌یابد \cite{modlogix2022}. 
به‌علاوه، یکی از مشکلات رایج در پروژه‌های بازمهندسی، برآورد نادرست هزینه و زمان است. بسیاری از سازمان‌ها در آغاز پروژه تخمین دقیقی از میزان بدهی فنی و سطح ناسازگاری فناوری ندارند؛ در نتیجه، با افزایش غیرمنتظره‌ی هزینه‌ها، پروژه در میانه‌ی راه متوقف یا محدود می‌شود \cite{devsquad2025}. 

\subsection*{تفسیر نادرست از منطق کد}
در فرآیند بازمهندسی، درک صحیح از منطق درونی نرم‌افزار اهمیت حیاتی دارد. با این حال، بسیاری از سیستم‌های میراثی فاقد مستندات کامل هستند و مهندسان مجبورند رفتار سیستم را تنها از طریق تحلیل کد استنباط کنند. این امر منجر به تفسیر نادرست از منطق برنامه و روابط میان اجزای آن می‌شود \cite{medium2023}.
مطالعات نشان می‌دهند که درصد قابل‌توجهی از خطاهای به‌وجود‌آمده پس از بازمهندسی، ناشی از برداشت اشتباه از منطق کسب‌وکار و وابستگی‌های داخلی سیستم است \cite{modelcode2024}. 
علاوه بر این، خروج نیروهای کلیدی از سازمان و از بین رفتن دانش ضمنی باعث می‌شود که درک دقیق از چرایی و چگونگی تصمیمات گذشته از بین برود \cite{joiv2023}. ابزارهای خودکار تحلیل معنایی و مدل‌سازی معکوس هنوز در بسیاری از محیط‌های صنعتی به‌طور کامل توسعه نیافته‌اند، و این امر احتمال سوء‌برداشت از منطق کد را بیشتر می‌کند.

\subsection*{ریسک‌های امنیتی}
ریسک‌های امنیتی از مهم‌ترین موانع در مسیر بازمهندسی و بازطراحی نرم‌افزار محسوب می‌شوند. بسیاری از سیستم‌های قدیمی بر پایه‌ی فناوری‌ها و چارچوب‌هایی بنا شده‌اند که دیگر به‌روزرسانی نمی‌شوند \cite{integrity2023}. این وضعیت باعث می‌شود که آسیب‌پذیری‌های شناخته‌شده برای مدت طولانی در سیستم باقی بمانند و مهاجمان بتوانند از آن‌ها سوء‌استفاده کنند \cite{herodevs2024}. 
در فرآیند بازمهندسی، این خطر وجود دارد که مهاجرت داده‌ها، تقسیم ماژول‌ها یا اتصال سیستم‌های جدید با سیستم‌های قدیمی، مسیرهای جدیدی برای نفوذ ایجاد کند. همچنین، اگر فرآیند \en{DevOps} به‌درستی با الزامات امنیتی یکپارچه نشود، اتوماسیون نادرست می‌تواند دروازه‌هایی برای نفوذ به محیط تولید باز کند \cite{atiba2025}. 
یکی دیگر از چالش‌های امنیتی، ضعف در مدیریت وصله‌های امنیتی است. پژوهش \en{Dissanayake} و همکاران \cite{dissanayake2020} نشان می‌دهد که کمتر از ۲۰ درصد از سازمان‌ها فرآیند وصله‌گذاری خود را به‌صورت نظام‌مند و خودکار انجام می‌دهند، که این امر احتمال بروز آسیب‌پذیری در چرخه‌ی بازمهندسی را افزایش می‌دهد.

